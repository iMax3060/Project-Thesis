\newenvironment{myabstract}
    {
        \KOMAoptions{DIV = last}
        \begin{abstract}
            \thispagestyle{plain}
            \small
    }{
        \end{abstract}
    }

\begin{myabstract}
    Needless to say, any database management system (\textbf{DBMS}) must be able to manage data. The data structures used to manage this data in a database (\textbf{DB}) have a great influence on various characteristics (e.g. performance) of a DBMS and therefore the use of certain data structures (e.g. B-tree indexes) and even some implementation details of them are very important decisions in DBMS design.

    While (re-)implementing and refactoring the buffer management of my DBMS testbed \emph{Zero}\footnote{\url{https://github.com/iMax3060/zero}}, I came across a number of less performance-critical data structures and algorithms that I wanted to replace. But since even these less performance-critical submodules can become performance-critical in certain situations, one should choose a reasonable implementation. Therefore, it is appropriate to use open source libraries that implement the required algorithms.

    While the hash table for finding potentially buffered DB pages in the buffer pool is performance-critical, the free list of the buffer pool, which is used to manage free buffer frames, is not performance-critical. Some open source implementations of concurrent queues that can be used to manage the buffer pool's free list are evaluated in chapter \ref{ch:free-list}.

    For my master's thesis I am evaluating a number of page eviction strategies for the buffer manager. The most important performance metric of a page eviction strategy is the hit rate achieved under a given workload. But this performance metric does not change for different implementations of the RANDOM and LOOP page evictions strategies. Different implementations of these page eviction strategies change the overhead they cause during the already expensive buffer pool page misses. In chapter \ref{ch:random} I compare an enormous number of different pseudorandom number generators (\textbf{PRNG}s) available in open source libraries and in chapter \ref{ch:loop} I compare some concurrent counters that can be used for the LOOP page displacement strategy.
\end{myabstract}