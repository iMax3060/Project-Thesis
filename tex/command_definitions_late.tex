%%%%%%%%%%%%%%%%%%%%%%%%%%%%%%%%%%%%%%%%%%%%%%%
%%%%%%%%%%%%%%%%%%%%%%%%%%%%%%%%%%%%%%%%%%%%%%%
%%%%%%%%%%%%%%%%%%%%%%%%%%%%%%%%%%%%%%%%%%%%%%%
% Defines a shape 'square' for tikz that behaves like a square :-):
%%%%%%%%%%%%%%%%%%%%%%%%%%%%%%%%%%%%%%%%%%%%%%% begindefinition
\makeatletter
% the contents of \squarecorner were mostly stolen from pgfmoduleshapes.code.tex
\def\squarecorner#1{
    % Calculate x
    %
    % First, is width < minimum width?
    \pgf@x=\the\wd\pgfnodeparttextbox%
    \pgfmathsetlength\pgf@xc{\pgfkeysvalueof{/pgf/inner xsep}}%
    \advance\pgf@x by 2\pgf@xc%
    \pgfmathsetlength\pgf@xb{\pgfkeysvalueof{/pgf/minimum width}}%
    \ifdim\pgf@x<\pgf@xb%
        % yes, too small. Enlarge...
        \pgf@x=\pgf@xb%
    \fi%
    % Calculate y
    %
    % First, is height+depth < minimum height?
    \pgf@y=\ht\pgfnodeparttextbox%
    \advance\pgf@y by\dp\pgfnodeparttextbox%
    \pgfmathsetlength\pgf@yc{\pgfkeysvalueof{/pgf/inner ysep}}%
    \advance\pgf@y by 2\pgf@yc%
    \pgfmathsetlength\pgf@yb{\pgfkeysvalueof{/pgf/minimum height}}%
    \ifdim\pgf@y<\pgf@yb%
        % yes, too small. Enlarge...
        \pgf@y=\pgf@yb%
    \fi%
    %
    % this \ifdim is the actual part that makes the node dimensions square.
    \ifdim\pgf@x<\pgf@y%
        \pgf@x=\pgf@y%
    \else
        \pgf@y=\pgf@x%
    \fi
    %
    % Now, calculate right border: .5\wd\pgfnodeparttextbox + .5 \pgf@x + #1outer sep
    \pgf@x=#1.5\pgf@x%
    \advance\pgf@x by.5\wd\pgfnodeparttextbox%
    \pgfmathsetlength\pgf@xa{\pgfkeysvalueof{/pgf/outer xsep}}%
    \advance\pgf@x by#1\pgf@xa%
    % Now, calculate upper border: .5\ht-.5\dp + .5 \pgf@y + #1outer sep
    \pgf@y=#1.5\pgf@y%
    \advance\pgf@y by-.5\dp\pgfnodeparttextbox%
    \advance\pgf@y by.5\ht\pgfnodeparttextbox%
    \pgfmathsetlength\pgf@ya{\pgfkeysvalueof{/pgf/outer ysep}}%
    \advance\pgf@y by#1\pgf@ya%
}
\makeatother

\pgfdeclareshape{simplesquare}{
    \savedanchor\northeast{\squarecorner{}}
    \savedanchor\southwest{\squarecorner{-}}

    \foreach \x in {east,west} \foreach \y in {north,mid,base,south} {
        \inheritanchor[from=rectangle]{\y\space\x}
    }
    \foreach \x in {east,west,north,mid,base,south,center,text} {
        \inheritanchor[from=rectangle]{\x}
    }
    \inheritanchorborder[from=rectangle]
    \inheritbackgroundpath[from=rectangle]
}
%%%%%%%%%%%%%%%%%%%%%%%%%%%%%%%%%%%%%%%%%%%%%%% enddefinition

%%%%%%%%%%%%%%%%%%%%%%%%%%%%%%%%%%%%%%%%%%%%%%%
%%%%%%%%%%%%%%%%%%%%%%%%%%%%%%%%%%%%%%%%%%%%%%%
%%%%%%%%%%%%%%%%%%%%%%%%%%%%%%%%%%%%%%%%%%%%%%%
% Adds 'west north west', 'east north east', 'east south east', 'north north west', 
% 'south south west', 'south south east' to the tikz shape 'rectangle':
%%%%%%%%%%%%%%%%%%%%%%%%%%%%%%%%%%%%%%%%%%%%%%% begindefinition
\makeatletter
\pgfdeclareshape{square}{
  \inheritsavedanchors[from=simplesquare]
  \inheritanchorborder[from=simplesquare]
  \foreach \a in {%
      center,mid,base,north,south,west,east,%
      north west,mid west,base west,south west,%
      north east,mid east,base east,south east%
    }{\inheritanchor[from=simplesquare]{\a}}
  \inheritbackgroundpath[from=simplesquare]
  \anchor{north 1/3}{
    \southwest\pgf@xa=\pgf@x
    \northeast\pgfmathsetlength\pgf@x{\pgf@xa-(\pgf@xa-\pgf@x)/3}
  }
  \anchor{north 2/3}{
    \southwest\pgf@xa=\pgf@x
    \northeast\pgfmathsetlength\pgf@x{\pgf@x-(\pgf@x-\pgf@xa)/3}
  }
  \anchor{south 1/3}{
    \northeast\pgf@xa=\pgf@x
    \southwest\pgfmathsetlength\pgf@x{\pgf@x-(\pgf@x-\pgf@xa)/3}
  }
  \anchor{south 2/3}{
    \northeast\pgf@xa=\pgf@x
    \southwest\pgfmathsetlength\pgf@x{\pgf@xa-(\pgf@xa-\pgf@x)/3}
  }
  \anchor{east 1/3}{
    \southwest\pgf@ya=\pgf@y
    \northeast\pgfmathsetlength\pgf@y{\pgf@ya-(\pgf@ya-\pgf@y)/3}
  }
  \anchor{east 2/3}{
    \southwest\pgf@ya=\pgf@y
    \northeast\pgfmathsetlength\pgf@y{\pgf@y-(\pgf@y-\pgf@ya)/3}
  }
  \anchor{west 1/3}{
    \northeast\pgf@ya=\pgf@y
    \southwest\pgfmathsetlength\pgf@y{\pgf@y-(\pgf@y-\pgf@ya)/3}
  }
  \anchor{west 2/3}{
    \northeast\pgf@ya=\pgf@y
    \southwest\pgfmathsetlength\pgf@y{\pgf@ya-(\pgf@ya-\pgf@y)/3}
  }
}
\makeatother
%%%%%%%%%%%%%%%%%%%%%%%%%%%%%%%%%%%%%%%%%%%%%%% enddefinition

%%%%%%%%%%%%%%%%%%%%%%%%%%%%%%%%%%%%%%%%%%%%%%%
%%%%%%%%%%%%%%%%%%%%%%%%%%%%%%%%%%%%%%%%%%%%%%%
%%%%%%%%%%%%%%%%%%%%%%%%%%%%%%%%%%%%%%%%%%%%%%%
% Redefines the bibliography entry for inproceedings to allow ISSN and/or ISBN:
%%%%%%%%%%%%%%%%%%%%%%%%%%%%%%%%%%%%%%%%%%%%%%% begindefinition
\DeclareBibliographyDriver{inproceedings}{%
  \usebibmacro{bibindex}%
  \usebibmacro{begentry}%
  \usebibmacro{author/translator+others}%
  \setunit{\printdelim{nametitledelim}}\newblock
  \usebibmacro{title}%
  \newunit
  \printlist{language}%
  \newunit\newblock
  \usebibmacro{byauthor}%
  \newunit\newblock
  \usebibmacro{in:}%
  \usebibmacro{maintitle+booktitle}%
  \newunit\newblock
  \usebibmacro{event+venue+date}%
  \newunit\newblock
  \usebibmacro{byeditor+others}%
  \newunit\newblock
  \iffieldundef{maintitle}
    {\printfield{volume}%
     \printfield{part}}
    {}%
  \newunit
  \printfield{volumes}%
  \newunit\newblock
  \usebibmacro{series+number}%
  \newunit\newblock
  \printfield{note}%
  \newunit\newblock
  \printlist{organization}%
  \newunit
  \usebibmacro{publisher+location+date}%
  \newunit\newblock
  \usebibmacro{chapter+pages}%
  \newunit\newblock
  \iftoggle{bbx:isbn}
    {\printfield{isbn}%
     \newunit\newblock
     \printfield{issn}}
    {}%
  \newunit\newblock
  \usebibmacro{doi+eprint+url}%
  \newunit\newblock
  \usebibmacro{addendum+pubstate}%
  \setunit{\bibpagerefpunct}\newblock
  \usebibmacro{pageref}%
  \newunit\newblock
  \iftoggle{bbx:related}
    {\usebibmacro{related:init}%
     \usebibmacro{related}}
    {}%
  \usebibmacro{finentry}}
%%%%%%%%%%%%%%%%%%%%%%%%%%%%%%%%%%%%%%%%%%%%%%% enddefinition

%%%%%%%%%%%%%%%%%%%%%%%%%%%%%%%%%%%%%%%%%%%%%%%
%%%%%%%%%%%%%%%%%%%%%%%%%%%%%%%%%%%%%%%%%%%%%%%
%%%%%%%%%%%%%%%%%%%%%%%%%%%%%%%%%%%%%%%%%%%%%%%
% Redefines the bibliography field definitions for url and doi to reduce their font size:
%%%%%%%%%%%%%%%%%%%%%%%%%%%%%%%%%%%%%%%%%%%%%%% begindefinition
\DeclareFieldFormat{url}{\mkbibacro{URL}\addcolon\space\footnotesize\url{#1}}
\DeclareFieldFormat{doi}{%
  \mkbibacro{DOI}\addcolon\space\footnotesize
  \ifhyperref
    {\href{https://doi.org/#1}{\nolinkurl{#1}}}
    {\nolinkurl{#1}}}
%%%%%%%%%%%%%%%%%%%%%%%%%%%%%%%%%%%%%%%%%%%%%%% enddefinition

%%%%%%%%%%%%%%%%%%%%%%%%%%%%%%%%%%%%%%%%%%%%%%%
%%%%%%%%%%%%%%%%%%%%%%%%%%%%%%%%%%%%%%%%%%%%%%%
%%%%%%%%%%%%%%%%%%%%%%%%%%%%%%%%%%%%%%%%%%%%%%%
% Allows the positioning of nodes on a circle:
%%%%%%%%%%%%%%%%%%%%%%%%%%%%%%%%%%%%%%%%%%%%%%% begindefinition
\usepackage{tikz}
\usetikzlibrary{chains}
\tikzset{
  nodes around center/.style args={#1:#2:#3:#4}{%
    % #1 = Startwinkel,   #2 = Anzahl Knoten
    % #3 = Zentrums-Node, #4 = Abstand
    at={([shift={(#3)}] {{(\tikzchaincount-1)*360/(#2)+#1}}:{#4})}
  },
  nodes around center*/.style args={#1:#2:#3:#4}{% gleiche Optionen wie oben
    at={([shift={(#3.{(\tikzchaincount-1)*360/(#2)+#1})}] {{(\tikzchaincount-1)*360/(#2)+#1}}:{#4})},
    anchor={(\tikzchaincount-1)*360/(#2)+#1+180}
  }
}
%%%%%%%%%%%%%%%%%%%%%%%%%%%%%%%%%%%%%%%%%%%%%%% enddefinition
