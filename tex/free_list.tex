\chapter[Buffer Frame Free List]{Buffer Frame Free List} \label{ch:free-list}

\section[Purpose]{Purpose}

	A buffer manager is required for every disk-based DBMS. A disk-based DBMS stores the pages of a database on secondary storage but to read and write pages, they are required to be in memory.
	
	This feature is provided by the buffer pool management by managing the currently used subset of the database pages in buffer frames located in memory. A buffer frame is a portion of memory that can hold one database page and each of those frames got a frame index as identifier.
	
	During operation, database pages are dynamically fetched from the database into buffer frames. Once a page is not required anymore, it might be evicted from the buffer pool freeing a buffer frame.
	
	Due to the fact that pages are only allowed to be fetched into free buffer frames, the buffer manager needs to know all the free buffer frames. Therefore, a free list for the buffer frames is required.

\section[Compared Queue Implementations]{Compared Queue Implementations}

	To ease implementation of page eviction strategies like CLOCK, a free list should use a FIFO data structure like a queue. Therefore the buffer frame freed first is (re-)used first as well.
	
	Almost every state-of-the-art DBMS supports multithreading and therefore, there are usually multiple threads concurrently fetching pages into the buffer pool and evicting pages from the buffer pool. Following this, a buffer frame free list has to support thread-safe functions to add frame indexes to the free list and to retrieve/remove frame indexes from it. Queues providing those thread-safe access functions are usually called multi-producer (add frame indexes) multi-consumer (retrieve/remove frame indexes) queues.
	
	An approximate number of buffer indexes the free list must also be provided by any free list implementation to support the eviction of pages once there are only a few free buffer frames left. Thread-safe access to this number is desirable but not absolutely required.

\subsection[Boost Lock-Free Queue with variable size]{Boost Lock-Free Queue with variable size} \label{subsec:boost}

	The famous \textit{Boost C++ Libraries} offer a lock-free MPMC queue in the library \lstinline{Boost.Lockfree}. Like most other non-blocking data structures, this MPMC queue uses atomic operations instead of locks or mutexes. To support queues of dynamically changing sizes, this queue implementation also uses a free list internally.
	
	The non-thread-safe construction/destruction of the data structure is no limitation for the purpose as a buffer frame free list because the buffer pool of our prototype system is constructed single-threaded. This data structure does not offer the number of contained elements and therefore, an approximate number of buffer indexes the free list needs to be managed outside.

\subsection[Boost Lock-Free Queue with fixed size]{Boost Lock-Free Queue with fixed size} \label{subsec:boost-fixed}

	This data structure is identical to the data structure in Subsection \ref{subsec:boost} but does not use dynamic memory management internally. Therefore, the capacity of the queue (i.e. the maximum number of buffer frames of the buffer pool) needs to be specified beforehand which allows the usage of a fixed-size array instead of dynamically allocated nodes.

\subsection[CDS BasketQueue]{CDS Basket Lock-Free Queue} \label{subsec:cds-basket}

	\cite{Hoffman:2007}

\subsection[CDS FCQueue]{CDS Flat-Combining Lock-Free Queue} \label{subsec:cds-fc}

\subsection[CDS MSQueue]{CDS Michael \& Scott Lock-Free Queue} \label{subsec:cds-ms}

\subsection[CDS MoirQueue]{CDS Variation of Michael \& Scott Lock-Free Queue} \label{subsec:cds-moir}

\subsection[CDS OptimisticQueue]{CDS Ladan-Mozes \& Shavit Optimistic Queue} \label{subsec:cds-optimistic}

\subsection[CDS RWQueue]{CDS Michael \& Scott Blocking Queue with Fine-Grained Locking} \label{subsec:cds-rw}

\subsection[CDS SegmentedQueue]{CDS Segmented Queue} \label{subsec:cds-segmented}

\subsection[CDS VyukovMPMCCycleQueue]{CDS Vyukov's MPMC Bounded Queue} \label{subsec:cds-vyukovmpmccycle}


\section[Performance Evaluation]{Performance Evaluation}