\chapter[LOOP Page Eviction]{LOOP Page Eviction} \label{ch:loop}

\section[Purpose]{Purpose}

    The LOOP page eviction algorithm is the simplest form of the RANDOM page eviction strategy where the eviction candidates are selected in a round-robin way based on their buffer frame index.

    The LOOP eviction strategy uses a ``pseudo-random'' number generator---basically a CBRNG as in subsection \ref{subsec:cbrng}, but instead of using a complex mapping function, the identity function modulo the number of buffer pool frames is used to generate the pseudo-random numbers---which generates an ordered sequence of buffer indices.

\section[Compared Counter Implementations]{Compared Counter Implementations}

    Six different counter implementations were evaluated for the use in a LOOP page eviction algorithm. They can be grouped as follows:

\begin{@empty}
    \begin{itemize}
        \itemsep0em
        \item blocking counters
              \begin{itemize}
              \itemsep0em
                  \item mutex counter
                  \item spinlock counter
              \end{itemize}
        \item non-blocking counters
              \begin{itemize}
                  \itemsep0em
                  \item modulo counter
                  \item lock-free counter
              \end{itemize}
        \item local counters
              \begin{itemize}
                  \itemsep0em
                  \item local counter
                  \item local modulo counter
              \end{itemize}
    \end{itemize}
\end{@empty}

    The counters return values from $1$ to $\texttt{blockCount} - 1$---the range of the buffer frame indices.

\subsection[Mutex Counter]{Mutex Counter} \label{subsec:mutex_counter}

    The \emph{mutex counter} is a blocking counter which uses one global counter---used to select candidates for page eviction---synchronized with a mutex called via the \lstinline{std::mutex} interface.

\begin{@empty}
    \lstset{
        language = [ISO]C++
    }
\begin{centeredshadowboxlisting}
inline uint32_t mutexCounter() {
    static std::mutex indexLock;
    static uint32_t lastIndex = 0;
    std::lock_guard<std::mutex> guard(indexLock);
    lastIndex++;
    if (lastIndex >= blockCount) {
        lastIndex = 1;
    }
    return lastIndex;
}
\end{centeredshadowboxlisting}
\end{@empty}

\subsection[Spinlock Counter]{Spinlock Counter} \label{subsec:spinlock_counter}

    The \emph{spinlock counter} is also a blocking counter that uses one global counter, but it is protected by a spinlock implemented using a \lstinline{std::atomic_flag}.

\begin{@empty}
    \lstset{
        language = [ISO]C++
    }
\begin{centeredshadowboxlisting}
inline uint32_t spinlockCounter() {
    static std::atomic_flag indexLock =
        ATOMIC_FLAG_INIT;
    static uint32_t lastIndex = 0;
    uint32_t newIndex;
    while (indexLock.test_and_set(
               std::memory_order_acquire)) {}
    lastIndex++;
    if (lastIndex >= blockCount) {
        lastIndex = 1;
    }
    newIndex = lastIndex;
    indexLock.clear(std::memory_order_release);
    return newIndex;
}
\end{centeredshadowboxlisting}
\end{@empty}

\subsection[Modulo Counter]{Modulo Counter} \label{subsec:modulo_counter}

    The \emph{modulo counter} is a non-blocking counter that uses atomic increment operations on a global counter (of type \lstinline{std::atomic<uint64_t>}). The modulo is calculated thread-local and, therefore, the value of the global counter is strictly increasing.

\begin{@empty}
    \lstset{
        language = [ISO]C++
    }
\begin{centeredshadowboxlisting}
inline uint32_t moduloCounter() {
    static std::atomic<uint64_t> lastIndex = 0;
    static uint32_t moduloDivisor = blockCount
                                  - 1;
    return (lastIndex++ % moduloDivisor) + 1;
}
\end{centeredshadowboxlisting}
\end{@empty}

\subsection[Local Counter]{Local Counter} \label{subsec:local_counter}

    The \emph{local counter} is---obviously---a local counter, where each thread that performs page evictions uses its own circular counter.

\begin{@empty}
    \lstset{
        language = [ISO]C++
    }
\begin{centeredshadowboxlisting}
inline uint32_t localCounter() {
    static thread_local uint32_t lastIndex = 0;
    lastIndex++;
    if (lastIndex >= blockCount) {
        lastIndex = 1;
    }
    return lastIndex;
}
\end{centeredshadowboxlisting}
\end{@empty}

\subsection[Local Modulo Counter]{Local Modulo Counter} \label{subsec:local_modulo_counter}

    The \emph{local modulo counter} is a local counter where the circular counting is not achieved with a branch operation, but with a possibly cheaper modulo operation calculated during each call.

\begin{@empty}
    \lstset{
        language = [ISO]C++
    }
\begin{centeredshadowboxlisting}
inline uint32_t localModuloCounter() {
    static thread_local uint64_t lastIndex = 0;
    static uint32_t moduloDivisor = blockCount
                                  - 1;
    return (lastIndex++ % moduloDivisor) + 1;
}
\end{centeredshadowboxlisting}
\end{@empty}

\subsection[Lock-Free Counter]{Lock-Free Counter} \label{subsec:clunky_counter}

    The \emph{lock-free counter} is a non-blocking counter which uses atomic increment operations on a global counter (of type \lstinline{std::atomic<uint32_t>}) and algorithm-specific synchronization to achieve circular counting using branch operations.

\begin{@empty}
    \lstset{
        language = [ISO]C++
    }
\begin{centeredshadowboxlisting}
inline uint32_t lockFreeCounter() {
    static std::atomic<uint32_t> newIndex(1);
    uint32_t pickedIndex = newIndex;
    if (pickedIndex < blockCount) {
        newIndex++;
        return pickedIndex;
    } else {
        return newIndex = 1;
    }
}
\end{centeredshadowboxlisting}
\end{@empty}

\section[Performance Evaluation]{Performance Evaluation} \label{sec:loop-performance}

    Since the eviction decisions of LOOP page evictioners are largely im\-ple\-men\-ta\-tion-independent, the achieved hit rates in an exemplary DBMS are not required as performance benchmark for this evaluation. The only difference in performance between the different LOOP page eviction algorithms is the overhead resulting from the concurrent counting. Therefore, a microbenchmark measuring only the execution time of the counting is appropriate.

    The variables of the evaluation are the concurrent counter im\-ple\-men\-ta\-tion---the alternatives presented in the previous section are evaluated---and the number of threads concurrently incrementing the counter. The smallest ($>0$) and largest integers returned by the counters---representing the smallest and largest buffer pool indexes---do not significantly affect the performance of the LOOP page eviction, but when the largest buffer pool index is a power of $2$, the modulo operations of the \emph{modulo counter} and \emph{local modulo counter} are significantly faster. Therefore, this integer interval $\left[1 .. 53467\right]$---where $53467$ is not a power of $2$, because the evaluation should be as general as possible---is a constant in this evaluation.

\subsection[Micro-Benchmark]{Micro-Benchmark}

    The micro-benchmark used for the performance evaluation of the LOOP page eviction algorithms forks a certain number of worker threads---each of them calls the evaluated concurrent counter a certain number of times (\num{1000000})---and measures the wall time that has elapsed until all the worker threads finished their operation and joined.

\subsection[System Configuration]{Configuration of the Used System}

\begin{@empty}
    \begin{itemize}
        \itemsep0em
		\item	\textbf{CPU:} \emph{Intel® Core™ i7-8700} @$12 \times \SI{3.2}{\giga\hertz}$ from late 2017
        \item	\textbf{Main Memory:} $2 \times 8\text{GB} = 16\text{GB}$ of DDR4-SDRAM @\SI{2666}{\mega\hertz}
        \item	\textbf{OS:} \emph{Ubuntu 19.10}
        \item	\textbf{Compiler: } \emph{GCC 9.2.1} with \texttt{-O3} flag
    \end{itemize}
\end{@empty}

\subsection[Micro-Benchmark Results]{Micro-Benchmark Results} \label{subsec:loop-results}

\begin{@empty}
    Figure \ref{fig:loop_desktop_performance} shows the LOOP index generation throughput, which is total number of indexes generated (on all working threads) per time.

    \nottoggle{bwmode}{
        \tikzset{%
            mutex/.style = {color = cyan, mark = *},
            spinlock/.style = {color = magenta, mark = *},
            modulo/.style = {color = yellow, mark = square*},
            local/.style = {color = green, mark = diamond*},
            localModulo/.style = {color = red, mark = diamond*},
            lockFree/.style = {color = blue, mark = square*}
        }
    }{
        \tikzset{%
            mutex/.style = {color = black, mark = *},
            spinlock/.style = {color = black!50, mark = *},
            modulo/.style = {color = black, mark = square*},
            local/.style = {color = black, mark = diamond*},
            localModulo/.style = {color = black!50, mark = diamond*},
            lockFree/.style = {color = black!50, mark = square*}
        }
    }

    The \emph{blocking counters}---\emph{mutex counter} \ref{pgfplots:desktop-mutex} and \emph{spinlock counter} \ref{pgfplots:desktop-spinlock}---are the slowest counters when there are multiple working threads. The locking overhead and---for a higher number of working threads---the lock contention makes these concurrent counters slower than the \emph{non-blocking counters} by up to more than one order of magnitude. The more optimized mutex, which uses low-overhead busy-waiting (spinlock behavior) in situations of low contention and higher-overhead queuing and context switches in situations of higher contention, can better utilize the available physical CPU cores by suspending waiting threads---thus eliminating unnecessary uses of hyper-threading.

    The \emph{non-blocking counters}---\emph{modulo counter} \ref{pgfplots:desktop-modulo} and \emph{lock-free counter} \ref{pgfplots:desktop-lock-free}---perform almost identical to each other. In situations without contention, the \emph{modulo counter} is identical to the \emph{local modulo counter}, but the \emph{lock-free counter} suffers from the overhead due to the conditional branch operation. But most of the times, a modern CPU can correctly predict the targets of these branches. But when there are multiple working threads counting, the atomic operations on the counter variable are the bottleneck slowing down the \emph{non-blocking counters} compared to \emph{local counters}.

    \begin{figure}[ht!]
        \centering
        \resizebox{\textwidth}{!}{
            \begin{tikzpicture}[]
                \begin{axis}[xlabel = {Number of threads},
                             xlabel near ticks,
                             xmin = 1,
                             xmax = 12,
                             xtick distance = 3,
                             extra x ticks = {1},
                             scaled x ticks = false,
                             ylabel = {$\text{Average index generations }\left[\si{\indexes\per\second}\right]$},
                             ylabel near ticks,
                             ymode = log,
                             scaled y ticks = false,
                             grid = both,
                             width = \textwidth,
                             height = .65\textheight]

                    \addplot[mutex]       table[x = threads, y = throughput] {./data/loop/desktop_mutex_counter.csv};        \label{pgfplots:desktop-mutex}
                    \addplot[spinlock]    table[x = threads, y = throughput] {./data/loop/desktop_spinlock_counter.csv};     \label{pgfplots:desktop-spinlock}
                    \addplot[modulo]      table[x = threads, y = throughput] {./data/loop/desktop_modulo_counter.csv};       \label{pgfplots:desktop-modulo}
                    \addplot[local]       table[x = threads, y = throughput] {./data/loop/desktop_local_counter.csv};        \label{pgfplots:desktop-local}
                    \addplot[localModulo] table[x = threads, y = throughput] {./data/loop/desktop_local_modulo_counter.csv}; \label{pgfplots:desktop-local-modulo}
                    \addplot[lockFree]    table[x = threads, y = throughput] {./data/loop/desktop_lock_free_counter.csv};    \label{pgfplots:desktop-lock-free}
                \end{axis}
            \end{tikzpicture}
        }
        \caption{The throughput of index generations of the evaluated LOOP implementations}
        \label{fig:loop_desktop_performance}
    \end{figure}

    Obviously, the \emph{local counters}---\emph{local counter} \ref{pgfplots:desktop-local} and \emph{local modulo counter} \ref{pgfplots:desktop-local-modulo}---are the fastest counters, because they completely omit any synchronization between the counting threads. While the other counters count based on a global counter, these ones count based on local counters, resulting in a thread-wise round-robin selection of eviction candidates instead of a global round-robin selection of eviction candidates. The independence of the threads makes these counters very scalable as long as there are hardware threads available. This results in a performance advantage of up to almost two orders of magnitude compared over the \emph{non-blocking counters}. The significantly higher performance of the \emph{local counter} compared to the \emph{local modulo counter} is the result of the slow modulo operation when the divisor is not a power of two. The branch target of the conditional branch in the \emph{local counter} is predicted correctly most of the times, making its overhead negligible.
\end{@empty}

\section{Conclusion}

    Each LOOP page eviction algorithm is a good replacement for any other RANDOM page eviction algorithm. Tests in an OLTP database showed that the hit rate of the LOOP strategy is not worse than that of any other RANDOM page eviction strategy, and the overheads of the LOOP page eviction strategies---especially of the \emph{local counters}---are lower than the ones of any other RANDOM page eviction strategy. This makes the LOOP page evictioners superior to the other RANDOM page evictioners in OLTP applications (represented by the TPC-C benchmark). Since the use of local counters instead of a global counter does not change the hit rate in the DBS, the \emph{local counter}---the LOOP page eviction algorithm with the lowest overhead---can be recommended.
