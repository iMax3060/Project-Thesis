\documentclass[final]{scrreprt}	% setting of the document class; only the KOMA-Script classes are allowed here as this document uses many of those features

%%%%%%%%%%%%%%%%%%%%%%%%%%%%%%%%%%%%%%%%%%%%%%%
%%%%%%%%%%%%%%%%%%%%%%%%%%%%%%%%%%%%%%%%%%%%%%%
%%%%%%%%%%%%%%%%%%%%%%%%%%%%%%%%%%%%%%%%%%%%%%%
% Allows to align the headings of the numbered structure levels to the text using 
% \settoggle{alignheadings}{true} OR \toggletrue{alignheadings}:
%%%%%%%%%%%%%%%%%%%%%%%%%%%%%%%%%%%%%%%%%%%%%%% begindefinition
\usepackage{etoolbox}

\newtoggle{alignheadings}

\renewcommand*{\chapterformat}{%
    \iftoggle{alignheadings}{%
        \IfChapterUsesPrefixLine{%
            \chapapp~\thechapter\autodot\enskip%
        }{%
            \makebox[0pt][r]{\thechapter\autodot\enskip}%
        }%
    }{%
        \mbox{\chapappifchapterprefix{\nobreakspace}\thechapter \autodot\IfUsePrefixLine{}{\enskip}}%
    }%
}

\renewcommand*{\sectionformat}{%
    \iftoggle{alignheadings}{%
        \makebox[0pt][r]{\thesection\autodot\enskip}%
    }{%
        \thesection\autodot\enskip%
    }%
}

\renewcommand*{\subsectionformat}{%
    \iftoggle{alignheadings}{%
        \makebox[0pt][r]{\thesubsection\autodot\enskip}%
    }{%
        \thesubsection\autodot\enskip%
    }%
}

\renewcommand*{\subsubsectionformat}{%
    \iftoggle{alignheadings}{%
        \makebox[0pt][r]{\thesubsubsection\autodot\enskip}%
    }{%
        \thesubsubsection\autodot\enskip%
    }%
}

\renewcommand*{\paragraphformat}{%
    \iftoggle{alignheadings}{%
        \makebox[0pt][r]{\theparagraph\autodot\enskip}%
    }{%
        \theparagraph\autodot\enskip%
    }%
}

\renewcommand*{\subparagraphformat}{%
    \iftoggle{alignheadings}{%
        \makebox[0pt][r]{\thesubparagraph\autodot\enskip}%
    }{%
        \thesubparagraph\autodot\enskip%
    }%
}
%%%%%%%%%%%%%%%%%%%%%%%%%%%%%%%%%%%%%%%%%%%%%%% enddefinition

%%%%%%%%%%%%%%%%%%%%%%%%%%%%%%%%%%%%%%%%%%%%%%%
%%%%%%%%%%%%%%%%%%%%%%%%%%%%%%%%%%%%%%%%%%%%%%%
%%%%%%%%%%%%%%%%%%%%%%%%%%%%%%%%%%%%%%%%%%%%%%%
% Defines a toggle to switch between grayscale and colored output:
%%%%%%%%%%%%%%%%%%%%%%%%%%%%%%%%%%%%%%%%%%%%%%% begindefinition
\usepackage{etoolbox}            % allows the usage of different 

\newtoggle{bwmode}
%%%%%%%%%%%%%%%%%%%%%%%%%%%%%%%%%%%%%%%%%%%%%%% enddefinition

%%%%%%%%%%%%%%%%%%%%%%%%%%%%%%%%%%%%%%%%%%%%%%%
%%%%%%%%%%%%%%%%%%%%%%%%%%%%%%%%%%%%%%%%%%%%%%%
%%%%%%%%%%%%%%%%%%%%%%%%%%%%%%%%%%%%%%%%%%%%%%%
% Writes some text (i.e. "This page intentionally left blank.") on every blank page:
%%%%%%%%%%%%%%%%%%%%%%%%%%%%%%%%%%%%%%%%%%%%%%% begindefinition
\usepackage{scrlayer}                % allows the creation of individual page styles

\newcommand*{\blankpage}{%
    \vspace*{\fill}%
    \begin{center}%
        \textcolor{gray}{This page intentionally left blank.}%
    \end{center}%
    \vspace{\fill}%
}

\DeclareNewLayer[
    foreground,
    textarea,
    contents = \blankpage
  ]{blankpage.fg}
\DeclarePageStyleByLayers{blank}{blankpage.fg}
%%%%%%%%%%%%%%%%%%%%%%%%%%%%%%%%%%%%%%%%%%%%%%% enddefinition

%%%%%%%%%%%%%%%%%%%%%%%%%%%%%%%%%%%%%%%%%%%%%%%
%%%%%%%%%%%%%%%%%%%%%%%%%%%%%%%%%%%%%%%%%%%%%%%
%%%%%%%%%%%%%%%%%%%%%%%%%%%%%%%%%%%%%%%%%%%%%%%
% Redefines \thefootnote to use numbering starting from 0:
%%%%%%%%%%%%%%%%%%%%%%%%%%%%%%%%%%%%%%%%%%%%%%% begindefinition
\newcounter{indianfoot}
\newcommand{\useindianfootnotes}{
    \renewcommand{\thefootnote}{%
        \setcounter{indianfoot}{0}%
        \addtocounter{indianfoot}{\value{footnote}}%
        \arabic{indianfoot}}%
}
%%%%%%%%%%%%%%%%%%%%%%%%%%%%%%%%%%%%%%%%%%%%%%% enddefinition

%%%%%%%%%%%%%%%%%%%%%%%%%%%%%%%%%%%%%%%%%%%%%%%
%%%%%%%%%%%%%%%%%%%%%%%%%%%%%%%%%%%%%%%%%%%%%%%
%%%%%%%%%%%%%%%%%%%%%%%%%%%%%%%%%%%%%%%%%%%%%%%
% Redefines \thempfootnote to use numbering starting from 1 (should be 0 but buggy):
%%%%%%%%%%%%%%%%%%%%%%%%%%%%%%%%%%%%%%%%%%%%%%% begindefinition
\newcounter{indianmpfoot}
\newcommand{\useindianmpfootnotes}{
    \renewcommand\thempfootnote{\arabic{mpfootnote}}
%    \renewcommand{\thempfootnote}{%
%        \setcounter{indianmpfoot}{0}%
%        \addtocounter{indianmpfoot}{\value{mpfootnote}}%
%        \arabic{indianmpfoot}}%
}
%%%%%%%%%%%%%%%%%%%%%%%%%%%%%%%%%%%%%%%%%%%%%%% enddefinition

%%%%%%%%%%%%%%%%%%%%%%%%%%%%%%%%%%%%%%%%%%%%%%%
%%%%%%%%%%%%%%%%%%%%%%%%%%%%%%%%%%%%%%%%%%%%%%%
%%%%%%%%%%%%%%%%%%%%%%%%%%%%%%%%%%%%%%%%%%%%%%%
% Allows the highlighting of lines in lstlisting environments using the key: 
%     linebackgroundcolor = {\btLstHL{line ranges}}
%%%%%%%%%%%%%%%%%%%%%%%%%%%%%%%%%%%%%%%%%%%%%%% begindefinition
\usepackage{listings}

% Define backgroundcolor
    \usepackage[
        style=1,
        skipbelow=\topskip,
        skipabove=\topskip
    ]{mdframed}

    \definecolor{bggray}{rgb}{0.85, 0.85, 0.85}
    \mdfsetup{
        leftmargin = 20pt,
        rightmargin = 20pt,
        backgroundcolor = bggray,
        middlelinecolor = black,
        roundcorner = 15
    }
    \BeforeBeginEnvironment{lstlisting}{\begin{mdframed}\vskip-.5\baselineskip}
    \AfterEndEnvironment{lstlisting}{\end{mdframed}}

\makeatletter
%
% \btIfInRange{number}{range list}{TRUE}{FALSE}
%
% Test if int number <number> is element of a (comma separated) list of ranges
% (such as: {1,3-5,7,10-12,14}) and processes <TRUE> or <FALSE> respectively
%
        \newcount\bt@rangea
        \newcount\bt@rangeb

        \newcommand\btIfInRange[2]{%
            \global\let\bt@inrange\@secondoftwo%
            \edef\bt@rangelist{#2}%
            \foreach \range in \bt@rangelist {%
                \afterassignment\bt@getrangeb%
                \bt@rangea=0\range\relax%
                \pgfmathtruncatemacro\result{ ( #1 >= \bt@rangea) && (#1 <= \bt@rangeb) }%
                \ifnum\result=1\relax%
                    \breakforeach%
                    \global\let\bt@inrange\@firstoftwo%
                \fi%
            }%
            \bt@inrange%
        }

        \newcommand\bt@getrangeb{%
            \@ifnextchar\relax%
            {\bt@rangeb=\bt@rangea}%
            {\@getrangeb}%
        }

        \def\@getrangeb-#1\relax{%
            \ifx\relax#1\relax%
                \bt@rangeb=100000%   \maxdimen is too large for pgfmath
            \else%
                \bt@rangeb=#1\relax%
            \fi%
        }

%
% \btLstHL{range list}
%
    \definecolor{lsthighlight}{RGB}{217, 216, 255}
        \newcommand{\btLstHL}[1]{%
            \btIfInRange{\value{lstnumber}}{#1}%
            {\color{lsthighlight}}%
            {\def\lst@linebgrd}%
        }%

%
% \btInputEmph[listing options]{range list}{file name}
%
        \newcommand{\btLstInputEmph}[3][\empty]{%
            \lstset{%
                linebackgroundcolor=\btLstHL{#2}%
                \lstinputlisting{#3}%
            }% \only
        }
        
% Patch line number key to call line background macro
%        \lst@Key{numbers}{none}{%
%            \def\lst@PlaceNumber{\lst@linebgrd}%
%            \lstKV@SwitchCases{#1}{%
%                none&\\%
%                left&\def\lst@PlaceNumber{\llap{\normalfont
%                \lst@numberstyle{\thelstnumber}\kern\lst@numbersep}\lst@linebgrd}\\%
%                right&\def\lst@PlaceNumber{\rlap{\normalfont
%                \kern\linewidth \kern\lst@numbersep
%                \lst@numberstyle{\thelstnumber}}\lst@linebgrd}%
%            }{%
%                \PackageError{Listings}{Numbers #1 unknown}\@ehc
%            }%
%        }

% New keys
        \lst@Key{linebackgroundcolor}{}{%
            \def\lst@linebgrdcolor{#1}%
        }
        \lst@Key{linebackgroundsep}{0pt}{%
            \def\lst@linebgrdsep{#1}%
        }
        \lst@Key{linebackgroundwidth}{\linewidth}{%
            \def\lst@linebgrdwidth{#1}%
        }
        \lst@Key{linebackgroundheight}{\ht\strutbox}{%
            \def\lst@linebgrdheight{#1}%
        }
        \lst@Key{linebackgrounddepth}{\dp\strutbox}{%
            \def\lst@linebgrddepth{#1}%
        }
        \lst@Key{linebackgroundcmd}{\color@block}{%
            \def\lst@linebgrdcmd{#1}%
        }

% Line Background macro
        \newcommand{\lst@linebgrd}{%
            \ifx\lst@linebgrdcolor\empty\else
                \rlap{%
                    \lst@basicstyle
                    \color{-.}% By default use the opposite (`-`) of the current color (`.`) as background
                    \lst@linebgrdcolor{%
                        \kern-\dimexpr\lst@linebgrdsep\relax%
                        \lst@linebgrdcmd{\lst@linebgrdwidth}{\lst@linebgrdheight}{\lst@linebgrddepth}%
                    }%
                }%
            \fi
        }

\makeatother
%%%%%%%%%%%%%%%%%%%%%%%%%%%%%%%%%%%%%%%%%%%%%%% enddefinition

%%%%%%%%%%%%%%%%%%%%%%%%%%%%%%%%%%%%%%%%%%%%%%%
%%%%%%%%%%%%%%%%%%%%%%%%%%%%%%%%%%%%%%%%%%%%%%%
%%%%%%%%%%%%%%%%%%%%%%%%%%%%%%%%%%%%%%%%%%%%%%%
% Defines a key=value switch for lstlistings with matchrangestart=<true/false> that allows
% the numbering following the linerange key=value settings:
%%%%%%%%%%%%%%%%%%%%%%%%%%%%%%%%%%%%%%%%%%%%%%% begindefinition
\usepackage{listings}

\makeatletter
\lst@Key{matchrangestart}{false}{\lstKV@SetIf{#1}\lst@ifmatchrangestart}
\def\lst@SkipToFirst{%
    \lst@ifmatchrangestart\c@lstnumber=\numexpr-1+\lst@firstline\fi
    \ifnum \lst@lineno<\lst@firstline
        \def\lst@next{\lst@BeginDropInput\lst@Pmode
        \lst@Let{13}\lst@MSkipToFirst
        \lst@Let{10}\lst@MSkipToFirst}%
        \expandafter\lst@next
    \else
        \expandafter\lst@BOLGobble
    \fi}
\makeatother
%%%%%%%%%%%%%%%%%%%%%%%%%%%%%%%%%%%%%%%%%%%%%%% enddefinition

%%%%%%%%%%%%%%%%%%%%%%%%%%%%%%%%%%%%%%%%%%%%%%%
%%%%%%%%%%%%%%%%%%%%%%%%%%%%%%%%%%%%%%%%%%%%%%%
%%%%%%%%%%%%%%%%%%%%%%%%%%%%%%%%%%%%%%%%%%%%%%%
% Allows the writing of dates specified as \specificdate{YYYY}{MM}{DD}:
%%%%%%%%%%%%%%%%%%%%%%%%%%%%%%%%%%%%%%%%%%%%%%% begindefinition
\newcommand{\specificdate}[3]{\setdate{#1}{#2}{#3} \datedate}
%%%%%%%%%%%%%%%%%%%%%%%%%%%%%%%%%%%%%%%%%%%%%%% enddefinition

%%%%%%%%%%%%%%%%%%%%%%%%%%%%%%%%%%%%%%%%%%%%%%%
%%%%%%%%%%%%%%%%%%%%%%%%%%%%%%%%%%%%%%%%%%%%%%%
%%%%%%%%%%%%%%%%%%%%%%%%%%%%%%%%%%%%%%%%%%%%%%%
% Allows inline comments using \ignore{comment text}:
%%%%%%%%%%%%%%%%%%%%%%%%%%%%%%%%%%%%%%%%%%%%%%% begindefinition
\newcommand{\ignore}[1]{}
%%%%%%%%%%%%%%%%%%%%%%%%%%%%%%%%%%%%%%%%%%%%%%% enddefinition


% Settings regarding the general paper format:
\KOMAoptions{paper = A4}			% the paper size of the printout
\KOMAoptions{paper = portrait}			% the orientation of the document
\KOMAoptions{pagesize = auto}		% sets the page size of the compiled document (not for the calculation of the type area)
\settoggle{bwmode}{false}				% enable/disable grayscale mode for the output (see ./tex/command_definitions.tex)

%\KOMAoptions{draft} 				% for a draft with additional hints about warnings inside the compiled document

% Settings regarding the type area:
\KOMAoptions{twoside = true}			% enable/disable for two-sided printouts (with specific type area)
\KOMAoptions{BCOR = 8.25mm}		% binding correction
\KOMAoptions{DIV = calc}			% setting for the type area (calc uses automatic calculation)
\KOMAoptions{twocolumn = false}		% enable/disable two-columned documents
\KOMAoptions{headinclude = false}		% enable/disable that the header is part of the type area (instead of top margin) (if header is empty or nearly empty like only pagination)
\KOMAoptions{footinclude = false}		% enable/disable that the footer is part of the type area (instead of bottom margin) (if footer is empty or nearly empty like only pagination)
\KOMAoptions{mpinclude = false}		% enable/disable that marginal notes are part of the type area (instead of the margin) (increases the margin if enabled; shouldn't be enabled)
% If mpinclude is enabled the following option controls if e.g. 50% of the width of the marginal notes should be part of the margin:
% \setlength{\marginparwidth}{2\marginparwidth}
% \setlength{\textwidth}{\textwidth - \marginparwidth}

% Settings regarding the used fonts and typographical details:
\KOMAoptions{fontsize = 11pt}			% the \normalsize font size
\usepackage[utf8]{inputenc}			% character encoding of this .tex-file
\usepackage[T1]{fontenc}				% character encoding within the compiled document
\usepackage{libertine, libertinust1math}	% used font/fonts
\usepackage{microtype}				% enables microtypography (can be further configured but the default mode is well)

% Settings regarding the text alignment:
\usepackage
		[
			newcommands,			% the commands \centering, \raggedleft, and \raggedright are redefined to work like \Centering, \RaggedLeft, and \RaggedRight
			newparameters			% the commands \Centering, \RaggedLeft, and \RaggedRight don't behave like vanilla \centering, \raggedleft, and \raggedright 
		]{ragged2e}				% offers new environments for ragged ( and justified) text alignment with many parameters that can be changed using \setlength 

% Settings regarding the preliminaries:
\KOMAoptions{titlepage = firstiscover}	% enable/disable that the title created using \maketitle is on an own page or activate an additional cover page using the value 'firstiscover'
\KOMAoptions{abstract = true}			% enable/disable the creation of an abstract before the table of contents as specified in \begin{abstract} ... \end{abstract}
%\raggedbottom					% Prevents the spread of paragraphs inside a page which is used to end a page on the same height
\newcommand*\origfootnoterule{}
\let\origfootnoterule\footnoterule
\renewcommand*{\footnoterule}{\vskip 0pt plus .01fil\origfootnoterule}

% Settings regarding the table of contents:
\setcounter{tocdepth}{\subsectionnumdepth}	% set the lowest structure level of the entries inside the table of contents
\KOMAoptions{toc = bib}					% adds the bibliography to the table of contents
\KOMAoptions{toc = index}				% adds the index to the table of contents
\KOMAoptions{toc = listof}					% adds the list of {figures, tables} to the table of contents (can be selected seperatly)
\KOMAoptions{toc = chapterentrywithoutdots}	% the chapter entry isn't connected to its page number with dots (as it is with section entries)
\PassOptionsToPackage{toc = bib}{biblatex}	% fixes toc = bib when using BibLaTex instead of BibTeX
\KOMAoptions{toc = graduated}			% makes the table of contents hierarchic
\KOMAoptions{toc = indenttextentries}		% indents unnumbered entries

% Settings regarding the headers and footers:
\KOMAoptions{headlines = 1.25}		% the height of the header in number of lines (increase e.g. for multiline headers)
% \KOMAoptions{headheight = 1cm}		% the height of the header in some measure of length
\KOMAoptions{footlines = 1.25}			% the height of the footer in number of lines (increase e.g. for multiline footers); might be incompatible with some packages
% \KOMAoptions{footheight = 1cm}		% the height of the footer in some measure of length; might be incompatible with some packages
\KOMAoptions{headsepline = false}		% enable/disable a line below the header (headsepline = true -> headinclude = true)
\KOMAoptions{footsepline = false}		% enable/disable a line above the footer (footsepline = true -> footinclude = true
\usepackage{scrlayer-scrpage}			% allows the creation of individual page styles
\KOMAoptions{cleardoublepage = blank}	% select the page style for blank pages
\automark[section]{chapter}			% select the content of \headmark for left and right pages
\clearpairofpagestyles				% empties the plain and main page styles
\ifoot[\pagemark]{\pagemark}			% sets the pagination in the inner (actually outer) field of the footer of the plain and main page styles
\ihead[]{\headmark}					% sets the chapter/section title in the inner (actually outer) field of the header main page style

% Settings regarding the structuring and headings:
\KOMAoptions{open = any}					% select the page (any, left, right) to open a new chapter in two-sided documents
\KOMAoptions{parskip = false}					% toggle between first line indent and paragraph distance (can be further specified)
\KOMAoptions{headings = big}					% select the font size of the headings of the different structure levels
\KOMAoptions{headings = optiontohead}			% select the usage of the short title of a disposition
\setcounter{secnumdepth}{\subsubsectionnumdepth}% set the lowest structure level with a numbering
\KOMAoptions{numbers = autoendperiod}			% enable/disable the usage of periods after the numbering of a chapter (auto mode enables option when there are numbers that aren't arabic numbers)
\KOMAoptions{chapterprefix = false}				% enable/disable the usage of ``Chapter n\\ \chaptername'' instead of ``n \chaptername''
\KOMAoptions{appendixprefix = true}			% enable/disable the usage of ``Appendix n\\ \chaptername'' instead of ``n \chaptername''
\settoggle{alignheadings}{true}					% enable/disable the alignment of the headings of the numbered structure levels to the text (which results in putting the numbers in the margin)

% Settings regarding the footnotes:
\KOMAoptions{footnotes = multiple}				% enable/disable a separation of multiple consecutive footnotes
\renewcommand*{\multfootsep}{,\nobreakspace}	% select the separater between multiple consecutive footnotes
%\useindianfootnotes						% activates the usage of footnote numbering starting with zero - ATTENTION: Doesn't work at the moment!
%\useindianmpfootnotes						% activates the usage of footnote numbering starting with zero within minipages - ATTENTION: Doesn't work at the moment!
\deffootnote[1em]{1.5em}{1em}{%				% select the indentation of the first line of a footnote (footnote mark aligned right in this indentation), the indentation of the further lines and first line indent
	\textsuperscript{\thefootnotemark}%			% specify the footnote mark inside the footnote
}

% Settings regarding the floats and captions:
\KOMAoptions{captions = figuresignature}			% select if the caption of a figure should be printed as signature or heading (to put the caption below or above the content of the figure)
\KOMAoptions{captions = tableheading}			% select if the caption of a table should be printed as signature or heading (to put the caption below or above the content of the table)
\KOMAoptions{captions = oneline}				% select if single-line captions should be centered
\KOMAoptions{captions = bottombeside}			% select where to put a caption created using ``captionsbeside'' vertically
\KOMAoptions{captions = outerbeside}			% select where to put a caption created using ``captionsbeside'' horizontally (different for one-sided and two-sided documents)
\setcapmargin{1em}							% select the left and right margin of a caption (different options can be used to set the left, right, inner or outer margin separately)
\setcapindent{0.5em}						% select the indent of the second and further lines of a caption (alternatives are hanging caption or only have an indentation in the second line)
\addtokomafont{caption}{\small\justifying}			% select the font of the captions
\addtokomafont{captionlabel}{\bfseries}			% select the font of the caption label (in addition to the font of the caption)
\KOMAoptions{listof = chaptergapsmall}			% select the highlighting of floats of different chapters in the lists of floats
\KOMAoptions{listof = graduated}				% select the layout of the lists of floats
\KOMAoptions{listof = indenttextentries}			% indents unnumbered floats in the lists of floats
\usepackage{float}							% allows the definition of new types of floats
\floatstyle{komaabove}						% every definition of a new type of floats will use this style (defined by KOMA-Script: komaabove, komabelow)
\newfloat{code}{tbp}{code}[chapter]				% defines a new type of float ``code'' that gets numbered per chapter and that can be used with \begin{code} ... \end{code} and \listof{code}{List of Listings}
\floatname{code}{Listing}						% the name of the float ``code'' that is printed in the caption is ``Listing''
\newfloat{algorithm}{tbp}{algorithm}[chapter]		% defines a new type of float ``algorithm'' that gets numbered per chapter and that can be used with \begin{code} ... \end{code} and \listof{code}{List of Listings}
\floatname{algorithm}{Algorithm}				% the name of the float ``algorithm'' that is printed in the caption is ``Algorithm''
\usepackage{caption}						% used to split floats using \ContinuedFloat but has many more features

% Settings regarding the used languages:
\usepackage[main = english, ngerman]{babel}	% define the used languages (has influence on different things like hyphenation, date format and figure labels)
\babeltags{eng = english, de = ngerman}		% allows to switch between the loaded languages by using the tags like \begin{<tag>} ... \end{<tag>} or \text<tag>{ ... }
\usepackage[
			detect-all]					% use the font settings of the surrounding text for numbers and units set with siunitx
		{siunitx}						% allows the easy typesetting of numbers, units and combinations including lists and ranges
\sisetup{
	range-phrase = \text{--},				% select the phrase between upper and lower bounds of ranges (here: -)
	binary-units = true,					% enable/disable loading of binary prefixes
	per-mode = fraction,					% select how to display \per (symbol: uses exponents; fraction: uses fractions)
	fraction-function = \sfrac}				% select the kind of fraction used in siunitx (e.g. frac, cfrac, rfrac, sfrac, ...)
\DeclareSIUnit{\euro}{\text{\texteuro}}		% define the unit \euro using the €-symbol
\DeclareSIUnit{\usdollar}{\text{US-\textdollar}}	% define the unit \usdollar using the US-$-symbol
\DeclareSIUnit{\cy}{\text{Cyc.}}				% define the unit \cy used for Cycles using the abbreviation Cyc.
\DeclareSIUnit\century{\text{century}}		% define the unit \century
\DeclareSIUnit\year{\text{year}}				% define the unit \year
\DeclareSIUnit\queries{\text{queries}}		% define the unit \queries
\DeclareSIUnit\transactions{\text{transactions}}	% define the unit \transactions
\usepackage[hyphens]{url}

\usepackage{graphicx}				% allows including of graphics and the scaling and rotating of elements
\usepackage{etoolbox}				% toolbox used by packages and classes
\usepackage{xpatch}					% extends the patching facility of etoolbox
\usepackage{csquotes}				% context sensitive quotation environment e.g. used by biblatex
\usepackage{datenumber}				% allows to create a number from a date and especially it allows to create a specific date
\usepackage{tabularx}				% a tabular* environment that can control the width of columns
\usepackage{multirow}				% allows tabular cells spanning multiple rows
\usepackage
		[
			usenames,			% allows the usage of the names of the 16 default colors
			dvipsnames,			% allows the usage of the names of 64 additional colors
			svgnames,			% allows the usage of the names of ca. 150 additional colors
			x11names,			% allows the usage of the names of ca. 300 additional colors
			table					% allows the coloring of whole tabular by changing the color before the definition of the tabular
		]{xcolor}					% extended color facility
\usepackage{blindtext}				% allows to easily add some dummy text (similar to lipsum)
\usepackage{prelim2e}				% marks every page as being a preliminary version when this document is compiled as a draft

\usepackage
		[
			style = alphabetic, 		% select the style of the citation and of the bibliography
			backend = biber		% select the backend that processes the .bib file (run the selected backend instead of BibTeX)
		]{biblatex}					% used to create the bibliography, more modern alternative to the standard BibTeX

\addbibresource[datatype = bibtex]{./tex/references.bib}		% loads the specified bibliography file (in BibTeX format) into BibLaTeX
\usepackage{breakcites}				% multiple citations within one \cite break at the end of the line
\usepackage[]{hyperref}				% allows hyperlinks within the output document (hyperfootnotes = false to make it compatible with package footmisc)
\usepackage{nameref}				% allows the usage of the command \nameref which prints the title of the referenced label instead of its number

\usepackage{tikz}					% extremely powerful facility to create diagrams
\usetikzlibrary{shapes.geometric, shapes.misc, shapes.callouts, shapes.multipart}			% provide several shapes besides the standard ones
\usetikzlibrary{decorations.pathreplacing}	% allows decorated paths without having the original (undecorated) line
\usetikzlibrary{patterns}				% allows the usage of several patterns to fill shapes
\usetikzlibrary{positioning}				% defines additional options for placing nodes conventionally
\usetikzlibrary{calc}					% allows extended coordinate calculation
\usetikzlibrary{spy}					% allows the magnification of parts of a tikz diagram
\usetikzlibrary{chains}				% allows the creation of chains of nodes
\usepackage{pgfplots}				% allows the creation of plots to visualize data
\usepackage{pgfplotstable}			% allows the loading of .csv-files for pgfplots
\usepgflibrary{plotmarks}				% extends the available plot marks used e.g. for pgfplots
\usepgfplotslibrary{fillbetween}			% allows the filling of areas between curves of pgfplots using colors or patterns
\usepackage{ifthen}					% allows the usage of the \ifthenelse control structure and some boolean operations with it

\pgfplotsset{compat = 1.14}

\usepackage{xfrac}					% adds the \sfrac fraction mode
\usepackage{rotating}				% allows different kinds of rotations for many kinds of elements
\usepackage{stackengine}				% allows the stacking of elements like symbols
\usepackage{bm}					% adds one way of bold math
\usepackage{ulem}					% allows many kinds of text decorations like underlines or strikes
\robustify\uline						% allows the usage of \uline together with typewriter fonts
\normalem						% \emph uses italic fonts to emphasize text

\usepackage{listings}				% allows the printing of source code
% for a definition of the parameter 'matchrangestart' see ./tex/command_definitions.tex
\usepackage{algpseudocode}			% allows the creation of pseudocode listings

\nottoggle{bwmode}{
	\definecolor{keyword}{RGB}{5,0,111}
%	\definecolor{keyword2}{RGB}{0, 111, 109}
%	\definecolor{keyword3}{RGB}{5,0,111}
%	\definecolor{keyword4}{RGB}{80, 0, 201}
%	\definecolor{keyword5}{RGB}{80, 0, 201}
%	\definecolor{keyword6}{RGB}{22, 69, 33}
	\definecolor{string}{RGB}{0,114,0}
	\definecolor{comment}{RGB}{109,109,109}
	\definecolor{listingsbackground}{RGB}{255, 255, 225}
	\definecolor{listingsrulesep}{RGB}{205, 205, 205}
	\definecolor{lsthighlight}{RGB}{217, 216, 255}			% used to highlight specific lines of a lstlisting
	\definecolor{alert}{RGB}{244, 0, 68}
}{
	\definecolor{keyword}{RGB}{0,0,0}
%	\definecolor{keyword2}{RGB}{0,0,0}
%	\definecolor{keyword3}{RGB}{0,0,0}
%	\definecolor{keyword4}{RGB}{0,0,0}
%	\definecolor{keyword5}{RGB}{0,0,0}
%	\definecolor{keyword6}{RGB}{0,0,0}
	\definecolor{string}{RGB}{0,0,0}
	\definecolor{comment}{RGB}{109,109,109}
	\definecolor{listingsbackground}{RGB}{255, 255, 255}
	\definecolor{listingsrulesep}{RGB}{205, 205, 205}
	\definecolor{lsthighlight}{RGB}{205, 205, 205}			% used to highlight specific lines of a lstlisting
	\definecolor{alert}{RGB}{0, 0, 0}
}

\lstdefinestyle{basic}{
	basicstyle = \scriptsize \ttfamily,
	keywordstyle = \color{keyword}\bfseries,
%	keywordstyle=[2]\color{keyword2},
%	keywordstyle=[3]\color{keyword3},
%	keywordstyle=[4]\color{keyword4}\bfseries,
%	keywordstyle=[5]\color{keyword5},
%	keywordstyle=[6]\color{keyword6}\bfseries,
	stringstyle = \color{string}\bfseries,
	commentstyle = \color{comment},
	morecomment = [l][\color{comment}]{\#},
	breaklines = false,
	showlines = true,
	matchrangestart = true,
	numbers = left,
	numberstyle = \tiny,
	stepnumber = 1,
	numberblanklines = false,
	numbersep = 10pt,
	frame = shadowbox,
	backgroundcolor = \color{listingsbackground},
	rulesepcolor = \color{listingsrulesep},
	inputencoding = utf8
}

\lstdefinestyle{inline}{
	basicstyle = \scriptsize \ttfamily,
	inputencoding = utf8
}

\lstset{
	language = [ISO]C++,
	style = inline,
%	keywords = [2]{w_rc_t, bf_tree_m, generic_page, latch_mode_t, lsn_t, bf_tree_cb_t, sthread_t, btree_page_h, smlevel_0, fixable_page_h, GeneralRecordIds, std},
%	keywords = [3]{PageID, bf_idx, bf_idx_pair, general_recordid_t},
%	keywords = [4]{WAIT_IMMEDIATE, WAIT_FOREVER, LATCH_EX, LATCH_NL, INVALID, FOSTER_CHILD, stINUSE},
%	keywords = [5]{_swizzled, _pid, _buffer, _evictioner, _hashtable, first, second, null, lsn, _pin_cnt, _enable_swizzling},
%	keywords = [6]{W_DO, RC}
}

% Select the chapters (included files) for partial compilation of drafts (better than just commenting \include commands because of .aux-files); omit .tex file-ending here and in each of the \include commands:
\includeonly{./tex/titlepage,./tex/abstract,\ignore{}./tex/free_list,./tex/random,\ignore{}./tex/tu_logo,./tex/command_definitions_early,./tex/command_definitions_late}

\newcommand*{\ptsans}{\fontfamily{PTSans-TLF}\selectfont}	% toggle to change to the standard font of the University of Kaiserslautern: PT Sans
\DeclareTextFontCommand{\textptsans}{\ptsans}			% environment to change to the standard font of the University of Kaiserslautern: PT Sans

\iftoggle{bwmode}{
	\definecolor{TUblue}{RGB}{0,0,0}					% The blue color used in the logo of the University of Kaiserslautern is printed black when in grayscale mode
	\definecolor{TUred}{RGB}{127,127,127}					% The red color used in the logo of the University of Kaiserslautern is printed black when in grayscale mode
}{
	\definecolor{TUblue}{RGB}{0,96,142}				% The blue color used in the logo of the University of Kaiserslautern
	\definecolor{TUred}{RGB}{188,38,26}				% The red color used in the logo of the University of Kaiserslautern
}

% The following new commands are tikzpicture-environments containing different logos of the University of Kaiserslautern. They are taken from logos published on their website.
% It defines the following commands: \TULogo, \TULogoWithText, \CSLogo

% The logo of the University of Kaiserslautern ():
% Taken from: http://www.uni-kl.de/fileadmin/prum/tupublic/TU_Logo_ohne_Feld/TUKL_LOGO_4C.svg on the 2016-12-14
% Manipulated using: Inkscape (https://inkscape.org/)
% Converted to TikZ using: svg2tikz (https://github.com/kjellmf/svg2tikz) as an Inkscape extension
\newcommand{\TULogo}[1][1]{
	\begin{tikzpicture}[
		y = 5pt,
		x = 5pt,
		opacity = #1
	]
		% Top part:
		\path[fill = TUblue] (2.898, 23.855) -- (2.898, 20.561) -- (24.299, 20.561)  -- (24.299, 24.034) -- (13.541, 27.197) -- cycle;
		% Left part:
		\path[fill = TUblue] (5.679, 19.307) -- (9.993, 19.307) -- (4.3, 0)  -- (0, 0) -- cycle;
		% Top rectangle:
		\path[fill = TUblue] (17.481, 19.316) rectangle (22.247, 14.727);
		% Middle rectangle:
		\path[fill = TUred] (17.481, 11.953) rectangle (22.247, 7.363);
		% Bottom rectangle:
		\path[fill = TUblue] (17.481, 4.59) rectangle (22.247, 0);
	\end{tikzpicture}
}

% The logo with text of the University of Kaiserslautern:
% Taken from: http://www.uni-kl.de/fileadmin/prum/tupublic/TU_Logo_ohne_Feld/TUKL_LOGO_4C.svg on the 2016-12-14
% Converted to TikZ using: svg2tikz (https://github.com/kjellmf/svg2tikz) as an Inkscape (https://inkscape.org/) extension
\newcommand{\TULogoWithText}{
	\begin{tikzpicture}[
		y = 0.8pt,
		x = 0.8pt,
		yscale = -1,
		xscale = 1
	]
		% KAISERSLAUTERN:
		\path[fill = TUblue] (164.0310,41.3410) -- (164.0310,36.5480) -- (163.8200,35.1050) -- (163.8890,35.1050) -- (164.6070,36.5480) -- (168.0610,41.4050) -- (169.3700,41.4050) -- (169.3700,32.1510) -- (167.6650,32.1510) -- (167.6650,36.9830) -- (167.8760,38.3870) -- (167.8120,38.3870) -- (167.1170,36.9810) -- (163.6440,32.0860) -- (162.3270,32.0860) -- (162.3270,41.3400) -- (164.0310,41.3400) -- cycle(155.4790,33.6880) .. controls (155.5790,33.6630) and (155.7170,33.6450) .. (155.8970,33.6370) .. controls (156.0780,33.6290) and (156.2590,33.6250) .. (156.4440,33.6250) .. controls (156.9200,33.6250) and (157.2800,33.7360) .. (157.5230,33.9570) .. controls (157.7670,34.1800) and (157.8890,34.4880) .. (157.8890,34.8810) .. controls (157.8890,35.4060) and (157.7410,35.7830) .. (157.4420,36.0120) .. controls (157.1450,36.2410) and (156.7460,36.3540) .. (156.2450,36.3540) -- (155.4790,36.3540) -- (155.4790,33.6880) -- cycle(155.4790,41.3410) -- (155.4790,37.5730) -- (156.4560,37.7820) -- (158.5240,41.3410) -- (160.5940,41.3410) -- (158.5110,37.8720) -- (157.8510,37.4330) .. controls (158.3790,37.2490) and (158.8020,36.9190) .. (159.1200,36.4470) .. controls (159.4350,35.9720) and (159.5950,35.3570) .. (159.5950,34.6050) .. controls (159.5950,34.1010) and (159.5020,33.6810) .. (159.3150,33.3430) .. controls (159.1290,33.0070) and (158.8800,32.7410) .. (158.5680,32.5480) .. controls (158.2570,32.3570) and (157.9040,32.2200) .. (157.5090,32.1420) .. controls (157.1140,32.0620) and (156.7140,32.0230) .. (156.3090,32.0230) .. controls (156.1320,32.0230) and (155.9380,32.0290) .. (155.7270,32.0370) .. controls (155.5160,32.0450) and (155.3000,32.0580) .. (155.0780,32.0760) .. controls (154.8540,32.0920) and (154.6310,32.1170) .. (154.4050,32.1480) .. controls (154.1790,32.1810) and (153.9700,32.2140) .. (153.7770,32.2480) -- (153.7770,41.3420) -- (155.4790,41.3420) -- cycle(151.9100,41.3410) -- (151.9100,39.7390) -- (148.1030,39.7390) -- (148.1030,37.4970) -- (151.5180,37.4970) -- (151.5180,35.8930) -- (148.1030,35.8930) -- (148.1030,33.7520) -- (151.8450,33.7520) -- (151.8450,32.1500) -- (146.3980,32.1500) -- (146.3980,41.3390) -- (151.9100,41.3390) -- cycle(140.2020,33.7530) -- (140.2020,41.3410) -- (141.9070,41.3410) -- (141.9070,33.7530) -- (144.6810,33.7530) -- (144.6810,32.1510) -- (137.4200,32.1510) -- (137.4200,33.7530) -- (140.2020,33.7530) -- cycle(132.7710,41.5010) .. controls (133.2500,41.5010) and (133.6930,41.4350) .. (134.0960,41.3040) .. controls (134.5000,41.1710) and (134.8420,40.9640) .. (135.1220,40.6850) .. controls (135.4020,40.4060) and (135.6200,40.0500) .. (135.7770,39.6210) .. controls (135.9320,39.1910) and (136.0100,38.6800) .. (136.0100,38.0860) -- (136.0100,32.1520) -- (134.3050,32.1520) -- (134.3050,37.9590) .. controls (134.3050,38.6390) and (134.1820,39.1310) .. (133.9360,39.4390) .. controls (133.6890,39.7460) and (133.2900,39.9000) .. (132.7390,39.9000) .. controls (132.4610,39.9000) and (132.2170,39.8670) .. (132.0070,39.8000) .. controls (131.7970,39.7330) and (131.6220,39.6240) .. (131.4800,39.4700) .. controls (131.3390,39.3160) and (131.2360,39.1160) .. (131.1690,38.8680) .. controls (131.1020,38.6200) and (131.0700,38.3150) .. (131.0700,37.9580) -- (131.0700,32.1510) -- (129.3650,32.1510) -- (129.3650,38.3090) .. controls (129.3660,40.4370) and (130.5010,41.5010) .. (132.7710,41.5010)(123.9030,35.8310) -- (124.1770,34.3760) -- (124.2410,34.3760) -- (124.5220,35.8170) -- (125.2030,37.8620) -- (123.2280,37.8620) -- (123.9030,35.8310) -- cycle(122.0790,41.3410) -- (122.7000,39.4640) -- (125.7220,39.4640) -- (126.3310,41.3410) -- (128.0350,41.3410) -- (124.7280,32.0870) -- (123.4420,32.0870) -- (120.2810,41.3410) -- (122.0790,41.3410) -- cycle(119.2220,41.3410) -- (119.2220,39.7390) -- (115.1150,39.7390) -- (115.1150,32.1510) -- (113.4100,32.1510) -- (113.4100,41.3400) -- (119.2220,41.3400) -- cycle(106.5570,41.3410) .. controls (107.0430,41.4640) and (107.6040,41.5290) .. (108.2410,41.5290) .. controls (108.7220,41.5290) and (109.1670,41.4700) .. (109.5720,41.3530) .. controls (109.9760,41.2380) and (110.3220,41.0600) .. (110.6060,40.8240) .. controls (110.8910,40.5880) and (111.1130,40.2910) .. (111.2700,39.9330) .. controls (111.4280,39.5760) and (111.5070,39.1480) .. (111.5070,38.6540) .. controls (111.5070,38.1770) and (111.4160,37.7790) .. (111.2330,37.4570) .. controls (111.0510,37.1350) and (110.8230,36.8650) .. (110.5450,36.6460) .. controls (110.2680,36.4290) and (109.9500,36.2380) .. (109.5900,36.0740) .. controls (109.2310,35.9080) and (108.8940,35.7540) .. (108.5810,35.6110) .. controls (108.2680,35.4680) and (107.9950,35.3080) .. (107.7620,35.1360) .. controls (107.5300,34.9620) and (107.4120,34.7470) .. (107.4120,34.4880) .. controls (107.4120,34.2090) and (107.5230,33.9880) .. (107.7470,33.8220) .. controls (107.9690,33.6580) and (108.2910,33.5760) .. (108.7110,33.5760) .. controls (109.1480,33.5760) and (109.5540,33.6250) .. (109.9300,33.7220) .. controls (110.3050,33.8240) and (110.5860,33.9330) .. (110.7740,34.0540) -- (111.3070,32.5250) .. controls (111.0130,32.3450) and (110.6360,32.2090) .. (110.1770,32.1150) .. controls (109.7170,32.0190) and (109.2280,31.9720) .. (108.7110,31.9720) .. controls (108.2630,31.9720) and (107.8570,32.0250) .. (107.4920,32.1300) .. controls (107.1270,32.2370) and (106.8100,32.4000) .. (106.5440,32.6200) .. controls (106.2780,32.8400) and (106.0700,33.1180) .. (105.9250,33.4500) .. controls (105.7800,33.7840) and (105.7080,34.1750) .. (105.7080,34.6240) .. controls (105.7080,35.1340) and (105.8100,35.5540) .. (106.0150,35.8840) .. controls (106.2200,36.2120) and (106.4790,36.4880) .. (106.7910,36.7060) .. controls (107.1030,36.9250) and (107.4380,37.1140) .. (107.7990,37.2740) .. controls (108.1600,37.4340) and (108.4950,37.5830) .. (108.8080,37.7210) .. controls (109.1210,37.8620) and (109.3660,38.0140) .. (109.5400,38.1840) .. controls (109.7140,38.3540) and (109.8030,38.5790) .. (109.8030,38.8540) .. controls (109.8030,39.2150) and (109.6600,39.4830) .. (109.3760,39.6610) .. controls (109.0910,39.8370) and (108.6810,39.9250) .. (108.1450,39.9250) .. controls (107.9260,39.9250) and (107.7140,39.9090) .. (107.5080,39.8740) .. controls (107.3010,39.8410) and (107.1050,39.7980) .. (106.9200,39.7450) .. controls (106.7330,39.6920) and (106.5660,39.6360) .. (106.4190,39.5750) .. controls (106.2700,39.5130) and (106.1480,39.4580) .. (106.0540,39.4070) -- (105.4770,40.9620) .. controls (105.7110,41.0890) and (106.0710,41.2160) .. (106.5570,41.3410)(99.0840,33.6880) .. controls (99.1830,33.6630) and (99.3220,33.6450) .. (99.5020,33.6370) .. controls (99.6820,33.6290) and (99.8640,33.6250) .. (100.0480,33.6250) .. controls (100.5230,33.6250) and (100.8830,33.7360) .. (101.1270,33.9570) .. controls (101.3720,34.1800) and (101.4940,34.4880) .. (101.4940,34.8810) .. controls (101.4940,35.4060) and (101.3450,35.7830) .. (101.0470,36.0120) .. controls (100.7490,36.2410) and (100.3500,36.3540) .. (99.8490,36.3540) -- (99.0840,36.3540) -- (99.0840,33.6880) -- cycle(99.0840,41.3410) -- (99.0840,37.5730) -- (100.0600,37.7820) -- (102.1280,41.3410) -- (104.1980,41.3410) -- (102.1160,37.8720) -- (101.4550,37.4330) .. controls (101.9840,37.2490) and (102.4070,36.9190) .. (102.7230,36.4470) .. controls (103.0400,35.9720) and (103.1990,35.3570) .. (103.1990,34.6050) .. controls (103.1990,34.1010) and (103.1050,33.6810) .. (102.9200,33.3430) .. controls (102.7330,33.0070) and (102.4850,32.7410) .. (102.1730,32.5480) .. controls (101.8610,32.3570) and (101.5080,32.2200) .. (101.1130,32.1420) .. controls (100.7180,32.0620) and (100.3190,32.0230) .. (99.9140,32.0230) .. controls (99.7360,32.0230) and (99.5420,32.0290) .. (99.3320,32.0370) .. controls (99.1210,32.0450) and (98.9040,32.0580) .. (98.6810,32.0760) .. controls (98.4580,32.0920) and (98.2340,32.1170) .. (98.0090,32.1480) .. controls (97.7840,32.1810) and (97.5740,32.2140) .. (97.3800,32.2480) -- (97.3800,41.3420) -- (99.0840,41.3420) -- cycle(95.2520,41.3410) -- (95.2520,39.7390) -- (91.4460,39.7390) -- (91.4460,37.4970) -- (94.8610,37.4970) -- (94.8610,35.8930) -- (91.4460,35.8930) -- (91.4460,33.7520) -- (95.1880,33.7520) -- (95.1880,32.1500) -- (89.7410,32.1500) -- (89.7410,41.3390) -- (95.2520,41.3390) -- cycle(82.6260,41.3410) .. controls (83.1120,41.4640) and (83.6730,41.5290) .. (84.3090,41.5290) .. controls (84.7910,41.5290) and (85.2350,41.4700) .. (85.6400,41.3530) .. controls (86.0450,41.2380) and (86.3900,41.0600) .. (86.6750,40.8240) .. controls (86.9590,40.5880) and (87.1810,40.2910) .. (87.3390,39.9330) .. controls (87.4970,39.5750) and (87.5760,39.1480) .. (87.5760,38.6540) .. controls (87.5760,38.1770) and (87.4840,37.7790) .. (87.3020,37.4570) .. controls (87.1200,37.1350) and (86.8900,36.8650) .. (86.6130,36.6460) .. controls (86.3360,36.4290) and (86.0180,36.2380) .. (85.6580,36.0740) .. controls (85.2990,35.9080) and (84.9620,35.7540) .. (84.6490,35.6110) .. controls (84.3360,35.4680) and (84.0630,35.3080) .. (83.8300,35.1360) .. controls (83.5970,34.9620) and (83.4800,34.7470) .. (83.4800,34.4880) .. controls (83.4800,34.2090) and (83.5920,33.9880) .. (83.8150,33.8220) .. controls (84.0370,33.6580) and (84.3590,33.5760) .. (84.7780,33.5760) .. controls (85.2160,33.5760) and (85.6220,33.6250) .. (85.9980,33.7220) .. controls (86.3730,33.8240) and (86.6540,33.9330) .. (86.8410,34.0540) -- (87.3750,32.5250) .. controls (87.0810,32.3450) and (86.7040,32.2090) .. (86.2450,32.1150) .. controls (85.7850,32.0190) and (85.2960,31.9720) .. (84.7780,31.9720) .. controls (84.3310,31.9720) and (83.9250,32.0250) .. (83.5600,32.1300) .. controls (83.1950,32.2370) and (82.8790,32.4000) .. (82.6120,32.6200) .. controls (82.3440,32.8410) and (82.1380,33.1180) .. (81.9930,33.4500) .. controls (81.8480,33.7840) and (81.7760,34.1750) .. (81.7760,34.6240) .. controls (81.7760,35.1340) and (81.8780,35.5540) .. (82.0830,35.8840) .. controls (82.2880,36.2120) and (82.5470,36.4880) .. (82.8590,36.7060) .. controls (83.1710,36.9240) and (83.5070,37.1140) .. (83.8670,37.2740) .. controls (84.2270,37.4340) and (84.5630,37.5830) .. (84.8760,37.7210) .. controls (85.1890,37.8620) and (85.4330,38.0140) .. (85.6080,38.1840) .. controls (85.7830,38.3540) and (85.8710,38.5790) .. (85.8710,38.8540) .. controls (85.8710,39.2150) and (85.7280,39.4830) .. (85.4430,39.6610) .. controls (85.1590,39.8370) and (84.7480,39.9250) .. (84.2130,39.9250) .. controls (83.9940,39.9250) and (83.7820,39.9090) .. (83.5760,39.8740) .. controls (83.3690,39.8410) and (83.1730,39.7980) .. (82.9880,39.7450) .. controls (82.8020,39.6920) and (82.6350,39.6360) .. (82.4870,39.5750) .. controls (82.3380,39.5130) and (82.2160,39.4580) .. (82.1210,39.4070) -- (81.5450,40.9620) .. controls (81.7800,41.0890) and (82.1400,41.2160) .. (82.6260,41.3410)(79.7610,32.1510) -- (78.0560,32.1510) -- (78.0560,41.3400) -- (79.7610,41.3400) -- (79.7610,32.1510) -- cycle(72.1060,35.8310) -- (72.3820,34.3760) -- (72.4460,34.3760) -- (72.7280,35.8170) -- (73.4070,37.8620) -- (71.4340,37.8620) -- (72.1060,35.8310) -- cycle(70.2820,41.3410) -- (70.9030,39.4640) -- (73.9260,39.4640) -- (74.5350,41.3410) -- (76.2400,41.3410) -- (72.9330,32.0870) -- (71.6470,32.0870) -- (68.4850,41.3410) -- (70.2820,41.3410) -- cycle(62.0580,41.3410) -- (62.0580,37.4560) -- (62.5690,37.4560) -- (65.2560,41.3410) -- (67.4730,41.3410) -- (64.4440,37.0620) -- (63.6760,36.5440) -- (64.3890,36.0480) -- (67.0700,32.1520) -- (65.0190,32.1520) -- (62.4820,36.0430) -- (62.0580,36.2210) -- (62.0580,32.1530) -- (60.3530,32.1530) -- (60.3530,41.3420) -- (62.0580,41.3420) -- cycle;
		% TECHNISCHE UNIVERSITÄT:
		\path[fill=TUred] (166.8350,23.1740) -- (166.8350,28.6690) -- (168.0690,28.6690) -- (168.0690,23.1740) -- (170.0790,23.1740) -- (170.0790,22.0140) -- (164.8210,22.0140) -- (164.8210,23.1740) -- (166.8350,23.1740) -- cycle(162.7020,21.4690) .. controls (162.8160,21.5720) and (162.9970,21.6240) .. (163.2450,21.6240) .. controls (163.4980,21.6240) and (163.6830,21.5720) .. (163.7950,21.4690) .. controls (163.9080,21.3650) and (163.9640,21.2260) .. (163.9640,21.0510) .. controls (163.9640,20.8740) and (163.9080,20.7310) .. (163.7950,20.6240) .. controls (163.6830,20.5170) and (163.4980,20.4640) .. (163.2450,20.4640) .. controls (162.9970,20.4640) and (162.8160,20.5180) .. (162.7020,20.6270) .. controls (162.5870,20.7350) and (162.5300,20.8770) .. (162.5300,21.0510) .. controls (162.5300,21.2260) and (162.5870,21.3650) .. (162.7020,21.4690)(160.6690,21.4690) .. controls (160.7840,21.5720) and (160.9680,21.6240) .. (161.2210,21.6240) .. controls (161.4660,21.6240) and (161.6460,21.5720) .. (161.7600,21.4690) .. controls (161.8740,21.3650) and (161.9320,21.2260) .. (161.9320,21.0510) .. controls (161.9320,20.8740) and (161.8750,20.7310) .. (161.7620,20.6240) .. controls (161.6500,20.5170) and (161.4690,20.4640) .. (161.2210,20.4640) .. controls (160.9680,20.4640) and (160.7840,20.5180) .. (160.6690,20.6270) .. controls (160.5560,20.7350) and (160.4970,20.8770) .. (160.4970,21.0510) .. controls (160.4970,21.2260) and (160.5560,21.3650) .. (160.6690,21.4690)(161.9410,24.6780) -- (162.1400,23.6240) -- (162.1870,23.6240) -- (162.3910,24.6680) -- (162.8820,26.1490) -- (161.4530,26.1490) -- (161.9410,24.6780) -- cycle(160.6200,28.6690) -- (161.0700,27.3090) -- (163.2580,27.3090) -- (163.6990,28.6690) -- (164.9330,28.6690) -- (162.5380,21.9670) -- (161.6060,21.9670) -- (159.3170,28.6690) -- (160.6200,28.6690) -- cycle(156.2580,23.1740) -- (156.2580,28.6690) -- (157.4920,28.6690) -- (157.4920,23.1740) -- (159.5020,23.1740) -- (159.5020,22.0140) -- (154.2440,22.0140) -- (154.2440,23.1740) -- (156.2580,23.1740) -- cycle(153.3940,22.0140) -- (152.1600,22.0140) -- (152.1600,28.6690) -- (153.3940,28.6690) -- (153.3940,22.0140) -- cycle(147.3960,28.6680) .. controls (147.7480,28.7580) and (148.1540,28.8030) .. (148.6150,28.8030) .. controls (148.9640,28.8030) and (149.2850,28.7610) .. (149.5790,28.6770) .. controls (149.8720,28.5930) and (150.1220,28.4650) .. (150.3280,28.2940) .. controls (150.5340,28.1220) and (150.6940,27.9070) .. (150.8080,27.6480) .. controls (150.9220,27.3900) and (150.9800,27.0810) .. (150.9800,26.7220) .. controls (150.9800,26.3770) and (150.9140,26.0880) .. (150.7820,25.8540) .. controls (150.6500,25.6210) and (150.4840,25.4260) .. (150.2830,25.2680) .. controls (150.0830,25.1110) and (149.8520,24.9720) .. (149.5920,24.8530) .. controls (149.3310,24.7340) and (149.0880,24.6220) .. (148.8620,24.5190) .. controls (148.6340,24.4150) and (148.4370,24.3000) .. (148.2690,24.1740) .. controls (148.1000,24.0490) and (148.0160,23.8930) .. (148.0160,23.7060) .. controls (148.0160,23.5040) and (148.0950,23.3430) .. (148.2570,23.2230) .. controls (148.4180,23.1040) and (148.6520,23.0440) .. (148.9550,23.0440) .. controls (149.2710,23.0440) and (149.5660,23.0800) .. (149.8370,23.1520) .. controls (150.1090,23.2240) and (150.3140,23.3040) .. (150.4480,23.3920) -- (150.8350,22.2850) .. controls (150.6220,22.1540) and (150.3500,22.0550) .. (150.0170,21.9870) .. controls (149.6840,21.9180) and (149.3290,21.8840) .. (148.9550,21.8840) .. controls (148.6300,21.8840) and (148.3360,21.9220) .. (148.0720,21.9990) .. controls (147.8080,22.0760) and (147.5790,22.1940) .. (147.3850,22.3530) .. controls (147.1920,22.5130) and (147.0420,22.7130) .. (146.9380,22.9550) .. controls (146.8330,23.1960) and (146.7800,23.4790) .. (146.7800,23.8040) .. controls (146.7800,24.1740) and (146.8540,24.4780) .. (147.0030,24.7160) .. controls (147.1510,24.9550) and (147.3390,25.1530) .. (147.5650,25.3120) .. controls (147.7910,25.4710) and (148.0350,25.6080) .. (148.2940,25.7240) .. controls (148.5560,25.8390) and (148.7990,25.9480) .. (149.0250,26.0480) .. controls (149.2520,26.1490) and (149.4280,26.2610) .. (149.5550,26.3840) .. controls (149.6820,26.5070) and (149.7450,26.6680) .. (149.7450,26.8690) .. controls (149.7450,27.1300) and (149.6410,27.3240) .. (149.4350,27.4520) .. controls (149.2290,27.5790) and (148.9320,27.6430) .. (148.5430,27.6430) .. controls (148.3850,27.6430) and (148.2310,27.6310) .. (148.0820,27.6060) .. controls (147.9330,27.5820) and (147.7900,27.5510) .. (147.6560,27.5130) .. controls (147.5210,27.4740) and (147.4000,27.4330) .. (147.2930,27.3890) .. controls (147.1860,27.3450) and (147.0980,27.3040) .. (147.0290,27.2670) -- (146.6110,28.3940) .. controls (146.7820,28.4860) and (147.0430,28.5770) .. (147.3960,28.6680)(142.4110,23.1280) .. controls (142.4820,23.1090) and (142.5830,23.0970) .. (142.7140,23.0900) .. controls (142.8440,23.0840) and (142.9770,23.0810) .. (143.1100,23.0810) .. controls (143.4540,23.0810) and (143.7140,23.1620) .. (143.8920,23.3220) .. controls (144.0690,23.4830) and (144.1570,23.7060) .. (144.1570,23.9910) .. controls (144.1570,24.3710) and (144.0490,24.6440) .. (143.8340,24.8100) .. controls (143.6180,24.9750) and (143.3290,25.0580) .. (142.9660,25.0580) -- (142.4120,25.0580) -- (142.4120,23.1280) -- cycle(142.4110,28.6690) -- (142.4110,25.9400) -- (143.1180,26.0910) -- (144.6150,28.6690) -- (146.1140,28.6690) -- (144.6060,26.1570) -- (144.1270,25.8380) .. controls (144.5100,25.7050) and (144.8160,25.4670) .. (145.0450,25.1230) .. controls (145.2750,24.7800) and (145.3900,24.3360) .. (145.3900,23.7910) .. controls (145.3900,23.4260) and (145.3230,23.1210) .. (145.1880,22.8770) .. controls (145.0520,22.6330) and (144.8730,22.4410) .. (144.6470,22.3020) .. controls (144.4200,22.1620) and (144.1660,22.0640) .. (143.8790,22.0070) .. controls (143.5920,21.9500) and (143.3040,21.9210) .. (143.0100,21.9210) .. controls (142.8820,21.9210) and (142.7410,21.9240) .. (142.5890,21.9300) .. controls (142.4370,21.9370) and (142.2790,21.9460) .. (142.1180,21.9580) .. controls (141.9560,21.9710) and (141.7940,21.9880) .. (141.6310,22.0110) .. controls (141.4680,22.0350) and (141.3170,22.0590) .. (141.1760,22.0830) -- (141.1760,28.6690) -- (142.4110,28.6690) -- cycle(140.1590,28.6690) -- (140.1590,27.5080) -- (137.4030,27.5080) -- (137.4030,25.8840) -- (139.8760,25.8840) -- (139.8760,24.7240) -- (137.4030,24.7240) -- (137.4030,23.1740) -- (140.1130,23.1740) -- (140.1130,22.0140) -- (136.1680,22.0140) -- (136.1680,28.6690) -- (140.1590,28.6690) -- cycle(132.1210,28.7150) -- (133.0490,28.7150) -- (135.5030,22.0140) -- (134.2700,22.0140) -- (132.9980,25.9120) -- (132.8090,27.0540) -- (132.7620,27.0540) -- (132.5900,25.9210) -- (131.2550,22.0140) -- (129.7500,22.0140) -- (132.1210,28.7150) -- cycle(129.0200,22.0140) -- (127.7860,22.0140) -- (127.7860,28.6690) -- (129.0200,28.6690) -- (129.0200,22.0140) -- cycle(122.5950,28.6690) -- (122.5950,25.1970) -- (122.4430,24.1530) -- (122.4940,24.1530) -- (123.0150,25.1970) -- (125.5160,28.7150) -- (126.4620,28.7150) -- (126.4620,22.0140) -- (125.2280,22.0140) -- (125.2280,25.5130) -- (125.3800,26.5290) -- (125.3340,26.5290) -- (124.8300,25.5120) -- (122.3160,21.9670) -- (121.3620,21.9670) -- (121.3620,28.6690) -- (122.5950,28.6690) -- cycle(117.7930,28.7860) .. controls (118.1410,28.7860) and (118.4600,28.7370) .. (118.7530,28.6410) .. controls (119.0450,28.5450) and (119.2930,28.3960) .. (119.4960,28.1930) .. controls (119.6990,27.9900) and (119.8570,27.7340) .. (119.9700,27.4230) .. controls (120.0830,27.1120) and (120.1400,26.7410) .. (120.1400,26.3110) -- (120.1400,22.0140) -- (118.9060,22.0140) -- (118.9060,26.2180) .. controls (118.9060,26.7100) and (118.8160,27.0680) .. (118.6370,27.2900) .. controls (118.4590,27.5130) and (118.1700,27.6250) .. (117.7710,27.6250) .. controls (117.5700,27.6250) and (117.3930,27.6010) .. (117.2410,27.5530) .. controls (117.0890,27.5050) and (116.9620,27.4250) .. (116.8600,27.3140) .. controls (116.7560,27.2020) and (116.6820,27.0570) .. (116.6330,26.8770) .. controls (116.5850,26.6980) and (116.5620,26.4780) .. (116.5620,26.2180) -- (116.5620,22.0140) -- (115.3280,22.0140) -- (115.3280,26.4740) .. controls (115.3270,28.0140) and (116.1490,28.7860) .. (117.7930,28.7860)(111.8860,28.6690) -- (111.8860,27.5080) -- (109.1290,27.5080) -- (109.1290,25.8840) -- (111.6030,25.8840) -- (111.6030,24.7240) -- (109.1290,24.7240) -- (109.1290,23.1740) -- (111.8390,23.1740) -- (111.8390,22.0140) -- (107.8960,22.0140) -- (107.8960,28.6690) -- (111.8860,28.6690) -- cycle(102.8960,28.6690) -- (102.8960,25.8840) -- (105.3360,25.8840) -- (105.3360,28.6690) -- (106.5710,28.6690) -- (106.5710,22.0140) -- (105.3360,22.0140) -- (105.3360,24.7240) -- (102.8960,24.7240) -- (102.8960,22.0140) -- (101.6620,22.0140) -- (101.6620,28.6690) -- (102.8960,28.6690) -- cycle(96.3130,26.9470) .. controls (96.4580,27.3870) and (96.6530,27.7450) .. (96.8990,28.0200) .. controls (97.1440,28.2950) and (97.4450,28.4940) .. (97.8020,28.6180) .. controls (98.1590,28.7420) and (98.5360,28.8030) .. (98.9360,28.8030) .. controls (99.2630,28.8030) and (99.5850,28.7720) .. (99.8990,28.7090) .. controls (100.2120,28.6460) and (100.4720,28.5420) .. (100.6750,28.3980) -- (100.4060,27.3360) .. controls (100.2600,27.4260) and (100.0890,27.5000) .. (99.8930,27.5570) .. controls (99.6960,27.6140) and (99.4560,27.6430) .. (99.1710,27.6430) .. controls (98.8680,27.6430) and (98.6000,27.5870) .. (98.3680,27.4760) .. controls (98.1360,27.3640) and (97.9430,27.2080) .. (97.7880,27.0080) .. controls (97.6340,26.8080) and (97.5180,26.5670) .. (97.4430,26.2850) .. controls (97.3670,26.0030) and (97.3290,25.6900) .. (97.3290,25.3460) .. controls (97.3290,24.5580) and (97.4890,23.9780) .. (97.8090,23.6040) .. controls (98.1290,23.2310) and (98.5520,23.0440) .. (99.0780,23.0440) .. controls (99.3630,23.0440) and (99.6050,23.0600) .. (99.8050,23.0910) .. controls (100.0040,23.1220) and (100.1770,23.1730) .. (100.3220,23.2440) -- (100.5770,22.1390) .. controls (100.4070,22.0680) and (100.1910,22.0080) .. (99.9280,21.9580) .. controls (99.6650,21.9090) and (99.3430,21.8840) .. (98.9620,21.8840) .. controls (98.6070,21.8840) and (98.2520,21.9430) .. (97.8970,22.0610) .. controls (97.5430,22.1780) and (97.2360,22.3720) .. (96.9750,22.6410) .. controls (96.7140,22.9110) and (96.5020,23.2660) .. (96.3390,23.7060) .. controls (96.1760,24.1450) and (96.0940,24.6920) .. (96.0940,25.3460) .. controls (96.0950,25.9730) and (96.1680,26.5070) .. (96.3130,26.9470)(91.6750,28.6680) .. controls (92.0270,28.7580) and (92.4330,28.8030) .. (92.8940,28.8030) .. controls (93.2430,28.8030) and (93.5640,28.7610) .. (93.8570,28.6770) .. controls (94.1510,28.5930) and (94.4010,28.4650) .. (94.6070,28.2940) .. controls (94.8130,28.1220) and (94.9730,27.9070) .. (95.0870,27.6480) .. controls (95.2020,27.3900) and (95.2590,27.0810) .. (95.2590,26.7220) .. controls (95.2590,26.3770) and (95.1930,26.0880) .. (95.0610,25.8540) .. controls (94.9290,25.6210) and (94.7630,25.4260) .. (94.5620,25.2680) .. controls (94.3620,25.1110) and (94.1310,24.9720) .. (93.8710,24.8530) .. controls (93.6100,24.7340) and (93.3660,24.6220) .. (93.1400,24.5190) .. controls (92.9130,24.4150) and (92.7150,24.3000) .. (92.5470,24.1740) .. controls (92.3780,24.0490) and (92.2940,23.8930) .. (92.2940,23.7060) .. controls (92.2940,23.5040) and (92.3740,23.3430) .. (92.5360,23.2230) .. controls (92.6970,23.1040) and (92.9290,23.0440) .. (93.2340,23.0440) .. controls (93.5500,23.0440) and (93.8450,23.0800) .. (94.1160,23.1520) .. controls (94.3880,23.2240) and (94.5920,23.3040) .. (94.7270,23.3920) -- (95.1140,22.2850) .. controls (94.9010,22.1540) and (94.6280,22.0550) .. (94.2950,21.9870) .. controls (93.9620,21.9180) and (93.6080,21.8840) .. (93.2340,21.8840) .. controls (92.9090,21.8840) and (92.6150,21.9220) .. (92.3510,21.9990) .. controls (92.0870,22.0760) and (91.8580,22.1940) .. (91.6650,22.3530) .. controls (91.4710,22.5130) and (91.3220,22.7130) .. (91.2170,22.9550) .. controls (91.1120,23.1960) and (91.0590,23.4790) .. (91.0590,23.8040) .. controls (91.0590,24.1740) and (91.1330,24.4780) .. (91.2820,24.7160) .. controls (91.4300,24.9550) and (91.6180,25.1530) .. (91.8430,25.3120) .. controls (92.0690,25.4710) and (92.3120,25.6080) .. (92.5730,25.7240) .. controls (92.8340,25.8390) and (93.0780,25.9480) .. (93.3040,26.0480) .. controls (93.5310,26.1490) and (93.7080,26.2610) .. (93.8340,26.3840) .. controls (93.9610,26.5070) and (94.0250,26.6680) .. (94.0250,26.8690) .. controls (94.0250,27.1300) and (93.9210,27.3240) .. (93.7150,27.4520) .. controls (93.5090,27.5790) and (93.2120,27.6430) .. (92.8240,27.6430) .. controls (92.6660,27.6430) and (92.5120,27.6310) .. (92.3620,27.6060) .. controls (92.2130,27.5820) and (92.0710,27.5510) .. (91.9370,27.5130) .. controls (91.8020,27.4740) and (91.6810,27.4330) .. (91.5740,27.3890) .. controls (91.4660,27.3450) and (91.3780,27.3040) .. (91.3090,27.2670) -- (90.8920,28.3940) .. controls (91.0620,28.4860) and (91.3230,28.5770) .. (91.6750,28.6680)(89.8370,22.0140) -- (88.6030,22.0140) -- (88.6030,28.6690) -- (89.8370,28.6690) -- (89.8370,22.0140) -- cycle(83.4140,28.6690) -- (83.4140,25.1970) -- (83.2600,24.1530) -- (83.3110,24.1530) -- (83.8310,25.1970) -- (86.3330,28.7150) -- (87.2790,28.7150) -- (87.2790,22.0140) -- (86.0450,22.0140) -- (86.0450,25.5130) -- (86.1970,26.5290) -- (86.1510,26.5290) -- (85.6470,25.5120) -- (83.1330,21.9670) -- (82.1790,21.9670) -- (82.1790,28.6690) -- (83.4140,28.6690) -- cycle(77.1800,28.6690) -- (77.1800,25.8840) -- (79.6210,25.8840) -- (79.6210,28.6690) -- (80.8550,28.6690) -- (80.8550,22.0140) -- (79.6210,22.0140) -- (79.6210,24.7240) -- (77.1800,24.7240) -- (77.1800,22.0140) -- (75.9450,22.0140) -- (75.9450,28.6690) -- (77.1800,28.6690) -- cycle(70.5980,26.9470) .. controls (70.7430,27.3870) and (70.9380,27.7450) .. (71.1830,28.0200) .. controls (71.4290,28.2950) and (71.7300,28.4940) .. (72.0870,28.6180) .. controls (72.4430,28.7420) and (72.8210,28.8030) .. (73.2200,28.8030) .. controls (73.5480,28.8030) and (73.8690,28.7720) .. (74.1830,28.7090) .. controls (74.4970,28.6460) and (74.7560,28.5420) .. (74.9600,28.3980) -- (74.6910,27.3360) .. controls (74.5460,27.4260) and (74.3750,27.5000) .. (74.1780,27.5570) .. controls (73.9820,27.6140) and (73.7410,27.6430) .. (73.4570,27.6430) .. controls (73.1540,27.6430) and (72.8860,27.5870) .. (72.6540,27.4760) .. controls (72.4220,27.3640) and (72.2290,27.2080) .. (72.0740,27.0080) .. controls (71.9190,26.8080) and (71.8040,26.5670) .. (71.7280,26.2850) .. controls (71.6520,26.0030) and (71.6140,25.6900) .. (71.6140,25.3460) .. controls (71.6140,24.5580) and (71.7750,23.9780) .. (72.0950,23.6040) .. controls (72.4150,23.2310) and (72.8380,23.0440) .. (73.3640,23.0440) .. controls (73.6490,23.0440) and (73.8910,23.0600) .. (74.0900,23.0910) .. controls (74.2900,23.1220) and (74.4620,23.1730) .. (74.6080,23.2440) -- (74.8630,22.1390) .. controls (74.6930,22.0680) and (74.4760,22.0080) .. (74.2130,21.9580) .. controls (73.9500,21.9090) and (73.6290,21.8840) .. (73.2480,21.8840) .. controls (72.8920,21.8840) and (72.5370,21.9430) .. (72.1830,22.0610) .. controls (71.8290,22.1780) and (71.5210,22.3720) .. (71.2610,22.6410) .. controls (71.0000,22.9110) and (70.7880,23.2660) .. (70.6250,23.7060) .. controls (70.4620,24.1450) and (70.3800,24.6920) .. (70.3800,25.3460) .. controls (70.3800,25.9730) and (70.4530,26.5070) .. (70.5980,26.9470)(69.5710,28.6690) -- (69.5710,27.5080) -- (66.8150,27.5080) -- (66.8150,25.8840) -- (69.2880,25.8840) -- (69.2880,24.7240) -- (66.8150,24.7240) -- (66.8150,23.1740) -- (69.5250,23.1740) -- (69.5250,22.0140) -- (65.5800,22.0140) -- (65.5800,28.6690) -- (69.5710,28.6690) -- cycle(61.6160,23.1740) -- (61.6160,28.6690) -- (62.8500,28.6690) -- (62.8500,23.1740) -- (64.8600,23.1740) -- (64.8600,22.0140) -- (59.6020,22.0140) -- (59.6020,23.1740) -- (61.6160,23.1740) -- cycle;

		% Top part:
		\path[fill = TUblue] (31.2450,17.5150) -- (31.2450,20.8090) -- (52.6440,20.8090) -- (52.6460,17.3360) -- (41.8880,14.1730) -- cycle;
		% Left part:
		\path[fill = TUblue] (34.0260,22.0630) -- (38.3400,22.0630) -- (32.6470,41.3700) -- (28.3470,41.3700) -- cycle;
		% Top rectangle:
		\path[fill = TUblue,rounded corners=0.0000cm] (45.8280,22.0540) rectangle (50.5940,26.6430);
		% Middle rectangle:
		\path[fill = TUred,rounded corners=0.0000cm] (45.8280,29.4170) rectangle (50.5940,34.0070);
		% Bottom rectangle:
		\path[fill = TUblue,rounded corners=0.0000cm] (45.8280,36.7800) rectangle (50.5940,41.3700);
	\end{tikzpicture}
}

% Colors needed for the logo (sketchy) of the Department of Computer Science of the University of Kaiserslautern (black when in grayscale mode)
\iftoggle{bwmode}{
	\definecolor{ce5e8f5}{RGB}{0,0,0}
	\definecolor{cdfdbe2}{RGB}{0,0,0}
	\definecolor{cafcde9}{RGB}{0,0,0}
	\definecolor{c9c9afc}{RGB}{0,0,0}
	\definecolor{cb9babc}{RGB}{0,0,0}
	\definecolor{cddae9b}{RGB}{0,0,0}
	\definecolor{cf8776f}{RGB}{0,0,0}
	\definecolor{cac9b8b}{RGB}{0,0,0}
	\definecolor{c878cb9}{RGB}{0,0,0}
	\definecolor{c848387}{RGB}{0,0,0}
	\definecolor{c6c6898}{RGB}{0,0,0}
	\definecolor{cf52d21}{RGB}{0,0,0}
	\definecolor{c0503fc}{RGB}{0,0,0}
	\definecolor{cfa0305}{RGB}{0,0,0}
}{
	\definecolor{ce5e8f5}{RGB}{229,232,245}
	\definecolor{cdfdbe2}{RGB}{223,219,226}
	\definecolor{cafcde9}{RGB}{175,205,233}
	\definecolor{c9c9afc}{RGB}{156,154,252}
	\definecolor{cb9babc}{RGB}{185,186,188}
	\definecolor{cddae9b}{RGB}{221,174,155}
	\definecolor{cf8776f}{RGB}{248,119,111}
	\definecolor{cac9b8b}{RGB}{172,155,139}
	\definecolor{c878cb9}{RGB}{135,140,185}
	\definecolor{c848387}{RGB}{132,131,135}
	\definecolor{c6c6898}{RGB}{108,104,152}
	\definecolor{cf52d21}{RGB}{245,45,33}
	\definecolor{c0503fc}{RGB}{5,3,252}
	\definecolor{cfa0305}{RGB}{250,3,5}
}

% The logo (sketchy) of the Department of Computer Science of the University of Kaiserslautern:
% Taken from: http://dekanat.informatik.uni-kl.de/logo_dekanat_400x145.png on the 2016-12-14
% Converted to SVG using: vectorizer (https://www.vectorizer.io/)
% Manipulated using: Inkscape (https://inkscape.org/)
% Converted to TikZ using: svg2tikz (https://github.com/kjellmf/svg2tikz) as an Inkscape extension
\newcommand{\CSLogoSketchy}{
	\begin{tikzpicture}[
		y = 0.1pt,
		x = 0.1pt,
		yscale = -1,
		xscale = 1,
	]
		\begin{scope}[fill = ce5e8f5]
			\path[fill] (1043,1330) .. controls (1043,1305) and   (1045,1295) .. (1047,1308) .. controls   (1049,1320) and (1049,1340) .. (1047,1353) ..   controls (1045,1365) and (1043,1355) ..   (1043,1330) -- cycle;
			\path[fill] (1043,1050) .. controls (1043,1020) and   (1045,1007) .. (1047,1023) .. controls   (1049,1038) and (1049,1062) .. (1047,1078) ..   controls (1045,1093) and (1043,1080) ..   (1043,1050) -- cycle;
			\path[fill] (780,890) .. controls (767,882) and   (768,880) .. (783,880) .. controls (792,880) and   (800,885) .. (800,890) .. controls (800,902) and   (799,902) .. (780,890) -- cycle;
			\path[fill] (774,640) .. controls (774,516) and   (776,466) .. (777,528) .. controls (779,589) and   (779,691) .. (777,753) .. controls (776,814) and   (774,764) .. (774,640) -- cycle;
			\path[fill] (838,853) .. controls (844,851) and   (856,851) .. (863,853) .. controls (869,856) and   (864,858) .. (850,858) .. controls (836,858) and   (831,856) .. (838,853) -- cycle;
			\path[fill] (1043,715) .. controls (1043,682) and   (1045,670) .. (1047,688) .. controls (1049,706)   and (1049,733) .. (1047,748) .. controls   (1045,763) and (1043,748) .. (1043,715) --   cycle;
			\path[fill] (1043,420) .. controls (1043,384) and   (1045,370) .. (1047,388) .. controls (1049,405)   and (1049,435) .. (1047,453) .. controls   (1045,470) and (1043,456) .. (1043,420) --   cycle;
			\path[fill] (133,265) .. controls (133,221) and   (135,204) .. (137,228) .. controls (139,251) and   (139,287) .. (137,308) .. controls (135,328) and   (133,309) .. (133,265) -- cycle;
		\end{scope}
		\begin{scope}[fill = cdfdbe2]
			\path[fill] (933,1283) .. controls (942,1281) and (958,1281) .. (968,1283) .. controls (977,1286) and (969,1288) .. (950,1288) .. controls (931,1288) and (923,1286) .. (933,1283) -- cycle;
			\path[fill] (1090,1283) -- (1129,1279) -- (1133,1202) -- (1136,1125) -- (1136,1205) -- (1135,1285) -- (1092,1286) -- (1050,1287) -- (1090,1283) -- cycle;
			\path[fill] (909,923) .. controls (896,907) and (897,906) .. (913,919) .. controls (922,926) and (930,934) .. (930,936) .. controls (930,944) and (922,939) .. (909,923) -- cycle;
			\path[fill] (823,903) .. controls (838,901) and (860,901) .. (873,903) .. controls (885,905) and (873,907) .. (845,907) .. controls (818,907) and (807,905) .. (823,903) -- cycle;
			\path[fill] (880,846) .. controls (880,844) and (888,836) .. (898,829) .. controls (913,816) and (914,817) .. (901,833) .. controls (888,849) and (880,854) .. (880,846) -- cycle;
			\path[fill] (1133,580) .. controls (1133,533) and (1135,514) .. (1137,538) .. controls (1139,561) and (1139,599) .. (1137,623) .. controls (1135,646) and (1133,627) .. (1133,580) -- cycle;
			\path[fill=cdfdbe2] (828,393) .. controls (856,391) and (904,391) .. (933,393) .. controls (961,395) and (938,396) .. (880,396) .. controls (822,396) and (799,395) .. (828,393) -- cycle;
			\end{scope}
		\begin{scope}[fill = cafcde9]
			\path[fill] (328,383) .. controls (356,381) and (404,381) .. (433,383) .. controls (461,385) and (438,386) .. (380,386) .. controls (322,386) and (299,385) .. (328,383) -- cycle;
			\path[fill] (678,13) .. controls (685,10) and (694,11) .. (697,14) .. controls (701,17) and (695,20) .. (684,19) .. controls (673,19) and (670,16) .. (678,13) -- cycle;
		\end{scope}
		\begin{scope}[fill = c9c9afc]
			\path[fill] (993,1343) .. controls (985,1341) and (980,1321) .. (980,1299) .. controls (980,1267) and (983,1260) .. (1000,1260) .. controls (1017,1260) and (1020,1267) .. (1020,1305) .. controls (1020,1349) and (1017,1353) .. (993,1343) -- cycle;
			\path[fill] (56,1188) .. controls (79,1104) and (103,1019) .. (109,1000) .. controls (115,981) and (148,866) .. (181,745) .. controls (214,624) and (250,495) .. (261,458) -- (281,390) -- (381,390) .. controls (458,390) and (481,393) .. (477,403) .. controls (475,409) and (436,543) .. (390,700) .. controls (345,857) and (285,1064) .. (256,1160) -- (204,1335) -- (108,1338) -- (13,1341) -- (56,1188) -- cycle;
			\path[fill] (980,1046) .. controls (980,997) and (982,992) .. (1000,997) .. controls (1017,1001) and (1020,1011) .. (1020,1051) .. controls (1020,1093) and (1017,1100) .. (1000,1100) .. controls (982,1100) and (980,1093) .. (980,1046) -- cycle;
			\path[fill] (780,640) -- (780,400) -- (880,400) -- (980,400) -- (980,365) -- (980,330) -- (560,330) -- (140,330) -- (140,257) -- (140,183) -- (363,115) .. controls (485,78) and (608,41) .. (636,32) -- (688,17) -- (891,79) .. controls (1003,113) and (1127,151) .. (1165,163) -- (1235,185) -- (1238,258) -- (1241,330) -- (1130,330) -- (1020,330) -- (1020,405) .. controls (1020,473) and (1018,480) .. (1000,480) .. controls (984,480) and (980,473) .. (980,445) -- (980,410) -- (890,410) -- (800,410) -- (800,635) -- (800,860) -- (830,860) .. controls (847,860) and (860,865) .. (860,870) .. controls (860,876) and (842,880) .. (820,880) -- (780,880) -- (780,640) -- cycle;
			\path[fill] (980,688) .. controls (980,647) and (983,640) .. (1000,640) .. controls (1017,640) and (1020,647) .. (1020,688) .. controls (1020,729) and (1017,736) .. (1000,736) .. controls (983,736) and (980,729) .. (980,688) -- cycle;
		\end{scope}
		\begin{scope}[fill = cb9babc]
			\path[fill] (868,433) .. controls (891,431) and (927,431) .. (948,433) .. controls (968,435) and (949,437) .. (905,437) .. controls (861,437) and (844,435) .. (868,433) -- cycle;
			\path[fill] (363,353) .. controls (477,351) and (663,351) .. (778,353) .. controls (892,354) and (798,355) .. (570,355) .. controls (342,355) and (248,354) .. (363,353) -- cycle;
			\path[fill] (1093,353) .. controls (1124,351) and (1176,351) .. (1208,353) .. controls (1239,355) and (1213,356) .. (1150,356) .. controls (1087,356) and (1061,355) .. (1093,353) -- cycle;
		\end{scope}
		\begin{scope}[fill = cddae9b]
			\path[fill] (1080,815) .. controls (1056,790) and (1038,770) .. (1041,770) .. controls (1044,770) and (1066,790) .. (1090,815) .. controls (1114,840) and (1132,860) .. (1129,860) .. controls (1126,860) and (1104,840) .. (1080,815) -- cycle;
		\end{scope}
		\begin{scope}[fill = cf8776f]
			\path[fill] (959,963) -- (935,935) -- (963,959) .. controls (988,982) and (995,990) .. (987,990) .. controls (985,990) and (973,978) .. (959,963) -- cycle;
		\end{scope}
		\begin{scope}[fill = cac9b8b]
			\path[fill] (1090,945) .. controls (1120,915) and (1147,890) .. (1149,890) .. controls (1152,890) and (1130,915) .. (1100,945) .. controls (1070,975) and (1043,1000) .. (1041,1000) .. controls (1038,1000) and (1060,975) .. (1090,945) -- cycle;
		\end{scope}
		\begin{scope}[fill = c878cb9]
			\path[fill] (77,1343) .. controls (101,1341) and (139,1341) .. (162,1343) .. controls (186,1345) and (167,1347) .. (120,1347) .. controls (73,1347) and (54,1345) .. (77,1343) -- cycle;
			\path[fill] (805,635) -- (805,415) -- (890,414) -- (975,414) -- (893,417) -- (810,421) -- (807,638) -- (804,855) -- (805,635) -- cycle;
			\path[fill] (363,333) .. controls (477,331) and (663,331) .. (778,333) .. controls (892,334) and (798,335) .. (570,335) .. controls (342,335) and (248,334) .. (363,333) -- cycle;
			\path[fill] (1073,333) .. controls (1104,331) and (1156,331) .. (1188,333) .. controls (1219,335) and (1193,336) .. (1130,336) .. controls (1067,336) and (1041,335) .. (1073,333) -- cycle;
		\end{scope}
		\begin{scope}[fill = c848387]
			\path[fill] (1014,1354) .. controls (1017,1345) and (1020,1323) .. (1020,1304) .. controls (1020,1270) and (1020,1270) .. (1065,1270) -- (1110,1270) -- (1110,1195) .. controls (1110,1152) and (1114,1120) .. (1120,1120) .. controls (1126,1120) and (1130,1153) .. (1130,1200) -- (1130,1280) -- (1085,1280) -- (1040,1280) -- (1040,1325) .. controls (1040,1360) and (1036,1370) .. (1024,1370) .. controls (1013,1370) and (1010,1365) .. (1014,1354) -- cycle;
			\path[fill] (113,1353) .. controls (193,1350) and (202,1347) .. (210,1327) .. controls (219,1304) and (247,1211) .. (393,705) .. controls (439,546) and (482,414) .. (489,412) .. controls (495,410) and (500,412) .. (500,417) .. controls (500,425) and (422,694) .. (274,1198) -- (227,1360) -- (126,1358) -- (25,1356) -- (113,1353) -- cycle;
			\path[fill] (933,1273) .. controls (942,1271) and (958,1271) .. (968,1273) .. controls (977,1276) and (969,1278) .. (950,1278) .. controls (931,1278) and (923,1276) .. (933,1273) -- cycle;
			\path[fill] (1020,1042) .. controls (1020,985) and (1021,984) .. (1079,926) .. controls (1111,894) and (1140,872) .. (1143,877) .. controls (1146,882) and (1125,911) .. (1095,940) .. controls (1042,992) and (1040,996) .. (1040,1047) .. controls (1040,1076) and (1036,1100) .. (1030,1100) .. controls (1024,1100) and (1020,1074) .. (1020,1042) -- cycle;
			\path[fill] (800,890) .. controls (800,885) and (815,880) .. (834,880) .. controls (853,880) and (872,885) .. (875,890) .. controls (879,896) and (865,900) .. (841,900) .. controls (818,900) and (800,896) .. (800,890) -- cycle;
			\path[fill] (812,643) -- (810,420) -- (898,422) -- (985,424) -- (903,427) -- (820,431) -- (817,648) -- (815,865) -- (812,643) -- cycle;
			\path[fill] (1028,754) .. controls (1023,750) and (1020,723) .. (1020,693) -- (1020,640) -- (1065,640) -- (1110,640) -- (1110,575) .. controls (1110,538) and (1114,510) .. (1120,510) .. controls (1126,510) and (1130,542) .. (1130,585) -- (1130,660) -- (1086,660) -- (1041,660) -- (1038,711) .. controls (1036,739) and (1032,758) .. (1028,754) -- cycle;
			\path[fill] (920,650) .. controls (920,645) and (934,640) .. (950,640) .. controls (967,640) and (980,645) .. (980,650) .. controls (980,656) and (967,660) .. (950,660) .. controls (934,660) and (920,656) .. (920,650) -- cycle;
			\path[fill] (1020,410) -- (1020,340) -- (1130,340) -- (1240,340) -- (1240,270) .. controls (1240,230) and (1244,200) .. (1250,200) .. controls (1256,200) and (1260,232) .. (1260,275) -- (1260,350) -- (1150,350) -- (1040,350) -- (1040,415) .. controls (1040,452) and (1036,480) .. (1030,480) .. controls (1024,480) and (1020,450) .. (1020,410) -- cycle;
			\path[fill] (363,343) .. controls (477,341) and (663,341) .. (778,343) .. controls (892,344) and (798,345) .. (570,345) .. controls (342,345) and (248,344) .. (363,343) -- cycle;
		\end{scope}
		\begin{scope}[fill = c6c6898]
			\path[fill] (933,1263) .. controls (942,1261) and (958,1261) .. (968,1263) .. controls (977,1266) and (969,1268) .. (950,1268) .. controls (931,1268) and (923,1266) .. (933,1263) -- cycle;
			\path[fill] (1043,1263) .. controls (1058,1261) and (1080,1261) .. (1093,1263) .. controls (1105,1265) and (1093,1267) .. (1065,1267) .. controls (1038,1267) and (1027,1265) .. (1043,1263) -- cycle;
		\end{scope}
		\begin{scope}[fill = cf52d21]
			\path[fill] (1065,930) .. controls (1098,897) and (1127,870) .. (1129,870) .. controls (1132,870) and (1108,897) .. (1075,930) .. controls (1042,963) and (1013,990) .. (1011,990) .. controls (1008,990) and (1032,963) .. (1065,930) -- cycle;
			\path[fill] (914,918) -- (895,895) -- (918,914) .. controls (939,932) and (945,940) .. (937,940) .. controls (935,940) and (925,930) .. (914,918) -- cycle;
		\end{scope}
		\begin{scope}[fill = c0503fc]
			\path[fill] (890,1180) -- (890,1100) -- (994,1100) .. controls (1114,1100) and (1110,1097) .. (1110,1196) -- (1110,1260) -- (1000,1260) -- (890,1260) -- (890,1180) -- cycle;
			\path[fill] (890,560) -- (890,480) -- (994,480) .. controls (1114,480) and (1110,477) .. (1110,576) -- (1110,640) -- (1000,640) -- (890,640) -- (890,560) -- cycle;
		\end{scope}
		\begin{scope}[fill = cfa0305]
			\path[fill] (932,933) -- (869,868) -- (934,802) -- (1000,735) -- (1064,800) -- (1129,865) -- (1064,933) .. controls (1027,970) and (997,999) .. (996,998) .. controls (996,998) and (967,968) .. (932,933) -- cycle;
		\end{scope}
	
	\end{tikzpicture}
}

% Colors needed for the logo (sketchy) of the Department of Computer Science of the University of Kaiserslautern (black when in grayscale mode)
\iftoggle{bwmode}{
	\definecolor{c808080}{RGB}{191,191,191}
	\definecolor{cff0000}{RGB}{127,127,127}
	\definecolor{c9999ff}{RGB}{0,0,0}
	\definecolor{c0000ff}{RGB}{63,63,63}
}{
	\definecolor{c808080}{RGB}{128,128,128}
	\definecolor{cff0000}{RGB}{255,0,0}
	\definecolor{c9999ff}{RGB}{153,153,255}
	\definecolor{c0000ff}{RGB}{0,0,255}
}

% The logo of the Department of Computer Science of the University of Kaiserslautern:
% Taken from: http://sci.informatik.uni-kl.de/rechnerzugang/terminals/lageplan_sci/Lageplan_SCI.pdf on the 2017-03-16
% Manipulated using: Inkscape (https://inkscape.org/)
% Converted to TikZ using: svg2tikz (https://github.com/kjellmf/svg2tikz) as an Inkscape extension
\newcommand{\CSLogo}{
	\begin{tikzpicture}[
		y = 1.65pt,
		x = 1.65pt,
		yscale = -1,
		xscale = 1
	]
		\begin{scope}[cm={{0.0, 1.25, 1.25, 0.0, (-153.75, -108.75)}}]
			\path[cm = {{0.0, 0.82808, 1.0, 0.0, (125.0216, 604.3906)}}, fill = c808080, nonzero rule] (0.0000, 0.0000) node[above right] (text2307) {};
			\path[cm = {{0.0, 0.82379, 1.0, 0.0, (125.0216, 609.1211)}}, fill = c808080, nonzero rule] (0.0000, 0.0000) node[above right] (text2311) {};
			\path[cm = {{0.0, 0.82808, 1.0, 0.0, (145.8526, 604.3906)}}, fill = c808080, nonzero rule] (0.0000, 0.0000) node[above right] (text2315) {};
			\path[cm = {{0.0, 0.82379, 1.0, 0.0, (145.8526, 609.1211)}}, fill = c808080, nonzero rule] (0.0000, 0.0000) node[above right] (text2319) {};
			\path[cm = {{0.0, 0.75, 1.0, 0.0, (168.3434, 604.3906)}}, fill = c808080, nonzero rule] (0.0000, 0.0000) node[above right] (text2323) {};
			\path[cm = {{0.0, 1.0, 0.97692, 0.0, (125.1046, 587.6261)}}, fill = c808080, nonzero rule] (0.0000, 0.0000) node[above right] (text2327) {};
			\path[cm = {{0.0, 1.0, 0.95143, 0.0, (147.0145, 587.3771)}}, fill = c808080, nonzero rule] (0.0000, 0.0000) node[above right] (text2331) {};
			\path[cm = {{0.0, 1.0, 0.62474, 0.0, (169.6713, 591.1118)}}, fill = c808080, nonzero rule] (0.0000, 0.0000) node[above right] (text2355) {};
			\path[cm = {{0.0, 1.0, 0.97692, 0.0, (123.8597, 585.9662)}}, fill = cff0000, nonzero rule] (0.0000, 0.0000) node[above right] (text2379) {};
			\path[cm = {{0.0, 1.0, 0.95143, 0.0, (145.8526, 585.7173)}}, fill = cff0000, nonzero rule] (0.0000, 0.0000) node[above right] (text2383) {};
			\path[cm = {{0.0, 1.0, 0.62474, 0.0, (168.5094, 589.452)}}, fill = cff0000, nonzero rule] (0.0000, 0.0000) node[above right] (text2407) {};
			\path[fill = c808080, even odd rule] (123.5280, 563.8070) -- (123.5280, 558.9940) -- (122.4490, 558.9940) -- (122.4490, 568.6210) -- (123.5280, 568.6210) -- (123.5280, 563.8070);
			\path[fill = c808080, even odd rule] (144.9400, 563.8070) -- (144.9400, 558.9940) -- (143.7780, 558.9940) -- (143.7780, 568.6210) -- (144.9400, 568.6210) -- (144.9400, 563.8070);
			\path[fill = c808080, even odd rule] (133.7360, 559.0770) -- (122.4490, 559.0770) -- (122.4490, 560.2390) -- (145.0230, 560.2390) -- (145.0230, 559.0770) -- (133.7360, 559.0770);
			\path[fill = c808080, even odd rule] (143.1970, 568.6210) -- (119.7100, 568.6210) -- (119.7100, 570.3640) -- (166.6010, 570.3640) -- (166.6010, 568.6210) -- (143.1970, 568.6210);
			\path[fill = c808080, even odd rule] (112.4070, 529.1170) -- (104.6890, 554.7610) -- (112.4070, 580.5720) -- (119.7100, 580.5720) -- (119.7100, 529.2830) -- (112.4070, 529.2830) -- (112.5730, 529.1170) -- (112.4070, 529.1170);
			\path[fill = c808080, even odd rule] (122.1170, 535.7560) -- (122.1170, 545.3830) -- (166.4350, 532.4360) -- (166.4350, 523.2240) -- (122.1170, 535.7560);
			\path[fill = c808080, even odd rule] (147.2630, 566.2970) -- (144.2760, 563.2260) -- (138.1340, 569.4510) -- (144.1930, 575.5920) -- (150.3340, 569.3680) -- (147.2630, 566.2970);
			\path[fill = c808080, even odd rule] (133.8190, 569.4510) -- (133.8190, 564.3880) -- (126.4320, 564.3880) -- (126.4320, 574.4300) -- (133.8190, 574.4300) -- (133.8190, 569.4510);
			\path[fill = c808080, even odd rule] (162.6170, 569.4510) -- (162.6170, 564.3880) -- (155.1480, 564.3880) -- (155.1480, 574.4300) -- (162.6170, 574.4300) -- (162.6170, 569.4510);
			\path[fill = c9999ff, even odd rule] (122.6150, 562.8110) -- (122.6150, 557.9150) -- (121.5360, 557.9150) -- (121.5360, 567.6250) -- (122.6150, 567.6250) -- (122.6150, 562.8110);
			\path[fill = c9999ff, even odd rule] (144.0270, 562.8110) -- (144.0270, 557.9150) -- (142.8650, 557.9150) -- (142.8650, 567.6250) -- (144.0270, 567.6250) -- (144.0270, 562.8110);
			\path[fill = c9999ff, even odd rule] (132.8230, 557.9980) -- (121.5360, 557.9980) -- (121.5360, 559.1600) -- (144.1100, 559.1600) -- (144.1100, 557.9980) -- (132.8230, 557.9980);
			\path[fill = c9999ff, even odd rule] (142.2010, 567.6250) -- (118.7140, 567.6250) -- (118.7140, 569.2850) -- (165.6880, 569.2850) -- (165.6880, 567.6250) -- (142.2010, 567.6250);
			\path[fill = c9999ff, even odd rule] (111.4940, 528.0380) -- (103.6930, 553.6820) -- (111.4940, 579.4930) -- (118.7140, 579.4930) -- (118.7140, 528.2040) -- (111.4940, 528.2040) -- (111.6600, 528.0380) -- (111.4940, 528.0380);
			\path[fill = c9999ff, even odd rule] (121.1210, 534.6770) -- (121.1210, 544.3040) -- (165.5220, 531.3570) -- (165.5220, 522.1450) -- (121.1210, 534.6770);
			\path[fill = cff0000, even odd rule] (146.3510, 565.2180) -- (143.2800, 562.1480) -- (137.1380, 568.3720) -- (143.2800, 574.5130) -- (149.4210, 568.2890) -- (146.3510, 565.2180);
			\path[fill = c0000ff, even odd rule] (132.9060, 568.3720) -- (132.9060, 563.3090) -- (125.4370, 563.3090) -- (125.4370, 573.4340) -- (132.9060, 573.4340) -- (132.9060, 568.3720);
			\path[fill = c0000ff, even odd rule] (161.7040, 568.3720) -- (161.7040, 563.3090) -- (154.2350, 563.3090) -- (154.2350, 573.4340) -- (161.7040, 573.4340) -- (161.7040, 568.3720);
			\path[cm = {{0.0, 0.82808, 1.0, 0.0, (124.0257, 603.3947)}}, fill = c9999ff, nonzero rule] (0.0000, 0.0000) node[above right] (text2467) {};
			\path[cm = {{0.0, 0.82379, 1.0, 0.0, (124.0257, 608.1252)}}, fill = c9999ff, nonzero rule] (0.0000, 0.0000) node[above right] (text2471) {};
			\path[cm = {{0.0, 0.82808, 1.0, 0.0, (144.8567, 603.3947)}}, fill = c9999ff, nonzero rule] (0.0000, 0.0000) node[above right] (text2475) {};
			\path[cm = {{0.0, 0.82379, 1.0, 0.0, (144.8567, 608.1252)}}, fill = c9999ff, nonzero rule] (0.0000, 0.0000) node[above right] (text2479) {};
			\path[cm = {{0.0, 0.75, 1.0, 0.0, (167.4305, 603.3947)}}, fill = c0000ff, nonzero rule] (0.0000, 0.0000) node[above right] (text2483) {};
		\end{scope}
		
	\end{tikzpicture}
}


%%%%%%%%%%%%%%%%%%%%%%%%%%%%%%%%%%%%%%%%%%%%%%%
%%%%%%%%%%%%%%%%%%%%%%%%%%%%%%%%%%%%%%%%%%%%%%%
%%%%%%%%%%%%%%%%%%%%%%%%%%%%%%%%%%%%%%%%%%%%%%%
% Defines a shape 'square' for tikz that behaves like a square :-):
%%%%%%%%%%%%%%%%%%%%%%%%%%%%%%%%%%%%%%%%%%%%%%% begindefinition
\makeatletter
% the contents of \squarecorner were mostly stolen from pgfmoduleshapes.code.tex
\def\squarecorner#1{
    % Calculate x
    %
    % First, is width < minimum width?
    \pgf@x=\the\wd\pgfnodeparttextbox%
    \pgfmathsetlength\pgf@xc{\pgfkeysvalueof{/pgf/inner xsep}}%
    \advance\pgf@x by 2\pgf@xc%
    \pgfmathsetlength\pgf@xb{\pgfkeysvalueof{/pgf/minimum width}}%
    \ifdim\pgf@x<\pgf@xb%
        % yes, too small. Enlarge...
        \pgf@x=\pgf@xb%
    \fi%
    % Calculate y
    %
    % First, is height+depth < minimum height?
    \pgf@y=\ht\pgfnodeparttextbox%
    \advance\pgf@y by\dp\pgfnodeparttextbox%
    \pgfmathsetlength\pgf@yc{\pgfkeysvalueof{/pgf/inner ysep}}%
    \advance\pgf@y by 2\pgf@yc%
    \pgfmathsetlength\pgf@yb{\pgfkeysvalueof{/pgf/minimum height}}%
    \ifdim\pgf@y<\pgf@yb%
        % yes, too small. Enlarge...
        \pgf@y=\pgf@yb%
    \fi%
    %
    % this \ifdim is the actual part that makes the node dimensions square.
    \ifdim\pgf@x<\pgf@y%
        \pgf@x=\pgf@y%
    \else
        \pgf@y=\pgf@x%
    \fi
    %
    % Now, calculate right border: .5\wd\pgfnodeparttextbox + .5 \pgf@x + #1outer sep
    \pgf@x=#1.5\pgf@x%
    \advance\pgf@x by.5\wd\pgfnodeparttextbox%
    \pgfmathsetlength\pgf@xa{\pgfkeysvalueof{/pgf/outer xsep}}%
    \advance\pgf@x by#1\pgf@xa%
    % Now, calculate upper border: .5\ht-.5\dp + .5 \pgf@y + #1outer sep
    \pgf@y=#1.5\pgf@y%
    \advance\pgf@y by-.5\dp\pgfnodeparttextbox%
    \advance\pgf@y by.5\ht\pgfnodeparttextbox%
    \pgfmathsetlength\pgf@ya{\pgfkeysvalueof{/pgf/outer ysep}}%
    \advance\pgf@y by#1\pgf@ya%
}
\makeatother

\pgfdeclareshape{simplesquare}{
    \savedanchor\northeast{\squarecorner{}}
    \savedanchor\southwest{\squarecorner{-}}

    \foreach \x in {east,west} \foreach \y in {north,mid,base,south} {
        \inheritanchor[from=rectangle]{\y\space\x}
    }
    \foreach \x in {east,west,north,mid,base,south,center,text} {
        \inheritanchor[from=rectangle]{\x}
    }
    \inheritanchorborder[from=rectangle]
    \inheritbackgroundpath[from=rectangle]
}
%%%%%%%%%%%%%%%%%%%%%%%%%%%%%%%%%%%%%%%%%%%%%%% enddefinition

%%%%%%%%%%%%%%%%%%%%%%%%%%%%%%%%%%%%%%%%%%%%%%%
%%%%%%%%%%%%%%%%%%%%%%%%%%%%%%%%%%%%%%%%%%%%%%%
%%%%%%%%%%%%%%%%%%%%%%%%%%%%%%%%%%%%%%%%%%%%%%%
% Adds 'west north west', 'east north east', 'east south east', 'north north west', 
% 'south south west', 'south south east' to the tikz shape 'rectangle':
%%%%%%%%%%%%%%%%%%%%%%%%%%%%%%%%%%%%%%%%%%%%%%% begindefinition
\makeatletter
\pgfdeclareshape{square}{
  \inheritsavedanchors[from=simplesquare]
  \inheritanchorborder[from=simplesquare]
  \foreach \a in {%
      center,mid,base,north,south,west,east,%
      north west,mid west,base west,south west,%
      north east,mid east,base east,south east%
    }{\inheritanchor[from=simplesquare]{\a}}
  \inheritbackgroundpath[from=simplesquare]
  \anchor{north 1/3}{
    \southwest\pgf@xa=\pgf@x
    \northeast\pgfmathsetlength\pgf@x{\pgf@xa-(\pgf@xa-\pgf@x)/3}
  }
  \anchor{north 2/3}{
    \southwest\pgf@xa=\pgf@x
    \northeast\pgfmathsetlength\pgf@x{\pgf@x-(\pgf@x-\pgf@xa)/3}
  }
  \anchor{south 1/3}{
    \northeast\pgf@xa=\pgf@x
    \southwest\pgfmathsetlength\pgf@x{\pgf@x-(\pgf@x-\pgf@xa)/3}
  }
  \anchor{south 2/3}{
    \northeast\pgf@xa=\pgf@x
    \southwest\pgfmathsetlength\pgf@x{\pgf@xa-(\pgf@xa-\pgf@x)/3}
  }
  \anchor{east 1/3}{
    \southwest\pgf@ya=\pgf@y
    \northeast\pgfmathsetlength\pgf@y{\pgf@ya-(\pgf@ya-\pgf@y)/3}
  }
  \anchor{east 2/3}{
    \southwest\pgf@ya=\pgf@y
    \northeast\pgfmathsetlength\pgf@y{\pgf@y-(\pgf@y-\pgf@ya)/3}
  }
  \anchor{west 1/3}{
    \northeast\pgf@ya=\pgf@y
    \southwest\pgfmathsetlength\pgf@y{\pgf@y-(\pgf@y-\pgf@ya)/3}
  }
  \anchor{west 2/3}{
    \northeast\pgf@ya=\pgf@y
    \southwest\pgfmathsetlength\pgf@y{\pgf@ya-(\pgf@ya-\pgf@y)/3}
  }
}
\makeatother
%%%%%%%%%%%%%%%%%%%%%%%%%%%%%%%%%%%%%%%%%%%%%%% enddefinition

%%%%%%%%%%%%%%%%%%%%%%%%%%%%%%%%%%%%%%%%%%%%%%%
%%%%%%%%%%%%%%%%%%%%%%%%%%%%%%%%%%%%%%%%%%%%%%%
%%%%%%%%%%%%%%%%%%%%%%%%%%%%%%%%%%%%%%%%%%%%%%%
% Redefines the bibliography entry for inproceedings to allow ISSN and/or ISBN:
%%%%%%%%%%%%%%%%%%%%%%%%%%%%%%%%%%%%%%%%%%%%%%% begindefinition
\DeclareBibliographyDriver{inproceedings}{%
  \usebibmacro{bibindex}%
  \usebibmacro{begentry}%
  \usebibmacro{author/translator+others}%
  \setunit{\printdelim{nametitledelim}}\newblock
  \usebibmacro{title}%
  \newunit
  \printlist{language}%
  \newunit\newblock
  \usebibmacro{byauthor}%
  \newunit\newblock
  \usebibmacro{in:}%
  \usebibmacro{maintitle+booktitle}%
  \newunit\newblock
  \usebibmacro{event+venue+date}%
  \newunit\newblock
  \usebibmacro{byeditor+others}%
  \newunit\newblock
  \iffieldundef{maintitle}
    {\printfield{volume}%
     \printfield{part}}
    {}%
  \newunit
  \printfield{volumes}%
  \newunit\newblock
  \usebibmacro{series+number}%
  \newunit\newblock
  \printfield{note}%
  \newunit\newblock
  \printlist{organization}%
  \newunit
  \usebibmacro{publisher+location+date}%
  \newunit\newblock
  \usebibmacro{chapter+pages}%
  \newunit\newblock
  \iftoggle{bbx:isbn}
    {\printfield{isbn}%
     \newunit\newblock
     \printfield{issn}}
    {}%
  \newunit\newblock
  \usebibmacro{doi+eprint+url}%
  \newunit\newblock
  \usebibmacro{addendum+pubstate}%
  \setunit{\bibpagerefpunct}\newblock
  \usebibmacro{pageref}%
  \newunit\newblock
  \iftoggle{bbx:related}
    {\usebibmacro{related:init}%
     \usebibmacro{related}}
    {}%
  \usebibmacro{finentry}}
%%%%%%%%%%%%%%%%%%%%%%%%%%%%%%%%%%%%%%%%%%%%%%% enddefinition

%%%%%%%%%%%%%%%%%%%%%%%%%%%%%%%%%%%%%%%%%%%%%%%
%%%%%%%%%%%%%%%%%%%%%%%%%%%%%%%%%%%%%%%%%%%%%%%
%%%%%%%%%%%%%%%%%%%%%%%%%%%%%%%%%%%%%%%%%%%%%%%
% Redefines the bibliography field definitions for url and doi to reduce their font size:
%%%%%%%%%%%%%%%%%%%%%%%%%%%%%%%%%%%%%%%%%%%%%%% begindefinition
\DeclareFieldFormat{url}{\mkbibacro{URL}\addcolon\space\footnotesize\url{#1}}
\DeclareFieldFormat{doi}{%
  \mkbibacro{DOI}\addcolon\space\footnotesize
  \ifhyperref
    {\href{https://doi.org/#1}{\nolinkurl{#1}}}
    {\nolinkurl{#1}}}
%%%%%%%%%%%%%%%%%%%%%%%%%%%%%%%%%%%%%%%%%%%%%%% enddefinition

%%%%%%%%%%%%%%%%%%%%%%%%%%%%%%%%%%%%%%%%%%%%%%%
%%%%%%%%%%%%%%%%%%%%%%%%%%%%%%%%%%%%%%%%%%%%%%%
%%%%%%%%%%%%%%%%%%%%%%%%%%%%%%%%%%%%%%%%%%%%%%%
% Allows the positioning of nodes on a circle:
%%%%%%%%%%%%%%%%%%%%%%%%%%%%%%%%%%%%%%%%%%%%%%% begindefinition
\usepackage{tikz}
\usetikzlibrary{chains}
\tikzset{
  nodes around center/.style args={#1:#2:#3:#4}{%
    % #1 = Startwinkel,   #2 = Anzahl Knoten
    % #3 = Zentrums-Node, #4 = Abstand
    at={([shift={(#3)}] {{(\tikzchaincount-1)*360/(#2)+#1}}:{#4})}
  },
  nodes around center*/.style args={#1:#2:#3:#4}{% gleiche Optionen wie oben
    at={([shift={(#3.{(\tikzchaincount-1)*360/(#2)+#1})}] {{(\tikzchaincount-1)*360/(#2)+#1}}:{#4})},
    anchor={(\tikzchaincount-1)*360/(#2)+#1+180}
  }
}
%%%%%%%%%%%%%%%%%%%%%%%%%%%%%%%%%%%%%%%%%%%%%%% enddefinition


\KOMAoptions{DIV = last}				% recalculation of the type area with regard to all the settings

\begin{document}

	\newcommand{\thickrule}{\rule{\linewidth}{1mm}}
\newcommand{\thinrule}{\rule{\linewidth}{.25mm}}

\newcommand{\university}{University of Kaiserslautern}        % the name of the university
\newcommand{\universitylogo}{\scalebox{2}{\TULogoWithText}}   % the logo of the university (with name)
\newcommand{\faculty}{Department of Computer Science}         % the faculty
\newcommand{\chair}{Database and Information Systems Group}   % the chair
\newcommand{\student}{Max Fabian Gilbert}                     % the student's name
\newcommand{\email}{m\_gilbert13@cs.uni-kl.de}                % the student's email address
\newcommand{\doctype}{Project Thesis}                         % the type of the thesis (e.g. Bachelor Thesis)
\newcommand{\doctitle}{Performance Evaluation of Different Open Source Implementations of Data Structures and Other Algorithms in the context of a DBMS Buffer Manager}           % the title of the work
\newcommand{\releasedate}{\specificdate{2020}{1}{15}}         % the release date of the thesis

\titlehead{
%   \vspace{-1.5cm}%
    \begin{center}
        \universitylogo \\
        \vspace{1cm}%
        \LARGE \faculty \\
        \Large \chair%
    \end{center}
}

\subject{
    \textsf{\doctype:}
    \vspace{-1.5cm}
}

\title{
    \thinrule
    \vspace{.5cm}
    \begin{ptsans}
        \vbox{
        \doctitle
        }
    \end{ptsans}
    \vspace{.5cm}
    \thinrule
}

\author{
    \vspace{-2cm}
    \hfill \\
    \protect{
        by
        \textbf{\student \thanks{\email}}
    }
}

\date{
    \setlength{\tabcolsep}{0pt}
    \begin{flushleft}
        \begin{tabular}{ll}
            \textbf{Day of release: }     & \releasedate
        \end{tabular}
    \end{flushleft}
    \setlength{\tabcolsep}{6pt}
}

\begin{ptsans}
    \renewcommand*{\thefootnote}{\fnsymbol{footnote}}        % fixes the usage of arabic numbers instead of symbols for footnotes in the further preliminaries when using extratitle
    \thispagestyle{empty}
    \KOMAoptions{DIV = 12}
    \maketitle
    \KOMAoptions{DIV = last}
\end{ptsans}

			
	% declaration of authorship
	
	\pagenumbering{Roman}

	\newenvironment{myabstract}
    {
        \KOMAoptions{DIV = last}
        \begin{abstract}
            \thispagestyle{plain}
            \small
    }{
        \end{abstract}
    }

\begin{myabstract}
    Needless to say, every database management system needs to be able to manage data. The data structures used to manage those data in a database have a major influence on various characteristics (e.g. performance) of a database management system and therefore, the usage of specific data structures (e.g. B-tree indexes) and even some implementation details of those are very important decisions in DBMS design.

    But for correct and performant operation, a DBMS needs to manage various kinds of meta data as well. Some of those meta data needs to be persistent (e.g. the catalog of a relational DBMS) but some can also be non-persistent. Because of the non-persistence of data managed by the buffer management of a DB, the meta data required for the buffer manager are also usually non-persistent. The data structures used to manage those meta data are---unlike the data structures used to manage the data---more an implementation than a design decision. For some kinds of those meta data, it's---due to the non-criticality of the specific meta data management---even reasonable to use data structures provided by the used programming language even though there might be more performant data structures for the purpose. But more performant implementations for most of those data structures don't need to be implemented specifically for one project, there are many different implementations available in open source and proprietary libraries.

    This work is a performance evaluation of various MPMC %TODO: Complete abstract
\end{myabstract}
	
	\cleardoublepage
	\tableofcontents 			% generates a table of contents
	% \listoffigures 				% generates a list of all figures (use of the figure environment (or environment using the figure environment) with a \label)
	% \listoftables 				% generates a list of all tables (use of the ? environment with a ?)
	% \listof{code}{List of Listings}	% generates a list of all listings as defined using the float package
	
	\cleardoublepage
	\pagenumbering{arabic}
	
	\chapter[Buffer Frame Free List]{Buffer Frame Free List} \label{ch:free-list}

\section[Purpose]{Purpose}

	A buffer manager is required for every disk-based DBMS. A disk-based DBMS stores the pages of a database on secondary storage but to read and write pages, they are required to be in memory.
	
	This feature is provided by the buffer pool management by managing the currently used subset of the database pages in buffer frames located in memory. A buffer frame is a portion of memory that can hold one database page and each of those frames got a frame index as identifier.
	
	During operation, database pages are dynamically fetched from the database into buffer frames. Once a page is not required anymore, it might be evicted from the buffer pool freeing a buffer frame.
	
	Due to the fact that pages are only allowed to be fetched into free buffer frames, the buffer manager needs to know all the free buffer frames. Therefore, a free list for the buffer frames---storing the frame indexes of free buffer frames---is required.

\section[Compared Queue Implementations]{Compared Queue Implementations}

	To ease implementation of page eviction strategies like CLOCK, a free list should use a FIFO data structure like a queue. Therefore the buffer frame freed first is (re-)used first as well.
	
	Almost every state-of-the-art DBMS support multithreading and therefore, there are usually multiple threads concurrently fetching pages into the buffer pool and evicting pages from the buffer pool. Following this, a buffer frame free list has to support thread-safe functions to push frame indexes to the free list and to pop frame indexes from it. Queues providing those thread-safe access functions are usually called multi-producer (add frame indexes) multi-consumer (retrieve/remove frame indexes) queues (\textbf{MPMC} queues).
	
	An approximate number of buffer indexes in the free list must also be provided by any free list implementation to support the eviction of pages once there are only a few free buffer frames left. Thread-safe access to this number is desirable but not absolutely required.

\subsection[Boost Lock-Free Queue with variable size]{Boost Lock-Free Queue with variable size} \label{subsec:boost}

	The famous \textit{Boost C++ Libraries}\footnote{\url{https://www.boost.org/}} offer a lock-free unbounded MPMC queue\footnote{\url{https://www.boost.org/doc/libs/release/doc/html/boost/lockfree/queue.html}} in the library \lstinline{Boost.Lockfree}\footnote{\url{https://www.boost.org/doc/libs/release/doc/html/lockfree.html}}. Like many other non-blocking thread-safe data structures, this MPMC queue uses atomic operations instead of locks or mutexes. To support queues of dynamically changing sizes, this queue implementation also uses a free list for the dynamic memory management internally.
	
	This data structure does not offer the number of contained elements and therefore, an approximate number of buffer indexes in the free list needs to be managed outside.

\subsection[Boost Lock-Free Queue with fixed size]{Boost Lock-Free Queue with fixed size} \label{subsec:boost-fixed}

	This data structure is identical to the data structure in Subsection \ref{subsec:boost} but does not use dynamic memory management internally---it is a bounded queue. Therefore, the capacity of the queue (i.e. the maximum number of buffer frames of the buffer pool) needs to be specified beforehand which allows the usage of a fixed-size array instead of dynamically allocated nodes.

\subsection[CDS BasketQueue]{CDS Basket Lock-Free Queue} \label{subsec:cds-basket}

	Besides other concurrent data structures, the \textit{Concurrent Data Structures} C++ library\footnote{\url{https://github.com/khizmax/libcds}} offers many different thread-safe queue implementations. The unbounded \emph{Basket Lock-Free Queue}\footnote{\url{http://libcds.sourceforge.net/doc/cds-api/classcds\_1\_1container\_1\_1\_basket\_queue.html}} is based on the algorithm proposed by M. Hoffman, O. Shalev and N. Shavit in \cite{Hoffman:2007}.
	
	Internally, this queue does not use an absolute FIFO order. Instead, it puts concurrently enqueued elements into one ``basket'' of elements. The elements within one basket are not specifically ordered but the different ``baskets'' used over time are ordered according to FIFO. Therefore, the dequeue operation just dequeues one of the elements in the oldest ``basket''. The dynamic memory management uses a garbage collector to deallocate emptied ``baskets''.

\subsection[CDS FCQueue]{CDS Flat-Combining Lock-Free Queue} \label{subsec:cds-fc}

	The \textit{Concurrent Data Structures} C++ library does also offer an unbounded thread-safe queue that uses Flat Combining\footnote{\url{http://libcds.sourceforge.net/doc/cds-api/classcds\_1\_1container\_1\_1\_f\_c\_queue.html}}. The Flat Combining technique was proposed by D. Hendler, I. Incze, N. Shavit and M. Tzafrir in \cite{Hendler:2010}.  This technique is used to make any sequential data structure thread-safe---in case of the \emph{Flat-Combining Lock-Free Queue}, the \lstinline{std::queue}\footnote{\url{https://en.cppreference.com/w/cpp/container/queue}} of the \textit{C++ Standard Library}\footnote{\url{https://en.cppreference.com/w/cpp}} is used as base data structure.
	
	The Flat Combining technique uses thread-local publication lists to record operations performed by those threads. A global lock is needed to be acquired to combine these thread-local publication lists into the global, sequential data structure. The thread which acquired the global lock also combines the publication lists of all other threads reducing the locking overhead. The returned value of each operation executed during the combining is stored into the respective publication list together with the global combining pass number. A thread with a non-empty publication list that cannot acquire the global lock needs to wait till the combining thread updated its publication list.

\subsection[CDS MSQueue]{CDS Michael \& Scott Lock-Free Queue} \label{subsec:cds-ms}

	Another unbounded lock-free queue implementation offered by the \textit{Concurrent Data Structures} C++ library is based on the famous Michael \& Scott lock-free queue algorithm\footnote{\url{http://libcds.sourceforge.net/doc/cds-api/classcds\_1\_1container\_1\_1\_m\_s\_queue.html}} which was proposed by M. Michael and M. Scott in \cite{Michael:1996}.
	
	The Michael \& Scott lock-free queue basically uses compare-and-swap (\textbf{CAS}) operations on the tail of the queue to synchronize enqueue operations. If a thread reads a NULL value as next element after the queue's tail, it swaps this value atomically with the value enqueued by this thread. Afterwards it adjusts the tail pointer. If a thread does not read the NULL value there during the CAS operation, another thread has not already adjusted the tail pointer and this thread needs to retry its enqueue operation with the new tail pointer. The dequeue operation is implemented similarly. The memory occupied by already dequeued elements is deallocated using a garbage collector provided by the library.

\subsection[CDS MoirQueue]{CDS Variation of Michael \& Scott Lock-Free Queue} \label{subsec:cds-moir}

	The \textit{Concurrent Data Structures} C++ library also offers an optimized variation of the Michael \& Scott unbounded lock-free queue algorithm\footnote{\url{http://libcds.sourceforge.net/doc/cds-api/classcds\_1\_1container\_1\_1\_moir\_queue.html}} which is based on the works of S. Doherty, L. Groves, V. Luchangco and M. Moir in \cite{Doherty:2004}.
	
	This optimization of the Michael \& Scott lock-free queue optimizes the dequeue operation to only read the tail pointer once.

\subsection[CDS RWQueue]{CDS Michael \& Scott Blocking Queue with Fine-Grained Locking} \label{subsec:cds-rw}

	M. Michael and M. Scott did also propose a blocking queue algorithm in \cite{Michael:1996}. This unbounded blocking queue implementation\footnote{\url{http://libcds.sourceforge.net/doc/cds-api/classcds\_1\_1container\_1\_1\_r\_w\_queue.html}} is also offered by the \textit{Concurrent Data Structures} C++ library.
	
	This blocking queue algorithm uses one read and one write lock protecting the head and tail of the queue. Therefore, only one thread a time can enqueue and only one thread at a time can dequeue elements. The deallocation of memory during dequeuing is done by the dequeuing thread instead of relying on a garbage collector.

\subsection[CDS OptimisticQueue]{CDS Ladan-Mozes \& Shavit Optimistic Queue} \label{subsec:cds-optimistic}

	The \textit{Concurrent Data Structures} C++ library also offers an unbounded optimistic queue implementation\footnote{\url{http://libcds.sourceforge.net/doc/cds-api/classcds\_1\_1container\_1\_1\_optimistic\_queue.html}} which is based on an algorithm proposed by E. Ladan-Mozes and N. Shavit in \cite{Ladan-Mozes:2004}.
	
	Instead of using expensive CAS operations on a singly-linked list (like in the Michael \& Scott lock-free queue), this algorithm uses a doubly-linked list with the possibility to detect and fix inconsistent enqueue and dequeue operations. Deallocation of memory is done using a garbage collector.

\subsection[CDS SegmentedQueue]{CDS Segmented Queue} \label{subsec:cds-segmented}

	The unbounded segmented queue implementation\footnote{\url{http://libcds.sourceforge.net/doc/cds-api/classcds\_1\_1container\_1\_1\_segmented\_queue.html}} of the \textit{Concurrent Data Structures} C++ library is based on an algorithm proposed by Y. Afek, G. Korland and E. Yanovsky in \cite{Afek:2010}.
	
	This thread-safe queue algorithm is very similar to the basket lock-free queue. It also uses a relaxed FIFO order by ordering segments containing multiple elements instead of single elements. A thread enqueuing or dequeuing elements into the tail segment or from the head segment selects one of the slots inside the segment randomly. CAS operations are used to atomically enqueue or dequeue an element from a slot. If the CAS fails, another slot is taken randomly. The size of each segment---which can be selected ($8$ was used for the performance evaluation in Section \ref{sec:free-list-performance})---determines the relaxness of the FIFO order. Deallocation of emptied segments is done by a garbage collector.

\subsection[CDS VyukovMPMCCycleQueue]{CDS Vyukov's MPMC Bounded Queue} \label{subsec:cds-vyukovmpmccycle}

	The last thread-safe queue implementation\footnote{\url{http://libcds.sourceforge.net/doc/cds-api/classcds\_1\_1container\_1\_1\_vyukov\_m\_p\_m\_c\_cycle\_queue.html}} provided by the \textit{Concurrent Data Structures} C++ library is bounded and was developed by D. Vyukov\footnote{\url{http://www.1024cores.net/home/lock-free-algorithms/queues/bounded-mpmc-queue}}. The queue implementation in Subsection \ref{subsec:vyukov} is his original implementation.
	
	Vyukov's thread-safe queue implementation is very similar to Michael \& Scott blocking queue with fine-grained locking from Subsection \ref{subsec:cds-rw} but instead of using mutexes as locks, his implementation uses atomic read-modify-write (\textbf{RMW}) operations. This results in a cost of basically one CAS operation per enqueue/dequeue operation.

\subsection[Folly MPMC Queue]{Folly MPMC Queue} \label{subsec:folly-mpmc}

	Facebook's open source library \textit{Folly}\footnote{\url{https://github.com/facebook/folly}} provides a bounded lock-free queue implementation. An unbounded queue is also provided but due to the typically lower performance of unbounded queues, it is not evaluated in Section \ref{sec:free-list-performance}.
	
	\textit{Folly}'s MPMC queue uses a ticket dispenser system to give a thread access to one of the single-element queues used. Those ticket dispensers for the head and tail of the queue use atomic increment operations which are supposed to be more robust to contention than CAS operations used e.g. in the Michael \& Scott lock-free queue.

\subsection[Dmitry Vyukov's MPMC Queue]{Dmitry Vyukov's Bounded MPMC Queue} \label{subsec:vyukov}

	This\footnote{\url{http://www.1024cores.net/home/lock-free-algorithms/queues/bounded-mpmc-queue}} is Vyukov's original implementation of his bounded thread-safe MPMC queue.

\subsection[Gavin Lambert's MPMC Queue]{Gavin Lambert's MPMC Bounded Lock-Free Queue} \label{subsec:lampert}

	This\footnote{\url{https://gist.github.com/uecasm/b547db812ae4bba39bb1bd0443801507}} is another implementation of Vyukov's thread-safe queue made by [Gavin Lambert.

\subsection[\lstinline{moodycamel::ConcurrentQueue}]{\lstinline{moodycamel::ConcurrentQueue}} \label{subsec:moodycamel}

	This lock-free queue implementation\footnote{\url{https://github.com/cameron314/concurrentqueue}} is either unbounded or bounded depending on the used enqueuing functions and on the optional preallocation of memory (bounded behavior is used during the performance evaluation in Section \ref{sec:free-list-performance}). A blocking queue implementation provided by the same library is not evaluated in Section \ref{sec:free-list-performance} because it is just a wrapper around the non-blocking version adding additional overhead in low-contention workloads (like the free list).
	
	Internally, this queue implementation uses one SPMC (single producer/multiple consumer) queue per thread. Each thread enqueues elements only into its thread-local SPMC queue. When a thread tries to dequeue an element, it checks SPMC queues for emptiness until it finds one containing elements. It then dequeues one element from the SPMC queue. Therefore, this thread-safe queue does not maintain the order of elements enqueued by different threads.
	
	Due to the implementation using multiple SPMC queues, this queue implementation should only be used as a buffer frame free list when there is exactly one thread evicting pages from the buffer pool---and therefore, enqueuing buffer frame indexes of emptied buffer frames.

\subsection[Matt Stump's MPMC Queue]{Matt Stump's Bounded MPMC Queue} \label{subsec:vyukov-variation}

	This\footnote{\url{https://github.com/mstump/queues}} is another implementation of Vyukov's thread-safe queue made by Matt Stump.

\subsection[Erik Rigtorp's MPMC Queue]{Erik Rigtorp's Bounded MPMC Queue} \label{subsec:rigtorp}

	The bounded lock-free queue\footnote{\url{https://github.com/rigtorp/MPMCQueue}} of Erik Rigtorp uses a ticket dispenser system similar to the one of \textit{Folly}'s MPMC queue from Subsection \ref{subsec:folly-mpmc}.

\subsection[TBB Concurrent Queue]{Threading Building Blocks Concurrent Queue} \label{subsec:intel-bounded}%Check license

	The \textit{Threading Building Blocks} library\footnote{\url{https://www.threadingbuildingblocks.org/}} is an open source library originally developed by Intel®. The first thread-safe queue implementation\footnote{\url{https://software.intel.com/en-us/node/506200}} of this library is unbounded and non-blocking.
	
	Internally, this queue implementation uses multiple lock-based micro queues to allow concurrent enqueue/dequeue executions. Therefore, the guarantees of this queue is similar to those of the \lstinline{moodycamel::ConcurrentQueue} from Subsection \ref{subsec:moodycamel}.

\subsection[TBB Bounded Concurrent Queue]{Threading Building Blocks Bounded Concurrent Dual Queue} \label{subsec:intel-unbounded}%Check license

	The other thread-safe queue implementation\footnote{\url{https://software.intel.com/en-us/node/506201}} of the \textit{Threading Building Blocks} library is unbounded and partially non-blocking.
	
	This queue implementation is almost identical to the other one of the Threading Building Blocks library but it does allow the limitation of the capacity. An enqueuing operation has to wait if the queue is already full according to the specified capacity.

\section[Performance Evaluation]{Performance Evaluation} \label{sec:free-list-performance}

\subsection[Micro Benchmark]{Micro Benchmark}

	The used micro benchmark simulates a high contented free list. The number of working threads, the number of iterations (either the fetching of a page into a free buffer frame or the eviction of a batch of pages) per thread and the batch size of buffer frames to be freed at once can be varied. It does not simulate a complete buffer pool---there is only the free list with operations to enqueue and dequeue buffer frame indexes. Each working thread performs the following operations per iterations:
	
\begin{@empty}
	\begin{itemize}
		\itemsep0em
		\item	If the free list is not empty:
			\begin{itemize}
				\item	Retrieve a buffer frame index from the free list.
				\item Mark the retrieved buffer frame used.
			\end{itemize}
		\item	If the free list is empty:
			\begin{itemize}
				\item	While the free list is smaller than the batch eviction size:
					\begin{itemize}
						\item	Select a random buffer frame index using a fast random numbers generator.
						\item If this buffer frame index is marked used:
							\begin{itemize}
								\item	Mark the selected buffer frame index unused.
								\item	Add the selected buffer frame index to the free list.
							\end{itemize}
					\end{itemize}
			\end{itemize}
	\end{itemize}
\end{@empty}

\subsection[Queue Versions]{Used Versions of the Libraries and Queue Implementations}

\begin{@empty}
	\begin{itemize}
		\itemsep0em
		\item	\textit{Boost C++ Libraries} 1.58
		\item	\textit{Concurrent Data Structures} C++ library 2.3.3\footnote{\url{https://github.com/khizmax/libcds/tree/5fc87a172bd82f8a7040b8b83f32ce0e635e82ea}}
		\item	\textit{Folly} \lstinline{a15fcb1e76}\footnote{\url{https://github.com/facebook/folly/tree/a15fcb1e76444f7d464b263ad37bf3b5fbfdf33e}}
		\item	Dmitry Vyukov's Original MPMC Queue as of September 2017
		\item	Gavin Lambert's MPMC Queue as of September 2017\footnote{\url{https://gist.github.com/uecasm/b547db812ae4bba39bb1bd0443801507/e40906811cb14118d328c353250559fe359f3ba7}}
		\item	\lstinline{moodycamel::ConcurrentQueue} \lstinline{9f9c4e0cf4}\footnote{\url{https://github.com/cameron314/concurrentqueue/tree/9f9c4e0cf400bcc5c27a041e524f04e950736b25}}
		\item	Matt Stump's MPMC Queue \lstinline{319c253d68}\footnote{\url{https://github.com/mstump/queues/tree/319c253d68f14ac9593c3727d1597a87af73c99b}}
		\item	Erik Rigtorp's MPMC Queue \lstinline{57366e41f3}\footnote{\url{https://github.com/rigtorp/MPMCQueue/tree/57366e41f3f48316f175c2e704795f519a92e1d5}}
		\item	Intel® Threading Building Blocks 2017 Update 7
	\end{itemize}
\end{@empty}

\subsection[System Configuration]{Configuration of the Used System}

\begin{@empty}
	\begin{itemize}
		\itemsep0em
		\item	\textbf{CPU:} $2 \times $ \emph{Intel® Xeon® Processor X5670} @$6 \times 2.93\text{GHz}$ released early 2010
		\item	\textbf{Main Memory:} $12 \times 8\text{GB} = 96\text{GB}$ of DDR2-SDRAM @$1333\text{MHz}$
		\item	\textbf{OS:} \emph{Ubuntu 16.04}
	\end{itemize}
\end{@empty}

\section{Conclusion}

	
	\chapter[RANDOM Page Eviction]{RANDOM Page Eviction} \label{ch:random}

\section[Purpose]{Purpose}

    The buffer manager of a DBMS must evict pages from buffer frames if currently not buffered pages need to be fetched from the DB while there are no more free buffer frames. For this purpose, each buffer manager got a page eviction module---implementing one of the many page eviction algorithms developed since the 1960s.

    According to Belady's classification in \cite{Belady:1966}, the RANDOM eviction algorithm is the most representative algorithm in his \textit{Class 1} of page eviction algorithms. These \textit{Class 1} page eviction algorithms do not use information about the usage of a buffered page but, but simply apply a static rule for the eviction decision. According to the more recent classification by Effelsberg and Härder in \cite{Effelsberg:1984}, the RANDOM eviction algorithm is the only algorithm in the class of algorithms that does neither use the age of a buffered page nor the references of it for the eviction decision.

    The RANDOM strategy is the simplest possible page eviction strategy, resulting in a low overhead and poor hit rates.

\section[Compared Pseudorandom Number Generators]{Compared Pseudorandom Number Generators}

    The only operation performed by a RANDOM page eviction module to decide which page to evict from the buffer pool is the generation of a pseudorandom number in the range of buffer frame indexes. The DB page contained in the selected buffer frame is then evicted.

    There are many different classes of pseudorandom number generators (\textbf{PRNG}). Some of them provide pseudorandom numbers of high randomness---suitable for cryptographic applications---others require only few CPU cycles and almost no memory to generate a random number.

    Due to the enormous number of PRNGs described in literature, an exhaustive comparison of PRNGs for use in RANDOM page eviction is not possible in this context. Therefore only a selection of PRNGs---mostly from the \textit{C++ Standard Library}\footnote{\url{https://en.cppreference.com/w/cpp}} and the \textit{Boost Random Number Library}\footnote{\url{https://www.boost.org/doc/libs/release/doc/html/boost_random.html}} (part of the \textit{Boost C++ Libraries}\footnote{\url{https://www.boost.org/}})---was selected for this evaluation.

\subsection[Linear Congruential Generator (LCG) -- 1958]{Linear Congruential Generator (LCG) -- 1958} \label{subsec:lcg}

    The \emph{linear congruential generator}---a generalization of the earlier proposed \emph{Lehmer generator}---is a family of PRNGs proposed independently by W. E. Thomson in \cite{Thomson:1958} and by A. Rotenberg in \cite{Rotenberg:1960}.

    A LCG is defined by the following recurrence relation $X$:
    \begin{equation*}
        X_{n + 1} = \left(a \cdot X_n + c\right) \bmod m \quad n \geq 0
    \end{equation*}
    In this definition, $a \in \left(0.. m\right)$ is the multiplier, $c \in \left[0.. m\right)$ is the increment, $m \in \left(0.. \infty\right)$ is the modulus and $X_0 \in \left[0.. m\right)$ is the seed.

    The following members of the LCG family of PRNGs, which do not belong to specializations defined in subsections, were compared:
    \begin{itemize}
		\itemsep0em
        \item \textbf{rand}: $a = 0x41C64E6D$, $c = 0x3039$, $m = 2^{31}$ if using \textit{GNU C Library}\footnote{\url{https://www.gnu.org/software/libc/}}
        \item \textbf{rand48}: $a = 0x5DEECE66D$, $c = 0xB$, $m = 2^{48}$
        \item \textbf{Kreutzer1986}: Buffers 97 random numbers of a LCG with $a = 0x556$, $c = 0x24D69$, $m = 0xAE529$ and returns them shuffled according to an algorithm proposed by Carter Bays and S. D. Durham in \cite{Bays:1976}. This was proposed by Wolfgang Kreutzer in \cite{Kreutzer:1986}.
    \end{itemize}

\subsubsection[Lehmer Generator (MCG) -- 1949]{Lehmer Generator -- 1949} \label{subsubsec:mcg}

    The \emph{Lehmer generator} (also known as \emph{multiplicative congruential generator}) is the earliest family of PRNGs of ``usable'' quality, proposed by Derrick H. Lehmer in \cite{Lehman:1951} in 1949.

    It is a specialization of the later proposed LCG with $c = 0$.

    The following members of the MCG family of PRNGs were compared:
    \begin{itemize}
		\itemsep0em
        \item \textbf{MCG128}:     $a = 0x1168C7BF168D765C661FD0407A968ADD$, $m = 2^{64} - 1$, \SI{128}{\bit} state
        \item \textbf{MCG128Fast}: \textit{MCG128} with $a = 0xDA942042E4DD58B5$
        \item \textbf{RANECU}: Combination of two Lehmer generators ($a_1 = 0x9C4E$, $m_1 = 0x7FFFFFAB$, $a_2 = 0x9EF4$, $m_2 = 0x7FFFFF07$) where the output is $o_1 - o_2$ if $o_2 < o_1$ or $o_1 - o_2 + 0x7FFFFFAA$ (unsigned \SI{32}{\bit} output) else for $o_1$, $o_2$ random numbers generated by the two Lehmer generators. This was proposed by Pierre L'Ecuyer in \cite{LEcuyer:1988} and modified by F. James in \cite{James:1990}.
    \end{itemize}

\subsubsection[Park-Miller Generator -- 1988]{Park-Miller Generator -- 1988} \label{subsubsec:minstd}

    The \emph{Park-Miller generator} (now known as MINSTD) is a set of parameters for the \emph{Lehmer generator} proposed by Stephen K. Mark and Keith W. Miller in \cite{Park:1988}. After the criticism from George Marsaglia and Stephen Sullivan they proposed a modified set of parameters in \cite{Park:1993}.

    In their initial proposal the parameters were $a = 16807$ and $m = 2^{31} - 1$. In their later proposal they used $a = 48271$ instead.

    The following Park-Miller generators were compared:
    \begin{itemize}
		\itemsep0em
        \item \textbf{MINSTD0}: $a = 0x41A7$, $m = 2^{31} - 1$
        \item \textbf{MINSTD}:  $a = 0xBC8F$, $m = 2^{31} - 1$
        \item \textbf{KnuthB}: Buffers 256 random numbers of \textit{MINSTD0} and returns them shuffled according to an algorithm proposed by Carter Bays and S. D. Durham in \cite{Bays:1976}. This was proposed by Donald E. Knuth in \cite{Knuth:1981}.
    \end{itemize}

\subsubsection[MIXMAX Generator -- 1991]{MIXMAX Generator -- 1991} \label{subsubsec:mixmax}

    The \emph{MIXMAX generator} is a \emph{matrix linear congruential generator} proposed by G. K. Savvidy and N. G. Ter-Arutyunyan-Savvidy in \cite{Savvidy:1991}.

    Unlike an LCG, a matrix LCG uses a $N{\times}N$-matrix of multipliers $A$ instead of a multiplier $a$:
    \begin{equation*}
        a'_i = \begin{cases}
                   \left(\sum_{j = 1}^{N} A_{ij} \cdot a_j\right) \bmod m + s \cdot a_2 & \text{if } i = 3 \\
                   \left(\sum_{j = 1}^{N} A_{ij} \cdot a_j\right) \bmod m               & \text{else}
               \end{cases}
    \end{equation*}
    In this definition, $s \in \mathbb{Z}$ is a small ``magic'' integer, $m \in \left(0.. \infty\right)$ is the modulus and the initial $N$-dimensional vector $a$ is the seed.

    The following members of the MIXMAX family of PRNGs was compared:
    \begin{itemize}
		\itemsep0em
        \item \textbf{MixMax2.0}: $N = 17$, $s = 0$, $m = 2^{36} + 1$
    \end{itemize}

\subsubsection[Permuted Congruential Generator (PCG) -- 2014]{Permuted Congruential Generator (PCG) -- 2014} \label{subsubsec:pcg}

    The \emph{permuted congruential generator} is a modified \emph{linear congruential generator} proposed by Melissa E. O'Neill in \cite{ONeill:2014}.

    In contrast to a typical LCG, the PCG state has twice the width of its output, the modulus $m$ is $m = 2^k$ for $k \in \mathbb{N}$ and the output is generated by a state-defined bitwise rotation of the state.

    The following members of the PCG family of PRNGs were compared:
    \begin{itemize}
		\itemsep0em
        \item \textbf{PCG32}:           $a = 0x5851F42D4C957F2D$, $c = 0x14057B7EF767814F$, $m = 2^{31} - 1$, \SI{64}{\bit} state
        \item \textbf{PCG32Unique}:     \textit{PCG32} where $c$ is based on a memory address
        \item \textbf{PCG32Fast}:       \textit{PCG32} where $c = 0$
        \item \textbf{PCG32K2}:         2-dimensionally equidistributed version of \textit{PCG32}
        \item \textbf{PCG32K2Fast}:     2-dim. equidistributed version of \textit{PCG32Fast}
        \item \textbf{PCG32K64}:        64-dim. equidistributed version of \textit{PCG32}
        \item \textbf{PCG32K64Fast}:    64-dim. equidistributed version of \textit{PCG32Fast}
        \item \textbf{PCG32K1024}:      1024-dim. equidistributed version of \textit{PCG32}
        \item \textbf{PCG32K1024Fast}:  1024-dim. equidistributed version of \textit{PCG32Fast}
        \item \textbf{PCG32K16384}:     16384-dim. equidistributed version of \textit{PCG32}
        \item \textbf{PCG32K16384Fast}: 16384-dim. equidistributed version of \textit{PCG32Fast}
    \end{itemize}

\subsection[Lagged Fibonacci Generator (LFG) -- 1958]{Lagged Fibonacci Generator (LFG) -- 1958} \label{subsec:lfg}

    The \emph{lagged Fibonacci generator} is a family of PRNGs---based on the generalization of the Fibonacci sequence---proposed (but never published) by G. J. Mitchell and D. P. Moore in 1958.

    A LFG is defined by the following recurrence relation $X$:
    \begin{equation*}
        X_n = \left(X_{n - j} + X_{n - k}\right) \mod m, n \geq j \land n \geq k
    \end{equation*}
    In this definition $j = 24$ and $k = 55$ are the lags of the original proposal and $\left(X_0, ..., X_{\max\left(j, k\right)}\right)$ is the seed to be seeded e.g. based on another random number generator.

    The following (floating-point) members of the LFG family of PRNGs, which do not belong to the specializations defined in subsections, were compared:
    \begin{itemize}
		\itemsep0em
        \item \textbf{LaggedFibonacci607}: $j = 607$, $k = 273$, $m = 1$
        \item \textbf{LaggedFibonacci1279}: $j = 1279$, $k = 418$, $m = 1$
        \item \textbf{LaggedFibonacci2281}: $j = 2281$, $k = 1252$, $m = 1$
        \item \textbf{LaggedFibonacci3217}: $j = 3217$, $k = 576$, $m = 1$
        \item \textbf{LaggedFibonacci4423}: $j = 4423$, $k = 2098$, $m = 1$
        \item \textbf{LaggedFibonacci9689}: $j = 9689$, $k = 5502$, $m = 1$
        \item \textbf{LaggedFibonacci19937}: $j = 19937$, $k = 9842$, $m = 1$
        \item \textbf{LaggedFibonacci23209}: $j = 23209$, $k = 13470$, $m = 1$
        \item \textbf{LaggedFibonacci44497}: $j = 44497$, $k = 21034$, $m = 1$
        \item \textbf{RANMAR}: $X_n = \begin{cases}
                                          X_{n - 97} - X_{n - 33}     & \text{if } X_{n - 97} \geq X_{n - 33} \\
                                          X_{n - 97} - X_{n - 33} + 1 & \text{else}
                                      \end{cases} \bmod 1$ \\
                               combined with a simple arithmetic sequence as proposed by G. Marsaglia et al. in \cite{Marsaglia:1990} and modified by F. James in \cite{James:1990}
    \end{itemize}

    Many parameters (lags) used were suggested by R. P. Brent in \cite{Brent:1992}.

\subsubsection[Subtract-With-Borrow (SWB) -- 1991]{Subtract-With-Borrow (SWB) -- 1991} \label{subsubsec:swb}

    The \emph{subtract-with-borrow} generator is a modification of the \emph{lagged Fibonacci generator} proposed by George Marsaglia and Arif Zaman in \cite{Marsaglia:1991}.

    A SWB generator is defined by the following iterating function $f$:
    \begin{equation*}
        f\left(x_1, ..., x_j, c\right) = \begin{cases}
                                             \left(x_{j + 1 - k} - x_1 - c, 0\right)     & \text{if } x_{j + 1 - k} - x_1 - c \geq 0 \\
                                             \left(x_{j + 1 - k} - x_1 - c + b, 1\right) & \text{if } x_{j + 1 - k} - x_1 - c < 0
                                         \end{cases}
    \end{equation*}
    In this definition $X_n = f\left(X_n\right)$ is the generated sequence. The lags $j$, $k$ and the base $b$ must be chosen appropriately with $j > k$ and the initial seed vector $\left(x_1, ..., x_j, c\right)$ must be set e.g. based on another random number generator.

    The following members of the SWB family of PRNGs were compared:
    \begin{itemize}
        \itemsep0em
        \item \textbf{Ranlux24Base}:    $j = 24$, $k = 10$, $b = 2^{24} - 1$
        \item \textbf{Ranlux24}:        \textit{Ranlux24Base} discarding $200$ per $223$ generated numbers
        \item \textbf{Ranlux3}:         \textit{Ranlux24Base} discarding $199$ per $223$ generated numbers
        \item \textbf{Ranlux4}:         \textit{Ranlux24Base} discarding $365$ per $389$ generated numbers
        \item \textbf{Ranlux48Base}:    $j = 12$, $k = 5$, $b = 2^{48} - 1$
        \item \textbf{Ranlux48}:        \textit{Ranlux48Base} discarding $378$ per $389$ generated numbers
        \item \textbf{Ranlux64\_3}:     $j = 10$, $k = 24$, $b = 2^{48} - 1$ discarding $199$ per $223$ generated numbers
        \item \textbf{Ranlux64\_4}:     $j = 10$, $k = 24$, $b = 2^{48} - 1$ discarding $365$ per $389$ generated numbers
        \item \textbf{Ranlux3\_01}:     floating-point version of \textit{Ranlux3}
        \item \textbf{Ranlux4\_01}:     floating-point version of \textit{Ranlux4}
        \item \textbf{Ranlux64\_3\_01}: floating-point version of \textit{Ranlux64\_3}
        \item \textbf{Ranlux64\_4\_01}: floating-point version of \textit{Ranlux64\_4}
    \end{itemize}

    The \textit{Ranlux} family was proposed by M. Lüscher in \cite{Luescher:1993}.

\subsection[Linear Feedback Shift Register (LFSR) -- 1965]{Linear Feedback Shift Register (LFSR) -- 1965} \label{subsec:lfsr}

    The \emph{linear feedback shift register} PRNG is a family of PRNGs proposed by Robert C. Tausworthe in \cite{Tausworthe:1965}.

    LSFR PRNGs operate on a bit sequence $a = \left\{a_k\right\}$, defined as follows:
    \begin{equation*}
        a_k = c_1 \cdot a_{k - 1} + c_2 \cdot a_{k - 2} + ... + c_n \cdot a_{k - n} \mod 2
    \end{equation*}
    The parameters $c_i \in \left\{0, 1\right\}$ with $1 \leq i \leq n$ are fixed and $n$ is the bit width of the state.

    Based on this state, the random number $y_k$ is generated as follows:
    \begin{equation*}
        y_k = \sum_{t = 1}^{L} 2^{-t} \cdot a_{qk + r - t}
    \end{equation*}
    Here $L \leq n$ represents the bit width of the random number output, $q$ is the number of bit between two successive $y_k$ in $a_k$ ($q \geq L$) and $r$ is a random integer in the state interval $\left[0 .. 2^n - 1\right]$.

    In \cite{LEcuyer:1996}, Pierre L'Ecuyer proposed a specific PRNG as the combination of three LSFR generators using bitwise XOR operations.

    The following LFSR PRNGs were compared:
    \begin{itemize}
        \itemsep0em
        \item \textbf{Taus88}:
            \begin{tabular}[t]{lllll}
                $n_1 = 32$, &$c_1 = 2^{32} - 2$, &$L_1 = 32$, &$q_1 = 12$, &$r_1 = 18$ \\
                $n_2 = 32$, &$c_2 = 2^{32} - 2$, &$L_2 = 32$, &$q_2 = 4$,  &$r_2 = 27$ \\
                $n_3 = 32$, &$c_3 = 2^{32} - 2$, &$L_3 = 32$, &$q_3 = 17$, &$r_3 = 25$
            \end{tabular}
        \item \textbf{Hurd160}: LFSR with 32 \SI{5}{\bit} shift registers by W. J. Hurd in \cite{Hurd:1974}
        \item \textbf{Hurd288}: LFSR with 32 \SI{9}{\bit} shift registers by W. J. Hurd in \cite{Hurd:1974}
    \end{itemize}

\subsubsection[Mersenne Twister (MT) -- 1998]{Mersenne Twister (MT) -- 1998} \label{subsubsec:mt}

    The \emph{Mersenne Twister}---a twisted \emph{generalized feedback shift register} (\textbf{GFSR}) operating on a state matrix---was proposed by M. Matsumoto and T. Nishimura in \cite{Matsumoto:1998}. It is by far the most widely used general-purpose PRNG.

    A more detailed description of the design and internals of the MT would unfortunately go beyond the scope of this thesis.

    The following MTs---all in the original proposal of MT---were compared:
    \begin{itemize}
        \itemsep0em
        \item \textbf{MT19937}: Mersenne prime is $2^{19937} - 1$
        \item \textbf{MT19937-64}: Mersenne prime is $2^{19937} - 1$, \SI{64}{\bit} version
        \item \textbf{MT11213B}: Mersenne prime is $2^{11213} - 1$
    \end{itemize}

\subsubsection[Xorshift -- 2003]{Xorshift -- 2003} \label{subsubsec:xorshift}

    The \emph{xorshift}---a subtype of the \emph{linear feedback shift register}, which was implemented purely using fast bitwise XOR and shift operations---was proposed by George Marsaglia in \cite{Marsaglia:2003}.

    The following implementation of a \SI{32}{\bit} xorshift was given in \cite{Marsaglia:2003}:
\begin{@empty}
    \lstset{
        language = [ISO]C++
    }
\begin{centeredshadowboxlisting}
uint32_t xorshift32() {
    static uint32_t state = 2463534242;
    state ^= (state << 13);
    state = (state >> 17);
    return (state ^= (state << 5));
}
\end{centeredshadowboxlisting}
\end{@empty}
    \textcolor{black!75}{The initial \lstinline|state|---hard-coded in this example to 2463534242---should be randomly seeded in any real use case.}

    The following xorshift generators were compared:
    \begin{itemize}
        \itemsep0em
        \item \textbf{xorshift32}: \SI{32}{\bit} xorshift
        \item \textbf{xorshift64*}: \SI{64}{\bit} xorshift with truncated output
        \item \textbf{xorwow}: \SI{128}{\bit} xorshift combined with a Weyl sequence
        \item \textbf{xorshift128+}: \SI{128}{\bit} xorshift with \SI{64}{\bit} shifts (\cite{Vigna:2017})
    \end{itemize}

\subsubsection[Well Equidistributed Long-Period Linear (WELL) -- 2006]{Well Equidistributed Long-Period Linear (WELL) -- 2006} \label{subsubsec:well}

    The \emph{well equidistributed long-period linear} generators---a family of PRNGs in the form of GFSR and MT generators---was proposed by François Panneton et al. in \cite{Panneton:2006}.

    The WELL algorithm is as follows:
    \begin{algorithmic}[]
        \State $z_0 \leftarrow \left(m_p \land v_{i, r - 1}\right) \oplus \left(\widetilde{m}_p \land v_{i, r - 2}\right)$
        \State $z_1 \leftarrow T_0 \cdot v_{i, 0} \oplus T_1 \cdot v_{i, m_1}$
        \State $z_2 \leftarrow T_2 \cdot v_{i, m_2} \oplus T_3 \cdot v_{i, m_3}$
        \State $z_3 \leftarrow z_1 \oplus z_2$
        \State $z_4 \leftarrow T_4 \cdot z_0 \oplus T_5 \cdot z_1 \oplus T_6 \cdot z_2 \oplus T_7 \cdot z_3$
        \State $v_{i + 1, r - 1} \leftarrow v_{i, r - 2} \land m_p$
        \For {$j \leftarrow r - 2, 2$}
            \State $v_{i + 1, j} \leftarrow v_{i, j - 1}$
        \EndFor
        \State $v_{i + 1, 1} \leftarrow z_3$
        \State $v_{i + 1, 0} \leftarrow z_4$
        \State \textbf{return} $y_i = v_{i, 0}$
    \end{algorithmic}
    In the algorithm $w$ is the bit-width of the random numbers output by the WELL algorithm, $r \in \left(0.. \infty\right)$ and $p \in \left[0.. w\right)$ are unique integers and $m_p \in \left(0.. r\right)$ are bitmasks. The bit-width of the elements of the $r$-dimensional state vector $x_i$ is $w$ and the last $p$ bits of the last element of this vector are $0$. Possible values for the transformation $w{\times}w$-matrices $T_0, ..., T_7$ and further limitations to the parameters are given in \cite{Panneton:2006}.

    Shin Harase suggested a tempering method in \cite{Harase:2009} to make some WELL generators maximally equidistributed.

    The following WELL generators were compared:
    \begin{itemize}
        \itemsep0em
        \item \textbf{WELL512}: $w = 32$, $r = 16$, $p = 0$, $m_1 = 13$, $m_2 = 9$, $m_3 = 5$
        \item \textbf{WELL521}: $w = 32$, $r = 17$, $p = 23$, $m_1 = 13$, $m_2 = 11$, $m_3 = 10$
        \item \textbf{WELL607}: $w = 32$, $r = 19$, $p = 1$, $m_1 = 16$, $m_2 = 15$, $m_3 = 14$
        \item \textbf{WELL800}: $w = 32$, $r = 25$, $p = 0$, $m_1 = 14$, $m_2 = 18$, $m_3 = 17$
        \item \textbf{WELL1024}: $w = 32$, $r = 32$, $p = 0$, $m_1 = 3$, $m_2 = 24$, $m_3 = 10$
        \item \textbf{WELL19937}: $w = 32$, $r = 624$, $p = 31$, $m_1 = 70$, $m_2 = 179$, $m_3 = 449$
        \item \textbf{WELL21701}: $w = 32$, $r = 679$, $p = 27$, $m_1 = 151$, $m_2 = 327$, $m_3 = 84$
        \item \textbf{WELL23209}: $w = 32$, $r = 726$, $p = 23$, $m_1 = 667$, $m_2 = 43$, $m_3 = 462$
        \item \textbf{WELL44497}: $w = 32$, $r = 1391$, $p = 15$, $m_1 = 23$, $m_2 = 481$, $m_3 = 229$
        \item \textbf{WELL800-ME}: \textit{WELL800} with $y_i \leftarrow v_{i, 0} \oplus \left(v_{i, 19} \land 0x4880\right)$
        \item \textbf{WELL19937-ME}: \textit{WELL19937} with $y_i \leftarrow v_{i, 0} \oplus \left(v_{i, 180} \land 0x4118000\right)$
        \item \textbf{WELL21701-ME}: \textit{WELL21701} with $y_i \leftarrow v_{i, 0} \oplus \left(v_{i, 328} \land 0x1002\right)$
        \item \textbf{WELL23209-ME}: \textit{WELL23209} with $y_i \leftarrow v_{i, 0} \oplus \left(v_{i, 44} \land 0x5100000\right)$
        \item \textbf{WELL44497-ME}: \textit{WELL44497} with $y_i \leftarrow v_{i, 0} \oplus \left(v_{i, 482} \land 0x48000000\right)$
    \end{itemize}

\subsubsection[Xoshiro -- 2018]{Xoshiro -- 2018} \label{subsubsec:xoshiro}

    The \emph{xoshiro}---a \emph{linear feedback shift register} generator implemented using XOR, shift and rotate operations---was---along with xoroshiro---proposed by David Blackman and Sebastiano Vigna in \cite{Blackman:2018}.

    The following (slightly modified) implementation of a \SI{32}{\bit} xoshiro with a \SI{128}{\bit} state was given in \cite{Blackman:2018}:
\begin{@empty}
    \lstset{
        language = [ISO]C++
    }
\begin{centeredshadowboxlisting}
void xoshiro128() {
    static uint32_t s0_ = 0x01d353e5f3993bb1;
    static uint32_t s1_ = 0xf7381bed96327640;
    static uint32_t s2_ = 0xfdfcaa91110765b5;
    static uint32_t s3_ = 0x0;
    const uint64_t t = s1_ << a;
    s2_ ^= s0_;
    s3_ ^= s1_;
    s1_ ^= s2_;
    s0_ ^= s3_;
    s2_ ^= t;
    s3_ = (s3_ << b) | (s3_ >> (32 - b));
}
\end{centeredshadowboxlisting}
\end{@empty}
    \textcolor{black!75}{The initial \lstinline|s0_|, \lstinline|s1_|, \lstinline|s2_| and \lstinline|s3_|, which are the state---hard-coded in this example to 0x01d353e5f3993bb1, 0xf7381bed96327640, 0xfdfcaa91110765b5 and 0x0---should be randomly seeded in any real use case.} For the \SI{32}{\bit} case, the authors suggested shift and rotate values to be $\texttt{\small a} = 9$ and $\texttt{\small b} = 11$.

    It is easy to see in the implementation that xoshiro does not define the generation of a pseudorandom number from its state. The authors proposed four scramblers to be used with xoshiro (and xoroshiro) where the two more advanced ones try to eliminate linear artifacts from the state.
    
    \begin{itemize}
        \itemsep0em
        \item[\textbf{$\mathbf{+}$ scrambler}]  The simple $\mathbf{+}$ scrambler returns just the sum of two of the state words (e.g. \lstinline|return s0_ + s3_|).
        \item[\textbf{$\mathbf{*}$ scrambler}]  The not less simple $\mathbf{*}$ scrambler returns just the product of one of the state words with a fixed, odd multiplier (e.g. \lstinline|return s1_ + mult|).
        \item[\textbf{$\mathbf{++}$ scrambler}] The $\mathbf{++}$ scrambler first adds two of the state words, rotates the sum by \lstinline|r| positions to the left and returns the sum of this rotated sum and the first of the two state words used in the first sum (e.g. \lstinline{return ((s0_+s3_) << r) | ((s0_+s3_) >> (32-r)) + s0_}). The authors propose \lstinline|r = 7| in the \SI{32}{\bit} case.
        \item[\textbf{$\mathbf{**}$ scrambler}] The $\mathbf{**}$ scrambler first multiplies one of the state words by a fixed, odd multiplier \lstinline|s|, rotates the product by \lstinline|r| positions to the left and returns the product of this rotated product and another fixed, odd multiplier \lstinline|t| (e.g. \lstinline{return ((s1_*s) << r) | ((s1_*s) >> (32-r)) * t}). The authors propose \lstinline|s = 5|, \lstinline|r = 7| and \lstinline|t = 9| in the \SI{32}{\bit} case.
    \end{itemize}

    The following xoshiro generators were compared:
    \begin{itemize}
        \itemsep0em
        \item \textbf{xoshiro128$\mathbf{+}$32}: \SI{32}{\bit} \emph{xoshiro} with \SI{128}{\bit} state and $\mathbf{+}$ scrambler
        \item \textbf{xoshiro128$\mathbf{**}$32}: \SI{32}{\bit} \emph{xoshiro} with \SI{128}{\bit} state and $\mathbf{**}$ scrambler
    \end{itemize}

\subsubsection[Xoroshiro -- 2018]{Xoroshiro -- 2018} \label{subsubsec:xoroshiro}

    The \emph{xoroshiro}---another \emph{linear feedback shift register} generator implemented using XOR, shift and rotate operations---was proposed by David Blackman and Sebastiano Vigna in \cite{Blackman:2018}.

    The following (slightly modified) implementation of a \SI{64}{\bit} xoroshiro with a \SI{128}{\bit} state was given in \cite{Blackman:2018}:
\begin{@empty}
    \lstset{
        language = [ISO]C++
    }
\begin{centeredshadowboxlisting}
void xoroshiro128() {
    static uint64_t s0_ = 0xc1f651c67c62c6e0;
    static uint64_t s1_ = 0x30d89576f866ac9f;
    const uint64_t t = s0_ ^ s1_;
    s0_ = ((s0_ << a) | (s0_ >> (64 - a)))
        ^ t ^ (t << b);
    s1_ = (t << c) | (t >> (64 - c));
}
\end{centeredshadowboxlisting}
\end{@empty}
    \textcolor{black!75}{The initial \lstinline|s0_| and \lstinline|s1_| which are the state---hard-coded in this example to 0xc1f651c67c62c6e0 and 0x30d89576f866ac9f---should be randomly seeded in any real use case.} For the \SI{64}{\bit} case, the authors proposed shift and rotate values to be $\texttt{\small a} = 24$, $\texttt{\small b} = 16$ and $\texttt{\small c} = 37$.

        To generate pseudo-random numbers from the states of a xoroshiro generator, the scramblers $\mathbf{+}$, $\mathbf{*}$, $\mathbf{++}$ and $\mathbf{**}$, which are also used for xoshiro, are used. The details of these scramblers are described in the subsection \ref{subsubsec:xoshiro}.

    The following xoroshiro generators were compared:
    \begin{itemize}
        \itemsep0em
        \item \textbf{xoroshiro128$\mathbf{+}$32}: \SI{64}{\bit} \emph{xoroshiro} with \SI{128}{\bit} state and $\mathbf{+}$ scrambler
        \item \textbf{xoroshiro64$\mathbf{+}$32}:  \SI{32}{\bit} \emph{xoroshiro} with \SI{64}{\bit} state and $\mathbf{+}$ scrambler
        \item \textbf{xoroshiro64$\mathbf{*}$32}:  \SI{32}{\bit} \emph{xoroshiro} with \SI{64}{\bit} state and $\mathbf{*}$ scrambler
        \item \textbf{xoroshiro64$\mathbf{**}$32}: \SI{32}{\bit} \emph{xoroshiro} with \SI{64}{\bit} state and $\mathbf{**}$ scrambler
    \end{itemize}

\subsection[Inversive Congruential Generator (ICG) -- 1986]{Inversive Congruential Generator (ICG) -- 1986} \label{subsec:icg}

    The \emph{inversive congruential generator} is a family of PRNGs proposed by Jürgen Eichenauer and Jürgen Lehn in \cite{Eichenauer:1986}.

    An ICG is defined by the following recurrence relation $X$:
    \begin{equation*}
        X_{n + 1} = \begin{cases}
                        \left(a \cdot X_n^{-1} + b\right) \bmod p & \text{if } X_n \geq 1 \\
                        b                                         & \text{else}
                    \end{cases}
    \end{equation*}
    In this definition $a \in \mathbb{N}$ is the multiplier, $b \in \mathbb{N}$ is the increment, $p$ is the prime modulus and $X_0 \in \left[0.. p\right)$ is the seed. $X_n^{-1}$ is the multiplicative inverse of $X_n$ in the finite field $GF\left(p\right)$.

    The following member of the ICG family of PRNGs was compared:
    \begin{itemize}
        \itemsep0em
        \item \textbf{Hellekalek1995}: $a = 0x238E$, $b = 0x7DCD313A$, $p = 0x7FFFFFFF$ as proposed by Peter Hellekalek in \cite{Hellekalek:1995}
    \end{itemize}

\subsection[Ranshi -- 1995]{Ranshi -- 1995} \label{subsec:ranshi}

    The \emph{ranshi} algorithm is a PRNG proposed by F. Gutbrod in \cite{Gutbrod:1995}.

    The idea behind the algorithm is a physical system consisting of a number of black balls, each of which has a position and a spin (state of the PRNG). A red ball---also having a spin and a position---colliding with the black balls is used to generate pseudorandom numbers.

\subsection[Gjrand -- 2005]{Gjrand -- 2005} \label{subsec:gjrand}

    The \emph{gjrand} algorithm is based on a random, invertible mapping\footnote{\url{http://www.pcg-random.org/posts/random-invertible-mapping-statistics.html}} of addition, XOR and rotate operations. It was proposed by David Blackman\footnote{\url{http://gjrand.sourceforge.net/}}.

    The following member of the gjrand family of PRNGs was compared:
    \begin{itemize}
        \itemsep0em
        \item \textbf{gjrand32}: \SI{32}{\bit} PRNG with \SI{128}{\bit} state, parameters from the author
    \end{itemize}

\subsection[A Small Noncryptographic PRNG (JSF) -- 2007]{A Small Noncryptographic PRNG (JSF) -- 2007} \label{subsec:jsf}

    The \emph{JSF} algorithm is based on a reversible, non-linear function in which all internal state bits affect one another using addition, XOR, rotate and conditional branch operations. It was proposed by Bob Jenkins\footnote{\url{http://burtleburtle.net/bob/rand/smallprng.html}}.

    The following implementation (with \lstinline|b_ = c_ = d_| properly seeded) of a \SI{32}{\bit} JSF was used:
\begin{@empty}
    \lstset{
        language = [ISO]C++
    }
\begin{centeredshadowboxlisting}
uint32_t jsf32() {
    static uint32_t a_ = 0xf1ea5eed;
    static uint32_t b_ = 0xcafe5eed00000001;
    static uint32_t c_ = 0xcafe5eed00000001;
    static uint32_t d_ = 0xcafe5eed00000001;
    uint32_t e = a_ - ((b_ << p)
                     | (b_ >> (32 - p)));
    a_ = b_ ^ ((c_ << q) | (c_ >> (32 - q)));
    b_ = c_ + (r ? ((d_ << r)
                  | (d_ >> (32 - r))) : d_);
    c_ = d_ + e;
    d_ = e + a_;
    return d_;
}
\end{centeredshadowboxlisting}
\end{@empty}

    The following members of the JSF family of PRNGs were compared:
    \begin{itemize}
        \itemsep0em
        \item \textbf{JSF32n}: \lstinline|p = 27|, \lstinline|q = 17|, \lstinline|r = 0|
        \item \textbf{JSF32r}: \lstinline|p = 23|, \lstinline|q = 16|, \lstinline|r = 11|
    \end{itemize}

\subsection[SFC -- 2010]{SFC -- 2010} \label{subsec:sfc}

    The \emph{SFC} algorithm is based on a random, invertible mapping\footnote{\url{http://www.pcg-random.org/posts/random-invertible-mapping-statistics.html}} of addition, XOR, shift and rotate operations. It was proposed by Chris Doty-Humphrey as part of his PractRand\footnote{\url{http://pracrand.sourceforge.net/}} statistical test and PRNG library.

    The following implementation (with \lstinline|a_|, \lstinline|b_| and \lstinline|c_| properly seeded) of a \SI{32}{\bit} SFC was used:
\begin{@empty}
    \lstset{
        language = [ISO]C++
    }
\begin{centeredshadowboxlisting}
uint32_t sfc32() {
    static uint32_t a_ = 0xcafef00dbeef5eed;
    static uint32_t b_ = 0xcafef00dbeef5eed;
    static uint32_t c_ = 0xcafef00dbeef5eed;
    static uint32_t d_ = 0x1;
    uint32_t t = a_ + b_ + d_++;
    a_ = b_ ^ (b_ >> q);
    b_ = c_ + (c_ << r);
    c_ = (c_ << p) | (c_ >> (64 - p)) + t;
    return t;
}
\end{centeredshadowboxlisting}
\end{@empty}

    The following member of the SFC family of PRNGs was compared:
    \begin{itemize}
        \itemsep0em
        \item \textbf{SFC32}: \lstinline|p = 21|, \lstinline|q = 9|, \lstinline|r = 3|
    \end{itemize}

\subsection[Counter-Based Random Number Generator (CBRNG) -- 2011]{Counter-Based Random Number Generator (CBRNG) -- 2011} \label{subsec:cbrng}

    The \emph{counter-based random number generator} is a family of PRNGs proposed by J. Salmon et al. in \cite{Salmon:2011}.

    The state of a CBRNG is a simple integer counter but the output mapping is done using a complex function---usually a cryptographic block cipher.

\subsubsection[ARC4 -- 1997]{ARC4 -- 1997} \label{subsubsec:arc4}

    The \emph{ARC4} is a PRNG first implemented in OpenBSD 2.1\footnote{\url{https://man.openbsd.org/arc4random}} in 1997 for the \lstinline|arc4random| function.

    It generates pseudorandom numbers from the keystream of the RC4 stream cipher, released by Ronald L. Rivest in 1987. ARC4 is not exactly a CBRNG since it uses a second state which is not a counter, but the PRNG is closely related to the other CBRNGs since it uses just a stream cipher to generate pseudorandom numbers.

\subsubsection[ChaCha -- 2008]{ChaCha -- 2008} \label{subsubsec:chacha}

    \emph{ChaCha} is a stream cipher proposed by Daniel J. Bernstein in \cite{Bernstein:2008}. It is used as PRNG by encoding the state of the PRNG---a simple integer counter---using the ChaCha stream cipher.

    The following PRNGs based on the family of ChaCha stream ciphers were compared:
    \begin{itemize}
        \itemsep0em
        \item \textbf{ChaCha4}: Based on ChaCha 4-round cipher
        \item \textbf{ChaCha5}: Based on ChaCha 5-round cipher
        \item \textbf{ChaCha6}: Based on ChaCha 6-round cipher
        \item \textbf{ChaCha8}: Based on ChaCha 8-round cipher
        \item \textbf{ChaCha20}: Based on ChaCha 20-round cipher
    \end{itemize}

\subsubsection[Advanced Randomization System (ARS) -- 2011]{Advanced Randomization System (ARS) -- 2011} \label{subsubsec:ars}

    The \emph{advanced randomization system} is a \emph{counter-based random number generator} where the state---a simple integer counter---is mapped to the random output using a simplified AES block cipher.

    The following ARS generator was compared:
    \begin{itemize}
		\itemsep0em
        \item \textbf{ARS4x32}: 7 rounds, operating on four \SI{32}{\bit} integers
    \end{itemize}

\subsubsection[Threefry -- 2011]{Threefry -- 2011} \label{subsubsec:threefry}

    The \emph{Threefry} is a \emph{counter-based random number generator} where the state is mapped to the random output using a simplified Threefish block cipher.

    The following Threefry generators have been compared:
    \begin{itemize}
		\itemsep0em
        \item \textbf{Threefry2x32}: 20 rounds, operating on two  \SI{32}{\bit} integers
        \item \textbf{Threefry4x32}: 20 rounds, operating on four \SI{32}{\bit} integers
        \item \textbf{Threefry2x64}: 20 rounds, operating on two  \SI{64}{\bit} integers
        \item \textbf{Threefry4x64}: 20 rounds, operating on four \SI{64}{\bit} integers
    \end{itemize}

\subsubsection[Philox -- 2011]{Philox -- 2011} \label{subsubsec:philox}

    The \emph{Philox} is a \emph{counter-based random number generator} where the state is mapped to the random output using a custom block cipher.

    The following Philox generators were compared:
    \begin{itemize}
		\itemsep0em
        \item \textbf{Philox2x32}: 10 rounds, operating on two  \SI{32}{\bit} integers
        \item \textbf{Philox4x32}: 10 rounds, operating on four \SI{32}{\bit} integers
        \item \textbf{Philox2x64}: 10 rounds, operating on two  \SI{64}{\bit} integers
        \item \textbf{Philox4x64}: 10 rounds, operating on four \SI{64}{\bit} integers
    \end{itemize}

\subsubsection[Advanced Encryption Standard (AES) -- 2011]{Advanced Encryption Standard (AES) -- 2011} \label{subsubsec:aes}

    The \emph{Advanced Encryption Standard} PRNG is a \emph{counter-based random number generator} where the state is mapped to the random output using the AES block cipher.

    The following AES generator was compared:
    \begin{itemize}
		\itemsep0em
        \item \textbf{AES4x32}: 10 rounds, operating on four \SI{32}{\bit} integers
    \end{itemize}

\subsection[SplitMix -- 2014]{SplitMix -- 2014} \label{subsec:splitmix}

    \emph{SplitMix} is a PRNG similar to the \emph{counter-based random number generators} that was proposed by Guy Steele et al. in \cite{Steele:2014}. It is derived from the PRNG \emph{DotMix} which was proposed by Charles Leiserson et al. in \cite{Leiserson:2012}.

    While the state of CBRNGs is advanced by adding $1$---it is a simple counter---the state of \emph{SplitMix} is advanced by adding a fixed $\gamma$. Instead of using a complex hash function for the generation of a pseudorandom integer from the state, \emph{SplitMix} uses the finalization mix of the MurmurHash3\footnote{\url{https://bit.ly/2tI7IqW}} hash function. This is sufficient as long as $\gamma$ is not a simple value like $1$, even or some other problematic value.

    The following implementation (with \lstinline|state| and \lstinline|gamma| properly seeded) of \SI{32}{\bit} \emph{SplitMix} was used:
\begin{@empty}
    \lstset{
        language = [ISO]C++
    }
\begin{centeredshadowboxlisting}
uint32_t splitmix32() {
    static uint64_t state = 0xbad0ff1ced15ea5e;
    static uint64_t gamma = 0x9e3779b97f4a7c15
                          | 1;
    uint64_t seed = state;
    state += gamma;
    seed ^= seed >> v;
    seed *= m5;
    seed ^= seed >> w;
    seed *= m6;
    return result_type(seed >> 32);
}
\end{centeredshadowboxlisting}
\end{@empty}
    The \lstinline{| 1} after the seed of \lstinline|gamma| takes care of even \lstinline|gamma|s, which would degrade the quality of the generated pseudorandom numbers.

    The \emph{SplitMix} PRNG used for the evaluation---\textbf{SplitMix32}---uses the following parameters: \lstinline|m5 = 0x62a9d9ed799705f5|, \lstinline|m6 = 0xcb24d0a5c88c35b3|, \lstinline|v = 33| and \lstinline|w = 28|.

\subsection[Combinations of different PRNGs]{Combinations of different PRNGs} \label{subsec:combination}

    The following combined PRNGs were compared:
    \begin{itemize}
        \itemsep0em
        \item \textbf{DualRand}: LCG with $a = 0x10405$, $c = 0x3035$, $m = 2^{32} - 1$ XORed with a LFSR approximated by $X_n = X_{n - 1 \bmod 64} \oplus X_{n - 33 \bmod 64} \bmod 2$ on a \SI{128}{\bit} state
        \item \textbf{TripleRand}: \textit{DualRand} XORed with \textit{Hurd288}
    \end{itemize}

\subsection[Biased Uniform Int Distribution]{Biased Uniform Integer Distribution} \label{subsec:distribution}

    The PRNGs listed above return pseudorandom integers in the range of $0$ to $2^{32} - 1$ or any other arbitrary range. But for the use in a RANDOM page eviction algorithm, the numbers need to be in the range of the buffer frame indices.

    This requires an algorithm that transforms pseudorandom numbers uniformly distributed in a given range to pseudorandom numbers in the desired range, keeping the uniform distribution. A blog post\footnote{\url{http://www.pcg-random.org/posts/bounded-rands.html}} by Melissa E. O'Neill revealed that this transformation is the bottleneck of fast random number generation when using the \lstinline|std::uniform_int_distribution| from the \textit{C++ Standard Library}.

    Therefore, the algorithm that turned out to be the fastest in her comparison was used when evaluating PRNGs for RANDOM page eviction. In contrast to the algorithm built into the \textit{C++ Standard Library}, this algorithm returns pseudorandom numbers in a biased uniform distribution when the range of the PRNG used is not a multiple of the range of the buffer frame indices. For example, if the PRNG returns numbers uniformly distributed in the integer interval $\left[1 .. 6\right]$ and if the buffer frame indices are in the interval $\left[1 .. 4\right]$, this algorithm returns $1$ and $2$ with a probability of $\sfrac{1}{3}$ and $3$ and $4$ with a probability of $\sfrac{1}{6}$. But as long as the range of buffer frame indices is much smaller than the range of the used PRNG, the bias is less severe.

    For a PRNG that returns random numbers in the range of $0$ to $2^{32} - 1$, the algorithm is as follows:
\begin{@empty}
    \lstset{
        language = [ISO]C++
    }
\begin{centeredshadowboxlisting}
uint32_t biased_int_dist(uint32_t ranNum,
                         uint32_t rangeMin,
                         uint32_t rangeMax) {
    uint64_t r = uint64_t(rangeMax - rangeMin);
    uint64_t m = uint64_t(ranNum) * (r + 1);
    return uint32_t(rangeMin + (m >> 32);
}
\end{centeredshadowboxlisting}
\end{@empty}
    The actual implementation used---available on GitHub\footnote{\url{https://bit.ly/37RQQNx}}---works with PRNGs returning both integers in any range as well as floating point numbers. It uses metaprogramming to utilize compile-time calculation wherever possible.

\section[Performance Evaluation]{Performance Evaluation} \label{sec:random-performance}

    Benchmarks of an exemplary DBMS using all the compared RANDOM page eviction implementations revealed that there is no statistically significant difference in the hit rates achieved with these different RANDOM page eviction implementations. The only performance difference between the different RANDOM page eviction implementations is therefore the overhead that results from the generation of pseudorandom numbers. For this reason, a microbenchmark only measuring the execution time of the PRNGs is appropriate.

    The only variable of the evaluation is the PRNG used---the alternatives presented in the previous section are evaluated.

    Another potential variable is the number of threats generating pseudorandom numbers. But the behavior of the PRNGs when used concurrently is usually not specified and therefore all the PRNGs would require synchronization when used by multiple evicting threads. However, since the quality of the generated pseudo-random numbers is not important here, it is assumed that each evicting thread uses its own thread-local instance of the used PRNG to choose candidates for eviction. And these thread-local instances scale perfectly as long as there are hardware threads available and therefore an evaluation on one thread is sufficient.

    Different algorithms to generate the pseudorandom integers uniformly distributed over a given range could also be compared. But a quick comparison of the custom algorithm presented in subsection \ref{subsec:distribution} with the ones provided by the \textit{C++ Standard Library}\footnote{\url{https://bit.ly/39Xuiwn}} and by the \textit{Boost Random Number Library}\footnote{\url{https://bit.ly/37JKHTs}} showed that the one\footnote{The classic modulo algorithm was used for \textbf{rand}, \textbf{xorwow} and \textbf{xorshift128+}.} used is never slower than the competition.

    The smallest ($>0$) and largest integers returned by the PRNGs---re\-pre\-sen\-ting the smallest and largest buffer pool indexes---do not significantly affect the performance of RANDOM page eviction. Therefore, this integer interval $\left[1 .. 53467\right]$ is a constant in this evaluation.

\begin{@empty}
    \nottoggle{bwmode}{
        \tikzset{%
            bar/.style = {draw = blue, fill = blue!25}
        }
    }{
        \tikzset{%
            bar/.style = {draw = black, fill = black!25}
        }
    }

    \begin{figure}[p]
        \centering
        \resizebox{\textwidth}{!}{
            \begin{tikzpicture}[]
                \begin{axis}[xbar,
                             xmin = 0,
                             width = 1.25\textwidth,
                             height = 1.325\textheight,
                             enlarge y limits = 0.025,
                             xlabel = {$\text{Average index generations }\left[\si{\indexes\per\second}\right]$},
                             xmin = 0,
                             xmax = 1290000000,
                             symbolic y coords = {MT11213B, MT19937-64, MT19937, Hurd288, Hurd160, Taus88, Ranlux64\_4\_01, Ranlux64\_3\_01, Ranlux4\_01, Ranlux3\_01, Ranlux64\_4, Ranlux64\_3, Ranlux48, Ranlux48Base, Ranlux4, Ranlux3, Ranlux24, Ranlux24Base, RANMAR, LaggedFibonacci44497, LaggedFibonacci23209, LaggedFibonacci19937, LaggedFibonacci9689, LaggedFibonacci4423, LaggedFibonacci3217, LaggedFibonacci2281, LaggedFibonacci1279, LaggedFibonacci607, PCG32K16384Fast, PCG32K16384, PCG32K1024Fast, PCG32K1024, PCG32K64Fast, PCG32K64, PCG32K2Fast, PCG32K2, PCG32Fast, PCG32Unique, PCG32, MixMax2.0, KnuthB, MINSTD, MINSTD0, RANECU, MCG128Fast, MCG128, Kreutzer1986, rand48, rand},
                             ytick = data,
                             nodes near coords,
                             nodes near coords align = {horizontal}]
                    \addplot[bar] table[x = throughput, y = prng] {./data/random/throughput_top.csv};
                \end{axis}
            \end{tikzpicture}
        }
        \caption{The index generation throughput of the evaluated RANDOM implementations (1 of 2)}
        \label{fig:random_performance_top}
    \end{figure}

    \begin{figure}[p]
        \centering
        \resizebox{\textwidth}{!}{
            \begin{tikzpicture}[]
                \begin{axis}[xbar,
                             xmin = 0,
                             width = 1.25\textwidth,
                             height = 1.325\textheight,
                             enlarge y limits = 0.025,
                             xlabel = {$\text{Average index generations }\left[\si{\indexes\per\second}\right]$},
                             xmin = 0,
                             xmax = 1290000000,
                             symbolic y coords = {TripleRand, DualRand, SplitMix32, AES4x32, Philox4x64, Philox2x64, Philox4x32, Philox2x32, Threefry4x64, Threefry2x64, Threefry4x32, Threefry2x32, ARS4x32, ChaCha20, ChaCha8, ChaCha6, ChaCha5, ChaCha4, ARC4, SFC32, JSF32r, JSF32n, gjrand32, ranshi, Hellekalek1995, Xoroshiro64**32, Xoroshiro64*32, Xoroshiro64+32, Xoroshiro128+32, Xoshiro128**32, Xoshiro128+32, WELL44497-ME, WELL23209-ME, WELL21701-ME, WELL19937-ME, WELL800-ME, WELL44497, WELL23209, WELL21701, WELL19937, WELL1024, WELL800, WELL607, WELL521, WELL512, xorshift128+, xorwow, xorshift64*, xorshift32},
                             ytick = data,
                             nodes near coords,
                             nodes near coords align = {horizontal}]
                    \addplot[bar] table[x = throughput, y = prng] {./data/random/throughput_bottom.csv};
                \end{axis}
            \end{tikzpicture}
        }
        \caption{The index generation throughput of the evaluated RANDOM implementations (2 of 2)}
        \label{fig:random_performance_bottom}
    \end{figure}
\end{@empty}

\subsection[Microbenchmark]{Microbenchmark}

    The microbenchmark used for the performance evaluation of the RANDOM page eviction algorithms instantiates the evaluated PRNG with a seed---generated with \lstinline|std::random_device|\footnote{\url{https://bit.ly/306x2TE}}---, calculates a given number (\num{5000000}) of pseudorandom integers in the interval $\left[1 .. 53467\right]$ using the PRNG and measures the wall time elapsed.

\subsection[System Configuration]{Configuration of the Used System}

\begin{@empty}
    \begin{itemize}
        \itemsep0em
		\item	\textbf{CPU:} \emph{Intel® Core™ i7-8700} @$12 \times \SI{3.2}{\giga\hertz}$ from late 2017
        \item	\textbf{Main Memory:} $2 \times 8\text{GB} = 16\text{GB}$ of DDR4-SDRAM @\SI{2666}{\mega\hertz}
        \item	\textbf{OS:} \emph{Ubuntu 19.10}
        \item	\textbf{Compiler: } \emph{GCC 9.2.1} with \texttt{-O3} flag
    \end{itemize}
\end{@empty}

\subsection[Benchmark Results]{Benchmark Results}

    The figures \ref{fig:random_performance_top} and \ref{fig:random_performance_bottom} show the index generation throughput of the evaluated RANDOM page eviction implementations.

    The \emph{ICG} \textbf{Hellekalek1995} and the \emph{SWB}s of the \textbf{Ranlux} family (not the non-discarding ``base'' ones) are the slowest PRNGs in the evaluation. The XOR-based \emph{LSFR} PRNGs and the \emph{LCG}s of the \textbf{PCG} and \textbf{MCG} families are among the fastest PRNGs. The very recent \textbf{SplitMix32} PRNG, described in subsection \ref{subsec:splitmix}, is by far the fastest algorithm in the competition with an average of \SI{1124474870}{\indexes\per\second} on the used system.

\section{Conclusion}

    Like any other DBMS component evaluated for this thesis, the performance (the overhead of the eviction candidate selection, not the achieved hit rate) of the RANDOM eviction algorithm is not critical in most cases. However, the hit rate achieved with the chosen eviction strategy is a significant performance factor of a DB system (\textbf{DBS}).

    When it comes to the selection of a page eviction strategy, the RANDOM eviction strategy is the worst but simplest option. Its simplicity makes the RANDOM page eviction strategy a good choice if the main memory is expected to be (almost) always large enough to hold the entire working set of a DBS. In this case, the RANDOM eviction strategy would not perform (much) worse than any other ``good'' page eviction strategy.

    If the RANDOM page eviction strategy is chosen for the use in a buffer pool manager of a DBMS, there is basically no reason not to use the fastest PRNG available in the particular programming language. The fastest PRNG in this comparison---\emph{SplitMix}---is part of the Java Development Kit and good implementations are also available in Haskell and C++. The somewhat slower XOR-based \emph{LSFR} PRNGs are available for many programming languages, and the mid-range PRNG \textbf{MT19937} is the default PRNG of most of the programming languages. Most of the fast general-purpose PRNGs can be easily implemented in any programming language typically used in the development of a DBMS, and therefore the lack of such a PRNG in a particular programming language can quickly be compensated.

	
	% glossary
	
	\sloppy
	
	\printbibliography			% generates the bibliography
				[	
					env = bibliography,			% select the environment to control the layout of the bibliography
					heading = bibliography,		% select the heading of the bibliography, alternatives can be defined using \defbibheading
%					title = Bibliography,			% select the title of the bibliography
%					prenote = prenote,			% select a note to be printed directly	after the heading of the bibliography defined by \defbibnote
%					postnote = postnote,			% select a note to be printed directly	after the list of references of the bibliography defined by \defbibnote
%					filter = bibfilter				% select a filter as defined in \defbibfilter
				]
		
\end{document}
